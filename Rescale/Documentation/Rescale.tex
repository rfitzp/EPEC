\documentclass[12pt]{article}
\usepackage{amsmath}
\usepackage{fullpage}
\def\vecsign{\mathchar"017E}
\def\dvecsign{\smash{\stackon[-2.3pt]{\vecsign}{\rotatebox{180}{$\vecsign$}}}}
\def\dvec#1{\def\useanchorwidth{T}\stackon[-4.2pt]{#1}{\,\dvecsign}}
\usepackage{stackengine}
\stackMath
\usepackage{graphicx}
\allowdisplaybreaks

\title{\bf Program {\sc RESCALE}}
\date{\today}
\author{Richard Fitzpatrick}

\begin{document}
\maketitle

\section{Rescaling gFile}
\subsection{Grad-Shafranov Equation}
The Grad-Shafranov equation takes the form
\begin{equation}
\frac{\partial^{\,2}\psi}{\partial R^{\,2}}-\frac{1}{R}\,\frac{\partial\psi}{\partial R} + \frac{\partial^{\,2}\psi}{\partial Z^{\,2}} = -R^{\,2}\,\frac{dP}{d\psi} - \frac{1}{2}\,\frac{dT^{\,2}}{d\psi},
\end{equation}
where
\begin{align}
j_\phi &= -R\,\frac{dP}{d\psi}- \frac{1}{2\,R}\,\frac{dT^{\,2}}{d\psi},\\[0.5ex]
q &= \frac{T}{2\pi}\oint_\psi \frac{dl}{R\,|\nabla \psi|}.
\end{align}

\subsection{Type I Rescaling}
The following rescaling of variables leaves the Grad-Shafranov equation invariant:
\begin{align}
R_{\rm new}&=R_{\rm old},\\[0.5ex]
Z_{\rm new}&=Z_{\rm old},\\[0.5ex]
\psi_{\rm new} &= a_1\,\psi_{\rm old},\\[0.5ex]
T_{\rm new} &=a_1\,T_{\rm old},\\[0.5ex]
P_{\rm new} &=a_1^{\,2}\,P_{\rm old}.
\end{align}
It follows that
\begin{align}
I_{\phi\,{\rm new}} &= a_1\,I_{\phi\,{\rm old}},\\[0.5ex]
q_{\rm new} &= q_{\rm old}.
\end{align} 

\subsection{Type II Rescaling}
The following rescaling of variable also leaves the Grad-Shafranov equation invariant:
\begin{align}
R_{\rm new}&=R_{\rm old},\\[0.5ex]
Z_{\rm new}&=Z_{\rm old},\\[0.5ex]
\psi_{\rm new} &=\psi_{\rm old},\\[0.5ex]
T_{\rm new} &={\rm sgn}(T_{\rm old})\sqrt{T_{\rm old}^{\,2}+a_2},\\[0.5ex]
P_{\rm new} &=P_{\rm old}.
\end{align}
It follows that
\begin{align}
I_{\phi\,{\rm new}} &=I_{\phi\,{\rm old}},\\[0.5ex]
q_{\rm new} &= \frac{T_{\rm new}}{T_{\rm old}}\,q_{\rm old}.
\end{align} 

\subsection{Program {\sc RESCALE}}
Let
\begin{equation}
{\mit\Psi}_N = \frac{\psi - \psi_{\rm axis}}{\psi_{\rm separatrix}-\psi_{\rm axis}}.
\end{equation}

Program {\sc RESCALE} first performs a type II rescaling such that 
\begin{equation}
a_2 = \left(\frac{q_{95\,{\rm target}}^{\,2}}{q_{95\,{\rm old}}^{\,2}}-1\right)T_{95\,{\rm old}}^{\,2},
\end{equation}
where
\begin{align}
q_{95} &\equiv q({\mit\Psi}_N=0.95),\\[0.5ex]
T_{95}&\equiv T({\mit\Psi}_N=0.95).
\end{align}
It follows that
\begin{align}
\psi_{\rm new} &= \psi_{\rm old},\\[0.5ex]
T_{\rm new} & =T_{\rm old}\sqrt{1 +\left(\frac{q_{95\,{\rm target}}^{\,2}}{q_{95\,{\rm old}}^{\,2}}-1\right)\frac{T_{95\,{\rm old}}^{\,2}}{T_{\rm old}^{\,2}}},\\[0.5ex]
P_{\rm new} &= P_{\rm old},
\end{align}
and
\begin{align}
I_{\phi\,{\rm new}} &=I_{\phi\,{\rm old}},\\[0.5ex]
q_{\rm new} &= q_{\rm old}\sqrt{1 +\left(\frac{q_{95\,{\rm target}}^{\,2}}{q_{95\,{\rm old}}^{\,2}}-1\right)\frac{T_{95\,{\rm old}}^{\,2}}{T_{\rm old}^{\,2}}}.
\end{align} 

Program {\sc RESCALE} then performs a type I rescaling such that 
\begin{equation}
a_1 = \frac{T_{1\,{\rm old}}}{T_{1\,{\rm new}}},
\end{equation}
where 
\begin{equation}
T_{1}\equiv  T({\mit\Psi}_N=1).
\end{equation}
It follows that 
\begin{align}
\psi_{\rm new\,new} &= a_1\,\psi_{\rm old},\\[0.5ex]
T_{\rm new\,new} &=a_1\,T_{\rm new} = T_{1\,{\rm old}}
\sqrt{
\frac{T_{\rm old}^{\,2} + (q_{95\,{\rm target}}^{\,2}/q_{95\,{\rm old}}^{\,2}-1)\,T_{95\,{\rm old}}^{\,2}}
{T_{1\,{\rm old}}^{\,2} + (q_{95\,{\rm target}}^{\,2}/q_{95\,{\rm old}}^{\,2}-1)\,T_{95\,{\rm old}}^{\,2}}
},
\\[0.5ex]
P_{\rm new\,new} &=a_1^{\,2}\,P_{\rm old},
\end{align}
and
\begin{align}
I_{\phi\,{\rm new\,new}} &= a_1\,I_{\phi\,{\rm old}},\\[0.5ex]
q_{\rm new\,new} &= q_{\rm new}= q_{\rm old}\sqrt{1 +\left(\frac{q_{95\,{\rm target}}^{\,2}}{q_{95\,{\rm old}}^{\,2}}-1\right)\frac{T_{95\,{\rm old}}^{\,2}}{T_{\rm old}^{\,2}}}.
\end{align}
Note that
\begin{align}
q_{95\,{\rm new\,new}} &= q_{95\,{\rm target}},\\[0.5ex]
T_{1\,{\rm new\,new}} &= T_{1\,{\rm old}},\\[0.5ex]
P_{\rm new\,new} &=a_1^{\,2}\,P_{\rm old}.
\end{align}

\section{Rescaling pFile}
The variables in pFiles are rescaled as follows:
\begin{align}
{\sf ne} &\rightarrow a_1\,{\sf ne},\\[0.5ex]
{\sf te} &\rightarrow a_1\,{\sf te},\\[0.5ex]
{\sf ni}& \rightarrow a_1\,{\sf ni},\\[0.5ex]
{\sf ti}&\rightarrow a_1\,{\sf ti},\\[0.5ex]
{\sf nb} &\rightarrow a_1\, {\sf nb},\\[0.5ex]
{\sf pb} &\rightarrow a_1^{\,2}\,{\sf pb},\\[0.5ex]
{\sf ptot} &\rightarrow a_1^{\,2}\,{\sf ne},\\[0.5ex]
{\sf omeg} &\rightarrow a_1\,{\sf omeg},\\[0.5ex]
{\sf omegp}& \rightarrow a_1\,{\sf omegp},\\[0.5ex]
{\sf omegvb}&\rightarrow a_1\,{\sf omegvb},\\[0.5ex]
{\sf omegpp} &\rightarrow a_1\, {\sf omegpp},\\[0.5ex]
{\sf omegeb} &\rightarrow a_1\,{\sf omegeb},\\[0.5ex]
{\sf er} &\rightarrow a_1\,{\sf er},\\[0.5ex]
{\sf ommvb}& \rightarrow a_1\,{\sf ommvb},\\[0.5ex]
{\sf ommpp}&\rightarrow a_1\,{\sf ommpp},\\[0.5ex]
{\sf omevb} &\rightarrow a_1\, {\sf omevb},\\[0.5ex]
{\sf omepp} &\rightarrow a_1\,{\sf omepp},\\[0.5ex]
{\sf kpol} &\rightarrow a_1\,{\sf kpol},\\[0.5ex]
{\sf omghb}& \rightarrow a_1\,{\sf omghb},\\[0.5ex]
{\sf nz1}&\rightarrow a_1\,{\sf nz1},\\[0.5ex]
{\sf vtor1} &\rightarrow a_1\, {\sf vtor1},\\[0.5ex]
{\sf vpol1} &\rightarrow a_1\,{\sf vpol1},\\[0.5ex]
{\sf N\,Z\,A} &\rightarrow {\sf N\,Z\,A}.
\end{align}

\end{document}

