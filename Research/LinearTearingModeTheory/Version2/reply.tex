\documentclass{article}[12pt]
\usepackage{fullpage}

\begin{document}
\begin{center}
{\em  Influence of Anomalous Perpendicular Transport on Linear Tearing Mode Dynamics in Tokamak Plasmas}\\[1ex]
by R.~Fitzpatrick\\[1ex]
{\bf Reply to Referees' Comments}
\end{center}
Let me thank the  referees for their helpful and insightful comments on my paper. In response to the
comments, I have made the following changes:
\begin{enumerate}
\item In Sect.~V, I have expressed the dimensionless parameters $Q$, $P$, and $D$ in terms of
physical quantities. This should make it easier to determine which particular resonant response regime is
appropriate to a given plasma discharge. 
\item In Sect.~VII, I have expressed the dimensionless parameters $\rho_\ast$,
$\nu_\ast$, $\beta_\ast$, $\hat{\rho}_\ast$, and $\hat{\nu}_\ast$ in terms of physical quantities. 
This should made it easer to determine which particular error-field penetration regime is appropriate to
a given plasma discharge.
\item I have added a discussion of L/H-mode scaling based on the ITER89-P scaling law to Section~VII. This discussion facilitates comparison with
experimental data. I find that I can account for the density and magnetic field-strength scalings of
the penetration threshold (in a particular response regime). However, there is a serious discrepancy between
the theoretical and experimental scalings of the penetration threshold with machine size. This
discrepancy has obvious implications for ITER. Clearly, more research is needed to resolve this discrepancy.
\item I have added a comment to the effect that the thermal force merely gives rise to an additional term
in the expression for the perpendicular electron fluid velocity that is proportional to the temperature gradient.
\item I have added a comment that clarifies that the neglect of the $\chi_{\perp\,e,i}$ contributions to
$D_\perp$ leads to the erroneous prediction that the perpendicular particle diffusivity is much smaller than the
perpendicular energy diffusivity. 
\item I have defined the new parameters $\tau_e$ and $\tau_i$ to avoid excessive repetition
of $\tau/(1+\tau)$ and $1/(1+\tau)$ factors.
\item I have corrected Eq.~(54), and the incorrect scaling $\beta_\ast\sim n\,T/B$. 
\item I cannot actually find a reference for the identities (107) and (108). However, I have added a
comment to the effect that these identities have been verified numerically.
\item I have added some parenthesis, where appropriate, so that ${\mit\Delta\Psi}_s$ is not confused with
${\mit\Delta}\,\,{\mit\Psi}_s$.
\item I have made clear, in the abstract, that the error-field scaling involves the plasma density at the
resonant surface, rather than the line-averaged density. 
\item I have tried to improve the discussion of the elimination of spurious resonances by finite perpendicular transport. 
\end{enumerate}

My additional responses to the referees' comments are as follows:
\begin{enumerate}
\item Eqs.~(112)--(116) follow very directly from Eq.~(109). 
\item The neoclassical drive only applies to relatively wide islands (i.e., islands wide enough to flatten the
temperature profile). Hence, it does not seem relevant to error-field penetration, which involves the loss
of torque balance of a very thin island. 
\item I cannot think of any obvious interpolation formula that would link all of the possible layer response regimes.
\end{enumerate}
\end{document}