\documentclass[12pt]{article}
\usepackage{amsmath}
\usepackage{fullpage}
\usepackage{graphicx}
\allowdisplaybreaks

\title{\bf Program PHASE}
\date{\today}
\author{Richard Fitzpatrick}

\begin{document}
\maketitle

\section{Plasma Angular Velocity Evolution}\label{appc}

\subsection{Plasma Angular Equations of Motion}
We can write
\begin{align}\label{e22x}
{\mit\Omega}_\theta(r,t) &={\mit\Omega}_{\theta\,0}(r) + {\mit\Delta\Omega}_\theta(r,t),\\[0.5ex]
{\mit\Omega}_\phi(r,t) &={\mit\Omega}_{\phi\,0}(r) + {\mit\Delta\Omega}_\phi(r,t),\label{e23x}
\end{align}
where  ${\mit\Omega}_{\theta}(r,t)$ and ${\mit\Omega}_{\phi}(r,t)$
  are the  poloidal and toroidal
plasma angular velocity profiles, respectively, whereas 
${\mit\Omega}_{\theta\,0}(r)$ and ${\mit\Omega}_{\phi\,0}(r)$ are
the corresponding unperturbed profiles, and 
 ${\mit\Delta\Omega}_{\theta}(r,t)$ and ${\mit\Delta\Omega}_{\phi}(r,t)$ 
are the respective modifications to the profiles induced by the  electromagnetic torques exerted at the various resonant surfaces
within the plasma. 
The modifications to the angular velocity profiles are governed by the poloidal and toroidal angular equations of
motion of the plasma, which take the respective
 forms
\begin{align}\label{eom1}
 4\pi^{\,2}\,R_0\left[(1+2\,q^{\,2})\,\rho\,r^{\,3}\,\frac{\partial{\mit\Delta\Omega}_\theta}{\partial t}-\frac{\partial}{\partial r}\!\left(
\mu\,r^{\,3}\,\frac{\partial{\mit\Delta\Omega}_\theta}{\partial r}\right) +\frac{\rho}{\tau_\theta}\,r^{\,3}\,{\mit\Delta\Omega}_\theta\right]
=\sum_{k=1,K} \delta T_{\theta\,k}\,\delta (r-r_k),\\[0.5ex]
 4\pi^{\,2}\,R_0^{\,3}\left[\rho\,r\,\frac{\partial{\mit\Delta\Omega}_\phi}{\partial t}-\frac{\partial}{\partial r}\!\left(
\mu\,r\,\frac{\partial{\mit\Delta\Omega}_\phi}{\partial r}\right) +\frac{\rho}{\tau_\phi}\,r\,{\mit\Delta\Omega}_\phi\right]
=\sum_{k=1,K} \delta T_{\phi\,k}\,\delta(r-r_k),\label{eom2}
\end{align}
and are subject to the spatial boundary conditions
\begin{align}
\frac{\partial{\mit\Delta\Omega}_\theta(0,t)}{\partial r} &=
\frac{\partial{\mit\Delta\Omega}_\phi(0,t)}{\partial r} =0,\\[0.5ex]
{\mit\Delta\Omega}_\theta(a,t)&={\mit\Delta\Omega}_\phi(a,t)=0.
\end{align}
Here, $\mu(r) = $ is the anomalous plasma perpendicular ion viscosity (due to plasma turbulence),
whereas 
$\rho(r)$ 
 is the plasma mass density profile. 
Furthermore, $1/\tau_\theta(r)$
is  the neoclassical poloidal flow-damping rate.
Finally, $1/\tau_\phi(r)$ is  the neoclassical toroidal flow-damping rate (which is, presently, set to zero in the code). 

 According to 
standard neoclassical theory, 
\begin{equation}\label{c7}
\frac{1}{\tau_\theta(r)} = \left(\frac{q\,R_0}{r}\right)^{\,2}
\frac{\mu_{00}^{\,i}}{\tau_{ii}}.
\end{equation}

\subsection{Simplified Plasma Angular Equations of Motion}
It is convenient to write
\begin{align}
{\mit\Delta\Omega}_\theta(r,t) &= \sum_{k=1,K} {\mit\Delta\Omega}_{\theta\,k}(r,t),\\[0.5ex]
{\mit\Delta\Omega}_\phi(r,t) &= \sum_{k=1,K} {\mit\Delta\Omega}_{\phi\,k}(r,t),
\end{align}
where 
\begin{align}\label{e84}
 4\pi^{\,2}\,R_0\left[(1+2\,q^{\,2})\,\rho\,r^{\,3}\,\frac{\partial{\mit\Delta\Omega}_{\theta\,k}}{\partial t}-\frac{\partial}{\partial r}\!\left(
\mu\,r^{\,3}\,\frac{\partial{\mit\Delta\Omega}_{\theta\,k}}{\partial r}\right) +\frac{\rho}{\tau_{\theta}}\,r^{\,3}\,{\mit\Delta\Omega}_{\theta\,k}\right]\nonumber\\[0.5ex]\phantom{==}=\delta T_{\theta\,k}\,\delta (r-r_k),\\[0.5ex]
 4\pi^{\,2}\,R_0^{\,3}\left[\rho\,r\,\frac{\partial{\mit\Delta\Omega}_{\phi\,k}}{\partial t}-\frac{\partial}{\partial r}\!\left(
\mu\,r\,\frac{\partial{\mit\Delta\Omega}_{\phi\,k}}{\partial r}\right) +\frac{\rho}{\tau_{\phi}}\,r\,{\mit\Delta\Omega}_{\phi\,k}\right]
= \delta T_{\phi\,k}\,\delta(r-r_k),\label{e85}
\end{align}
and 
\begin{align}
\frac{\partial{\mit\Delta\Omega}_{\theta\,k}(0,t)}{\partial r} &=
\frac{\partial{\mit\Delta\Omega}_{\phi\,k}(0,t)}{\partial r} =0,\\[0.5ex]
{\mit\Delta\Omega}_{\theta\,k}(a,t)&={\mit\Delta\Omega}_{\phi\,k}(a,t)=0.
\end{align}

In the presence of poloidal and toroidal flow damping, the
modified angular velocity profiles,  ${\mit\Delta\Omega}_{\theta\,k}$ and 
${\mit\Delta\Omega}_{\phi\,k}$,  are radially localized in the
vicinity of the $k$th resonant surface. Hence, it is a
good approximation to write Eqs.~(\ref{e84}) and (\ref{e85}) in the simplified forms 
\begin{align}
 4\pi^{\,2}\,R_0\left[(1+2\,q_k^{\,2})\,\rho_k\,r^{\,3}\,\frac{\partial{\mit\Delta\Omega}_{\theta\,k}}{\partial t}-\mu_k\,\frac{\partial}{\partial r}\!\left(
r^{\,3}\,\frac{\partial{\mit\Delta\Omega}_{\theta\,k}}{\partial r}\right) +\frac{\rho_k}{\tau_{\theta\,k}}\,r^{\,3}\,{\mit\Delta\Omega}_{\theta\,k}\right]\nonumber\\[0.5ex]\phantom{==}=\delta T_{\theta\,k}\,\delta (r-r_k),\\[0.5ex]
 4\pi^{\,2}\,R_0^{\,3}\left[\rho_k\,r\,\frac{\partial{\mit\Delta\Omega}_{\phi\,k}}{\partial t}-\mu_k\frac{\partial}{\partial r}\!\left(
r\,\frac{\partial{\mit\Delta\Omega}_{\phi\,k}}{\partial r}\right) +\frac{\rho_k}{\tau_{\phi\,k}}\,r\,{\mit\Delta\Omega}_{\phi\,k}\right]
= \delta T_{\phi\,k}\,\delta(r-r_k),
\end{align}
where $q_k = q(r_k)$, $\rho_k=\rho(r_k)$, $\mu_k=\mu(r_k)$, $\tau_{\theta\,k}=\tau_\theta(r_k)$, and $\tau_{\phi\,k}=\tau_\phi(r_k)$. 

\subsection{Normalized Plasma Equations of Angular Motion}
Let 
\begin{equation}
\tau_A = \left(\frac{\mu_0\,\rho_0\,a^{\,2}}{B_0^{\,2}}\right)^{1/2},
\end{equation}
where
 $\rho_0=\rho(0)$,
 and
\begin{equation}
\rho(r)
\simeq  m_i\,[n_i(r)+n_b(r)] + m_I\,n_I(r)
\end{equation}
 is the plasma mass density.  
  Furthermore, let $\hat{r}=r/a$, $\hat{r}_k=r_k/a$, $\hat{a}=a/R_0$, $\hat{t}=t/\tau_A$, $\hat{\rho}_k=\rho_k/\rho_0$, $\tau_{M\,k} = \rho_k\,a^{\,2}/\mu_k= a^2/\chi_\phi(r_k)$, $\hat{\tau}_{M\,k}= \tau_{M\,k}/\tau_A$, 
$\hat{\tau}_{\theta\,k}=\tau_{\theta\,k}/\tau_A$, $
\hat{\tau}_{\phi\,k}=\tau_{\phi\,k}/\tau_A$, $\hat{\mit\Omega}_\theta=\tau_A\,{\mit\Omega}_\theta$, $\hat{\mit\Omega}_{\theta\,0}=\tau_A\,{\mit\Omega}_{\theta\,0}$,
${\mit\Delta}\hat{\mit\Omega}_{\theta\,k} = \tau_A\,{\mit\Delta\Omega}_{\theta\,k}$, $\hat{\mit\Omega}_\phi=\tau_A\,{\mit\Omega}_\phi$, $\hat{\mit\Omega}_{\phi\,0}=\tau_A\,{\mit\Omega}_{\phi\,0}$,
and 
${\mit\Delta}\hat{\mit\Omega}_{\phi\,k} = \tau_A\,{\mit\Delta\Omega}_{\phi\,k}$. It follows that
\begin{align}
\hat{\mit\Omega}_\theta(\hat{r},\hat{t}) &=\hat{\mit\Omega}_{\theta\,0}(\hat{r}) + \sum_{k=1,K}{\mit\Delta}\hat{\mit\Omega}_{\theta\,k}(\hat{r},\hat{t}),\\[0.5ex]
\hat{\mit\Omega}_\phi(\hat{r},\hat{t}) &=\hat{\mit\Omega}_{\phi\,0}(\hat{r}) + \sum_{k=1,K}{\mit\Delta}\hat{\mit\Omega}_{\phi\,k}(\hat{r},\hat{t}),
\end{align}
where
\begin{align}
(1+2\,q_k^{\,2})\,\hat{r}^{\,3}\,\frac{\partial{\mit\Delta}\hat{\mit\Omega}_{\theta\,k}}{\partial \hat{t}}-\frac{1}{\hat{\tau}_{M\,k}}\,\frac{\partial}{\partial \hat{r}}\!\left(
\hat{r}^{\,3}\,\frac{\partial{\mit\Delta}\hat{\mit\Omega}_{\theta\,k}}{\partial \hat{r}}\right) +\frac{1}{\hat{\tau}_{\theta\,k}}\,\hat{r}^{\,3}\,{\mit\Delta}\hat{\mit\Omega}_{\theta\,k}\nonumber\\[0.5ex]\phantom{==}=-\frac{m_k}{2\,\hat{\rho}_k\,\hat{a}^{\,2}}\,\delta\hat{T}_k\,\delta (\hat{r}-\hat{r}_k),\label{exxx}\\[0.5ex]
\hat{r}\,\frac{\partial{\mit\Delta}\hat{\mit\Omega}_{\phi\,k}}{\partial \hat{t}}-\frac{1}{\hat{\tau}_{M\,k}}\,\frac{\partial}{\partial \hat{r}}\!\left(
\hat{r}\,\frac{\partial{\mit\Delta}\hat{\mit\Omega}_{\phi\,k}}{\partial \hat{r}}\right) +\frac{1}{\hat{\tau}_{\phi\,k}}\,\hat{r}\,{\mit\Delta}\hat{\mit\Omega}_{\phi\,k}
= \frac{n}{2\,\hat{\rho}_k}\,\delta \hat{T}_{k}\,\delta(\hat{r}-\hat{r}_k),\label{eyyy}
\end{align}
and
\begin{align}
\frac{\partial{\mit\Delta}{\hat{\mit\Omega}}_{\theta\,k}(0,\hat{t})}{\partial \hat{r}} &=
\frac{\partial{\mit\Delta}\hat{\mit\Omega}_{\phi\,k}(0,\hat{t})}{\partial \hat{r}} =0,\label{e152}\\[0.5ex]
{\mit\Delta}\hat{\mit\Omega}_{\theta\,k}(1,\hat{t})&={\mit\Delta}\hat{\mit\Omega}_{\phi\,k}(1,\hat{t})=0.\label{e153}
\end{align}

\subsection{Solution of Plasma Angular Equations of Motion}
Let
\begin{align}\label{e154}
{\mit\Delta}\hat{\mit\Omega}_{\theta\,k}(\hat{r},\hat{t})&=-\frac{1}{m_k}\sum_{p=1,P} \alpha_{k,p}(\hat{t})\,\frac{y_p(\hat{r})}{y_p(\hat{r}_k)},\\[0.5ex]
{\mit\Delta}\hat{\mit\Omega}_{\phi\,k}(\hat{r},\hat{t})&=\frac{1}{n}\sum_{p=1,P} \beta_{k,p}(\hat{t})\,\frac{z_p(\hat{r})}{z_p(\hat{r}_k)},\label{e155}
\end{align}
where
\begin{align}
y_p(\hat{r}) &=\frac{J_1(j_{1,p}\,\hat{r})}{\hat{r}},\\[0.5ex]
z_p(\hat{r}) &= J_0(j_{0,p}\,\hat{r}),\label{e158}
\end{align}
and $P\gg 1$. 
Here, $J_m(z)$ is a standard Bessel function, and $j_{m,p}$ denotes the $p$th zero of the $J_m(z)$ Bessel function. 
Note that  Eqs.~(\ref{e154}) and (\ref{e155}) automatically  satisfy the boundary conditions Eqs.~(\ref{e152}) and (\ref{e153}). 

It is easily demonstrated that
\begin{align}
\frac{d}{d\hat{r}}\!\left(\hat{r}^{\,3}\,\frac{dy_p}{d\hat{r}}\right)& = - j_{1,p}^{\,2}\,\hat{r}^{\,3}\,y_p,\\[0.5ex]
\frac{d}{d\hat{r}}\!\left(\hat{r}\,\frac{dz_p}{d\hat{r}}\right) &= - j_{0,p}^{\,2}\,\hat{r}\,z_p,
\end{align}
and
\begin{align}
\int_0^{1}\hat{r}^{\,3}\,y_p(\hat{r})\,y_q(\hat{r})\,d\hat{r}& = \frac{1}{2}\,\left[J_2(j_{1,p})\right]^{\,2}\,\delta_{pq},\\[0.5ex]
\int_0^1\hat{r}\,z_p(\hat{r})\,z_q(\hat{r})\,d\hat{r}& = \frac{1}{2}\,\left[J_1(j_{0,p})\right]^{\,2}\,\delta_{pq}\label{e162}
\end{align}
Hence, Eqs.~(\ref{exxx}) and (\ref{eyyy}) yield
\begin{align}
(1+2\,q_k^{\,2})\,\frac{d\alpha_{k,p}}{d\hat{t}}
+\left(\frac{j_{1,p}^{\,2}}{\hat{\tau}_{M\,k}} + \frac{1}{\hat{\tau}_{\theta\,k}}\right)\alpha_{k,p}&=
\frac{m_k^{\,2}\,[y_p(\hat{r}_k)]^{\,2}}{\hat{\rho}_k\,\hat{a}^{\,2}\,
[J_2(j_{1,p})]^{\,2}}\,\delta \hat{T}_k,\\[0.5ex]
\frac{d\beta_{k,p}}{d\hat{t}}
+\left(\frac{j_{0,p}^{\,2}}{\hat{\tau}_{M\,k}} + \frac{1}{\hat{\tau}_{\phi\,k}}\right)\beta_{k,n}&=
\frac{n^{\,2}\,[z_p(\hat{r}_k)]^{\,2}}{\hat{\rho}_k\,
[J_1(j_{0,p})]^{\,2}}\,\delta \hat{T}_k.
\end{align}
It follows that
\begin{align}\label{errr}
\hat{\mit\Omega}_\theta(\hat{r}_k,\hat{t}) &=
\hat{\mit\Omega}_{\theta\,0}(\hat{r}_k) - \sum_{k'=1,K}^{p=1,P} \frac{\alpha_{k',p}(\hat{t})}{m_{k'}}\,\frac{y_p(\hat{r}_k)}{y_p(\hat{r}_{k'})},\\[0.5ex]
\hat{\mit\Omega}_\phi(\hat{r}_k,\hat{t}) &=
\hat{\mit\Omega}_{\phi\,0}(\hat{r}_k) + \sum_{k'=1,K}^{p=1,P} \frac{\beta_{k',p}(\hat{t})}{n}\,\frac{z_p(\hat{r}_k)}{z_p(\hat{r}_{k'})}.\label{errt}
\end{align}
Here, $k$ indexes the various resonant surfaces in the plasma,
whereas $p$  indexes the various
velocity harmonics. 

\section{Resonant Plasma Response Model}
Fundamental equation:
\begin{equation}\label{e1}
\left(\hat{W}_k +\hat{\delta}_k\right){\cal S}_k\left(\frac{d}{d\hat{t}} + {\rm i}\,\hat{\varpi}_k\right){\mit\Psi}_k = \sum_{k'=1,K}E_{kk'}\,{\mit\Psi}_{k'}
+ \hat{E}_{kk}\,\chi_k,
\end{equation}
where
\begin{align}
{\cal S}_k&=\frac{\tau_R(\hat{r}_k)}{\tau_A},\\[0.5ex]
\hat{W}_k &=\frac{2\,{\cal I}}{\hat{a}\,\hat{r}_k}\left(\frac{q}{g\,s}\right)_{\hat{r}_k}^{1/2} |\hat{\mit\Psi}_k|^{\,1/2},\\[0.5ex]
\hat{\delta}_k &= \frac{\delta_{\rm SC}(\hat{r}_k)}{R_0\,\hat{a}\,\hat{r}_k},
\end{align}
with ${\cal I}=08227$, and
\begin{align}
\tau_R(r) &= \mu_0\,r^{\,2}\,Q_{00}^{ee}\,\sigma_{ee},\\[0.5ex]
\sigma_{ee}(r) &=\frac{n_e\,e^{\,2}\,\tau_{ee}}{m_e}.
\end{align}
Also,
\begin{align}
\delta_{\rm SC}(r)& = \pi \,\frac{|n\,\omega_{\ast\,e}|^{\,1/2}\,\tau_H}{(\rho_s/r)\,\tau_R^{\,1/2}}\,r,\\[0.5ex]
\tau_H &= \frac{R_0}{B_0\,g}\,\frac{\sqrt{\mu_0\,\rho}}{n\,s},\\[0.5ex]
\rho_s &= \frac{\sqrt{m_i\,T_e}}{e\,B_0\,g}.
\end{align}

Follows from Eq.~(\ref{e1}) that
\begin{align}
\left(\hat{W}_k +\hat{\delta}_k\right){\cal S}_k\frac{d\hat{\mit\Psi}_k}{d\hat{t}}&= \sum_{k'=1,K}\hat{E}_{kk'}\,\hat{\mit\Psi}_{k'}\,\cos(\varphi_k-\varphi_{k'}-\xi_{kk'})
+ \hat{E}_{kk}\,\hat{\chi}_k\,\cos(\varphi_k-\zeta_k),\\[0.5ex]
\left(\hat{W}_k +\hat{\delta}_k\right){\cal S}_k\,\hat{\mit\Psi}_k\left(\frac{d\varphi_k}{d\hat{t}}-\hat{\varpi}_k\right)&=- \sum_{k'=1,K}\hat{E}_{kk'}\,\hat{\mit\Psi}_{k'}\,\sin(\varphi_k-\varphi_{k'}-\xi_{kk'})-
 \hat{E}_{kk}\,\hat{\chi}_k\,\sin(\varphi_k-\zeta_k).
\end{align}

Alternatively, let
\begin{align}
X_k &= \hat{\mit\Psi}_k\,\cos\varphi_k,\\[0.5ex]
Y_k &= \hat{\mit\Psi}_k\,\sin\varphi_k.
\end{align}
So, Eq.~(\ref{e1}) gives
\begin{align}
\left(\hat{W}_k +\hat{\delta}_k\right){\cal S}_k\left(\frac{d}{d\hat{t}} + {\rm i}\,\hat{\varpi}_k\right)(X_k - {\rm i}\,Y_k)&=
\sum_{k'=1,K}\hat{E}_{kk'}\,(\cos\xi_{kk'} - {\rm i}\,\sin\xi_{kk'})\,(X_{k'} - {\rm i}\,Y_{k'})\nonumber\\[0.5ex]
&\phantom{=}+ \hat{E}_{kk}\,\hat{\chi}_k\,(\cos\zeta_k - {\rm i }\,\sin\zeta_k),
\end{align}
or
\begin{align}
&\left(\hat{W}_k +\hat{\delta}_k\right){\cal S}_k\left(\frac{dX_k}{d\hat{t}} - {\rm i}\,\frac{dY_k}{d\hat{t}} + {\rm i}\,\hat{\varpi}_k\,X_k
+\hat{\varpi}_k\,Y_k\right)&\nonumber\\[0.5ex]
&= \sum_{k'=1,K}\hat{E}_{kk'}\,(\cos\xi_{kk'}\,X_{k'} -{\rm i}\,\cos\xi_{kk'}\,Y_{k'}
-{\rm i}\,\sin\xi_{kk'}\,X_{k'} -\sin\xi_{kk'}\,Y_{k'})\nonumber\\[0.5ex]
&\phantom{=} + \hat{E}_{kk'}\,\hat{\chi}_k\,(\cos\zeta_k - {\rm i}\,\sin\zeta_k).
\end{align}
Follows that
\begin{align}
\left(\hat{W}_k +\hat{\delta}_k\right){\cal S}_k\left(\frac{dX_k}{d\hat{t}} 
+\hat{\varpi}_k\,Y_k\right)= \sum_{k'=1,K}\hat{E}_{kk'}\,(\cos\xi_{kk'}\,X_{k'}  -\sin\xi_{kk'}\,Y_{k'}) + \hat{E}_{kk}\,\hat{\chi}_k\,\cos\zeta_k,\\[0.5ex]
\left(\hat{W}_k +\hat{\delta}_k\right){\cal S}_k\left(\frac{dY_k}{d\hat{t}} -\hat{\varpi}_k\,X_k\right)
= \sum_{k'=1,K}\hat{E}_{kk'}\,(\cos\xi_{kk'}\,Y_{k'}
+\sin\xi_{kk'}\,X_{k'}) + \hat{E}_{kk}\,\hat{\chi}_k\,\sin\zeta_k,
\end{align}
where
\begin{equation}\label{e23}
\hat{W}_k =\frac{2\,{\cal I}}{\hat{a}\,\hat{r}_k}\left(\frac{q}{g\,s}\right)^{1/2}_{\hat{r}_k} (X_k^{\,2} + Y_k^{\,2})^{\,1/4}.
\end{equation}

Note, finally, that
\begin{equation}
\delta \hat{T}_k = \sum_{k'\neq k} \hat{E}_{kk'}\,\hat{\mit\Psi}_k\,\hat{\mit\Psi}_{k'}\,\sin(\varphi_k-\varphi_{k'}-\xi_{kk'}) + \hat{E}_{kk'}\,\hat{\mit\Psi}_k\,\hat{\chi}_k\,\sin(\varphi_k-\zeta_k).
\end{equation}

\section{Density and Temperature Flattening Widths}
Temperature flattening width (in $r$):
\begin{equation}
W_{T_e\,k} = \left(\frac{\chi_{E}}{v_{T\,e}\,r}\,\frac{1}{\epsilon\,s\,n}\right)^{1/3}_{r_k}\,r_k,
\end{equation}
where $\epsilon= r/R_0$. Density flattening width (in $r$):
\begin{equation}
W_{n_e\,k} = \left(\frac{D_\perp}{v_{T\,i}\,r}\,\frac{1}{\epsilon\,s\,n}\right)^{1/3}_{r_k}\,r_k.
\end{equation}

\end{document}