\documentclass[12pt,prb,aps]{revtex4-1}
\usepackage {amsmath}
\pdfoutput = 1 
\usepackage {graphicx}
\newcommand {\bomega}{\mbox{\boldmath$\omega$}}
\newcommand {\bpi}{\mbox{\boldmath$\pi$}}

\begin{document}

\title {Modelling $q_{95}$ Windows for the Suppression of Edge Localized Modes by Resonant Magnetic Perturbations in the DIII-D Tokamak: Relaxation of the No-Slip
Constraint}

\author{R.~Fitzpatrick\,\footnote{rfitzp@farside.ph.utexas.edu}}
\affiliation{Institute for Fusion Studies,  Department of Physics,  University of Texas at Austin,  Austin TX, 78712, USA}

%\begin{abstract}
%\end{abstract}

\section{Introduction}

\section{Cylindrical RMP Response Model}\label{s2}
\subsection{Plasma Equilibrium}
Consider a large aspect-ratio, low-$\beta$, tokamak plasma whose magnetic flux surfaces map out (almost)
concentric circles in the poloidal plane.
Such a plasma is well approximated as a periodic cylinder.  Suppose that the minor radius of the plasma
is $a$. Standard cylindrical coordinates ($r$, $\theta$, $z$) are adopted. The system is assumed to
be periodic in the $z$-direction, with periodicity length $2\pi\,R_0$, where $R_0\gg a$ is the simulated
plasma major radius. It is convenient to define the simulated toroidal angle $\phi=z/R_0$. 

The equilibrium magnetic field is written ${\bf B} = [0,B_\theta(r), B_\phi]$. The associated equilibrium
plasma current density takes the form ${\bf j} =[0,0,j_\phi(r)]$, where $\mu_0\,j_\phi= (1/r)\,d(r\,B_\theta)/dr$. The
so-called {\em safety-factor}, $q(r)=r\,B_\phi/(R_0\,B_\theta)$, parameterizes the helical pitches of equilibrium magnetic
field-lines. 

\subsection{Plasma Response to RMP}
Consider the response of the plasma to an externally generated, static, helical, resonant magnetic perturbation (RMP). 
Suppose that the RMP has $m$ periods in the poloidal direction, and $n$ periods in the toroidal direction.
It is convenient to express the perturbed magnetic field and the perturbed plasma current density in
terms of a perturbed poloidal magnetic flux, $\psi(r,\theta,\phi,t)$. In fact, $\delta{\bf B} = \nabla\psi\times \nabla z$, 
and $\mu_0\,\delta {\bf j} = - \nabla^2\psi\,\nabla z$,
where $\psi(r,\theta,\phi,t) = \hat{\psi}(r,t)\,\exp[{\rm i}\,(m\,\theta-n\,\phi)]$, and $\hat{\psi}$ is real. This particular representation
is valid provided that $m/n\gg a/R_0$.\cite{rfa}

The response of the plasma to the RMP is governed by the equations of marginally stable, ideal-MHD everywhere in the
plasma, apart from a relatively narrow (in $r$) region in the vicinity of the so-called {\em rational surface}, minor radius $r_s$, 
at which $q(r_s)=m/n$.\cite{rfa}

It is convenient to parameterize the RMP in terms of the so-called {\em vacuum magnetic flux}, ${\mit\Psi}_v(t) = |{\mit\Psi}_v|\,{\rm e}^{-{\rm i}\,\varphi_v}$,
which is defined to be the value of $\hat{\psi}(r,t)$ at radius $r_s$ in the presence of the RMP, but in the absence of plasma. Here,
$\varphi_v$ is the helical phase of the RMP. Likewise, the response of the plasma in the vicinity of the rational
surface is parameterized in terms of the so-called {\em reconnected magnetic flux}, ${\mit\Psi}_s(t)=|{\mit\Psi}_s|\,{\rm e}^{\,{\rm i}\,\varphi_s}$, 
which is the actual value of $\hat{\psi}(r,t)$ at radius $r_s$. Here, $\varphi_s$ is the helical phase of
the reconnected magnetic flux.

The intrinsic stability of the $m$, $n$ tearing mode is governed by the dimensionless parameter
$E_{ss}= [d\hat{\psi}_s/d\ln r]_{r_{s-}}^{r_{s+}}$, where $\hat{\psi}_s(r)$
is a solution of the marginally stable, ideal-MHD equations, for the case of an $m$, $n$ helical
perturbation, that satisfies physical boundary conditions at $r=0$ and $r=a$, in the absence of the
RMP, and is such that $\hat{\psi}_s(r_s)=1$.\cite{fkr,am1} According to resistive-MHD theory,\cite{fkr,ruth}
if $E_{ss}>0$ then the $m$, $n$ tearing mode spontaneously reconnects magnetic flux at the rational
surface to form a helical magnetic island chain. In this paper, it is assumed that $E_{ss}<0$, so that the
$m$, $n$ tearing mode is intrinsically stable. In this case, any magnetic reconnection that takes place
at the rational surface is due solely to the action of the RMP.

The ideal-MHD response of the plasma to the RMP is governed by the dimensionless parameter
$E_{sv}= [d\hat{\psi}_v/d\ln r]_{r_{s-}}^{r_{s+}}$, where $\hat{\psi}_v(r)$
is a solution of the marginally stable, ideal-MHD equations, for the case of an $m$, $n$ helical
perturbation, that satisfies physical boundary conditions at $r=0$ and $r=a$, in the presence of the
RMP, and is such that $\hat{\psi}_v(r_s)=0$.\cite{am1} 

\subsection{Linear Response Regime}
In the linear response regime, the reconnected magnetic flux induced by the RMP at the
rational surface is governed by\,\cite{rfa,rfb,rfc}
\begin{equation}\label{e1}
\frac{\delta_s}{r_s}\,\tau_R\left(\frac{d}{dt}+{\rm i}\,\omega_s\right){\mit\Psi}_s= E_{ss}\,{\mit\Psi}_s+E_{sv}\,{\mit\Psi}_v,
\end{equation}
where $\delta_s$ is the linear layer width, $\tau_R=\mu_0\,r_s^{\,2}\,\sigma(r_s)$ the global resistive diffusion timescale,
 and
\begin{equation}
\omega_s (t)=m\,{\mit\Omega}_\theta(r_s,t)-n\,{\mit\Omega}_\phi(r_s,t).
\end{equation}
Here, $\sigma(r)$, ${\mit\Omega}_\theta(r,t)$, and ${\mit\Omega}_\phi(r,t)$ are the 
plasma conductivity, poloidal angular velocity, and toroidal angular velocity profiles, respectively. 
Note that, in this paper, we are assuming that the constant-$\psi$ approximation\,\cite{fkr} holds at the rational surface. (It is demonstrated in
Ref.~\onlinecite{rf1} that the appropriate linear response regime for an RMP resonant in the pedestal of a typical DIII-D H-mode discharge
is the so-called  {\em first semi-collisional regime}, which is indeed a constant-$\psi$ response regime. Moreover, if the appropriate linear response regime
is constant-$\psi$ then so is the appropriate nonlinear response regime, because the constant-$\psi$ constraint is
easier to satisfy in the nonlinear regime than in the linear regime.\cite{rfb,rffig})

Equation~(\ref{e1}) can be rewritten as an {\em amplitude evolution equation},
\begin{equation}\label{e3}
\frac{\delta_s}{r_s}\,\tau_R\,\frac{d|{\mit\Psi}_s|}{dt}= E_{ss}\,|{\mit\Psi}_s| + E_{sv}\,|{\mit\Psi}_v|\,\cos(\varphi_s-\varphi_v),
\end{equation}
combined with a {\em phase evolution equation}, 
\begin{equation}\label{e4}
\frac{d\varphi_s}{dt} = \omega_s - \frac{E_{sv}}{\tau_R}\,\frac{r_s}{\delta_s}\,\frac{|{\mit\Psi}_v|}{|{\mit\Psi}_s|}\,\sin(\varphi_s-\varphi_v).
\end{equation}
The final term on the right-hand side of Eq.~(\ref{e4}) is termed the {\em slip frequency}, and is the difference between the helical phase velocity of
the reconnected magnetic flux and that expected on the  assumption that the flux is convected by the plasma flow at the
rational surface (i.e., $d\varphi_s/dt=\omega_s$). In general, the slip frequency is non-zero because the plasma in the vicinity of the rational surface is capable of diffusing resistively 
through the linear layer structure. 

\subsection{Nonlinear Response Regime}
In the nonlinear response regime, Eq.~(\ref{e3}) generalizes to the {\em Rutherford island width evolution equation}:\,\cite{ruth,ruth1}
\begin{equation}\label{e5}
\frac{({\cal I}/2)\,W_s}{r_s}\,\tau_R\,\frac{d|{\mit\Psi}_s|}{dt} = E_{ss}\,|{\mit\Psi}_s| + E_{sv}\,|{\mit\Psi}_v|\,\cos(\varphi_s-\varphi_v),
\end{equation}
where
\begin{equation}
W_s = 4\left(\frac{|{\mit\Psi}_s|}{s_s\,r_s\,B_\theta(r_s)}\right)^{1/2} r_s
\end{equation}
is the full radial width of the magnetic island chain, ${\cal I} = 0.8227$, $s(r) = d\ln q/d\ln r$, and $s_s=s(r_s)$. The
nonlinear regime holds when $W_s\gg \delta_s$, whereas the linear regime holds when $\delta_s\ll W_s$. 

Equation~(\ref{e5}) is usually coupled with the so-called {\em no-slip constraint}:\,\cite{rfa}
\begin{equation}\label{e6}
\frac{d\varphi_s}{dt} = \omega_s.
\end{equation}
The reasoning behind the imposition of this constraint is that the plasma is trapped inside the
magnetic separatrix of the magnetic island chain that forms in the vicinity of the rational surface,
which forces the chain to co-rotate with the local plasma flow. However, in reality, the plasma
is able to diffuse resistively across the separatrix to some extent. Hence, by analogy with Eqs.~(\ref{e3}), (\ref{e4}), and (\ref{e5}), and 
in accordance with Refs.~\onlinecite{slip1} and \onlinecite{slip2}, in this paper we shall modify Eq.~(\ref{e6}) such that
\begin{equation}\label{e7}
\frac{d\varphi_s}{dt} = \omega_s - \frac{E_{sv}}{\tau_R}\,\frac{r_s}{({\cal I}/2)\,W_s}\,\frac{|{\mit\Psi}_v|}{|{\mit\Psi}_s|}\,\sin(\varphi_s-\varphi_v).
\end{equation}
The final term on the right-hand side of the previous equation is the {\em nonlinear slip frequency}. In general,
we would expect this frequency to be relatively small (because it is generally the case that $|\omega_s|\,\tau_R\,(W_s/r_s)\gg 1$
for fully-developed magnetic islands in high-temperature tokamak plasmas\,\cite{rf1,slip2}). Nevertheless, the
nonlinear slip frequency may be non-negligible for developing island chains, which is why we are including it in our analysis.
(Note that the nonlinear slip frequency does not appear in rigorous island calculations, such as
that described in Ref.~\onlinecite{rfisland}, and references therein, because such calculations depend crucially on
complicated hierarchical ordering assumptions, according to which the nonlinear slip frequency is negligible.)

\subsection{Composite Model}
We can combine Eqs.~(\ref{e3}), (\ref{e4}), (\ref{e5}), and (\ref{e7}) to give the following
composite resonant response model that interpolates between the linear and the nonlinear
regimes:
\begin{align}\label{e9}
\left(\frac{({\cal I}/2)\,W_s+\delta_s}{r_s}\right)\tau_R\,\frac{d|{\mit\Psi}_s|}{dt} &= E_{ss}\,|{\mit\Psi}_s| + E_{sv}\,|{\mit\Psi}_v|\,\cos(\varphi_s-\varphi_v),\\[0.5ex]
\frac{d\varphi_s}{dt}& = \omega_s - \frac{E_{sv}}{\tau_R}\left(\frac{r_s}{({\cal I}/2)\,W_s+\delta_s}\right)\frac{|{\mit\Psi}_v|}{|{\mit\Psi}_s|}\,\sin(\varphi_s-\varphi_v).\label{e10}
\end{align}
If we define $X_s = |{\mit\Psi}_s|\,\cos\varphi_s$, $Y_s = |{\mit\Psi}_s|\,\sin\varphi_s$, $X_v = |{\mit\Psi}_v|\,\cos\varphi_v$, and $Y_v = |{\mit\Psi}_v|\,\sin\varphi_v$ 
then the previous two equations are more conveniently written in the nonsingular (when $|{\mit\Psi}_s|=0$) forms:\,\cite{slip1,slip2,slip3}
\begin{align}\label{e11}
\left(\frac{({\cal I}/2)\,W_s+\delta_s}{r_s}\right)\tau_R\left(\frac{dX_s}{dt}+\omega_s\,Y_s\right)&= E_{ss}\,X_s + E_{sv}\,X_v,\\[0.5ex]
\left(\frac{({\cal I}/2)\,W_s+\delta_s}{r_s}\right)\tau_R\left(\frac{dY_s}{dt}-\omega_s\,X_s\right)& = E_{ss}\,Y_s + E_{sv}\,Y_v,\label{e12}
\end{align}
where
\begin{equation}
W_s = 4\left(\frac{\sqrt{X_s^{\,2}+Y_s^{\,2}}}{s_s\,r_s\,B_\theta(r_s)}\right)^{1/2} r_s.
\end{equation}
With the benefit of hindsight, Eqs.~(\ref{e11}) and (\ref{e12}) could have been derived more directly from the
following heuristic generalization of Eq.~(\ref{e1}):\,\cite{slip2}
\begin{equation}
\left(\frac{({\cal I}/2)\,W_s+\delta_s}{r_s}\right)\tau_R\left(\frac{d}{dt}+{\rm i}\,\omega_s\right){\mit\Psi}_s= E_{ss}\,{\mit\Psi}_s+E_{sv}\,{\mit\Psi}_v.
\end{equation}

\section{Modified Toroidal RMP Response Model}
Our toroidal model of the response of a tokamak plasma to a static, externally generated, RMP is described in
detail in Ref.~\onlinecite{rftor}. In the light of the composite model developed in Sect.~\ref{s2}, we shall
modify the resonant plasma response component of our  toroidal response model; this component is described in Appendix~D of Ref.~\onlinecite{rftor}.

Equations~(D1) and (D2) in Ref.~\onlinecite{rftor} are modified such that 
\begin{align}\label{e15}
\left(\hat{W}_k + \hat{\delta}_k\right){\cal S}_k\,\frac{d\hat{\mit\Psi}_k}{d\hat{t}} &= \sum_{k'=1,K}\hat{E}_{kk'}\hat{\mit\Psi}_{k'}\,\cos(\varphi_k-\varphi_{k'}-\xi_{kk'})
+\hat{E}_{kk}\,\hat{\chi}_k\,\cos(\varphi_k-\zeta_k),\\[0.5ex]
{\cal S}_k&=\frac{\tau_R(\hat{r}_k)}{\tau_A},
\end{align}
respectively. Here, 
\begin{align}
\hat{W}_k &=\frac{2\,{\cal I}}{\hat{a}\,\hat{r}_k}\left(\frac{q}{g\,s}\right)_{\hat{r}_k} |\hat{\mit\Psi}_k|^{\,1/2},\\[0.5ex]
\hat{\delta}_k &= \frac{\delta_{\rm SC}(\hat{r}_k)}{R_0\,\hat{a}\,\hat{r}_k},
\end{align}
where $\delta_{\rm SC}(\hat{r})$ is the semi-collisional linear layer width specified in Ref.~\onlinecite{rf1}.

In the previous expressions, $R_0$ is the major radius of the magnetic axis, $r$ is a magnetic flux surface label with dimensions of length (which, roughly speaking, is the average minor radius of the
flux surface), $q(r)$ is the safety-factor profile, $s(r)=d\ln q/d\ln r$, 
$g(r)\,B_0$ is the toroidal magnetic field-strength at major radius $R_0$, $B_0$ is the vacuum
toroidal magnetic field-strength on the magnetic axis, $a$ is the value of $r$ on the last closed magnetic-flux surface, $r_k$ is the value of $r$ on the $k$th rational
surface (resonant with poloidal mode number $m_k$, and toroidal mode number $n$), $\hat{a}=a/R_0$, $\hat{r}=r/a$, and $\hat{r}_k=r_k/a$. 
Furthermore, $\tau_R(\hat{r})= \mu_0\,r^{\,2}\,\sigma_{ee}\,Q_{00}^{\,ee}$, where $\sigma_{ee}$ is the classical plasma parallel electrical conductivity, and
$Q_{00}^{\,ee}$ describes the corrections to this conductivity due to neoclassical effects and the presence of plasma impurities. Also, $\tau_A$ is a convenient
scale time, and $\hat{t}=t/\tau_A$. Finally, ${\mit\Psi}_k = B_0\,R_0\,\hat{\mit\Psi}_k\,{\rm e}^{-{\rm i}\,\varphi_k}$ is the reconnected magnetic flux
at the $k$th rational surface, ${\mit\chi}_k = B_0\,R_0\,\hat{\mit\chi}_k\,{\rm e}^{-{\rm i}\,\zeta_k}$ is a measure of the vacuum magnetic flux at the $k$th
rational surface (in fact, $\hat{E}_{kk}\,\chi_k$ is equivalent to $E_{sv}\,{\mit\Psi}_v$ in Sect.~\ref{s2}), and $E_{kk'} =\hat{E}_{kk'}\,{\rm e}^{-{\rm i}\,\xi_{kk'}}$ is the toroidal tearing stability matrix.\cite{am1} Here, $\hat{\mit\Psi}_k>0$, $\varphi_k$, $\hat{\chi}_k>0$,
$\zeta_k$, $\hat{E}_{kk'}>0$, and $\xi_{kk'}$ are all real quantities.

Equation~(\ref{e15}) represents a minor improvement to Eq.~(D1) of Ref.~\onlinecite{rftor} in which the interpolation between the
linear and nonlinear regimes is performed in a more systematic manner. 

Equation~(D6) of Ref.~\onlinecite{rftor} is modified such that
\begin{align}\label{e19}
\left(\hat{W}_k +\hat{\delta}_k\right){\cal S}_k\left(\frac{d\varphi_k}{d\hat{t}}-\hat{\varpi}_k\right)&=
- \sum_{k'=1,K}\hat{E}_{kk'}\,\hat{\mit\Psi}_{k'}\,\sin(\varphi_k-\varphi_{k'}-\xi_{kk'})-\hat{E}_{kk}\,\hat{\chi}_k\,\sin(\varphi_k-\zeta_k),
\end{align}
where
\begin{equation}
\hat{\varpi}_k= m_k\,\hat{\mit\Omega}_\theta(\hat{r}_k,\hat{t})-n\,\hat{\mit\Omega}_\phi(\hat{r}_k,\hat{t}),
\end{equation}
and $\hat{\mit\Omega}_\theta= {\mit\Omega}_\theta\,\tau_A$, $\hat{\mit\Omega}_\phi= {\mit\Omega}_\phi\,\tau_A$. Equation~(\ref{e19}) represents
a major improvement to Eq.~(D6) that takes into  account the finite slip frequencies of the magnetic island chains induced at
the various rational surfaces in the plasma. 

If we define $X_k = \hat{\mit\Psi}_k\,\cos\varphi_k$ and $Y_k=\hat{\mit\Psi}_k\,\sin\varphi_k$ then Eqs.~(\ref{e15}) and (\ref{e19}) are
more conveniently written in the nonsingular (when $\hat{\mit\Psi}_k=0$) forms
\begin{align}\label{e21}
\left(\hat{W}_k +\hat{\delta}_k\right){\cal S}_k\left(\frac{dX_k}{d\hat{t}} 
+\hat{\varpi}_k\,Y_k\right)= \sum_{k'=1,K}\hat{E}_{kk'}\,(\cos\xi_{kk'}\,X_{k'}  -\sin\xi_{kk'}\,Y_{k'}) + \hat{E}_{kk}\,\hat{\chi}_k\,\cos\zeta_k,\\[0.5ex]
\left(\hat{W}_k +\hat{\delta}_k\right){\cal S}_k\left(\frac{dY_k}{d\hat{t}} -\hat{\varpi}_k\,X_k\right)
= \sum_{k'=1,K}\hat{E}_{kk'}\,(\cos\xi_{kk'}\,Y_{k'}
+\sin\xi_{kk'}\,X_{k'}) + \hat{E}_{kk}\,\hat{\chi}_k\,\sin\zeta_k.\label{e22}
\end{align}
Equations (\ref{e15}), (\ref{e19}), (\ref{e21}), and (\ref{e22}) are the toroidal generalizations of Eqs.~(\ref{e9}), (\ref{e10}), (\ref{e11}), and (\ref{e12}), respectively.

\section{Discussion and Conclusions}

\section*{Acknowledgements}
This research was directly funded by the U.S.\ Department of Energy, Office of Science, Office of Fusion Energy Sciences,  under contract DE-FG02-04ER54742, and
incorporates work funded by the U.S.\ Department of Energy, Office of Science, Office of Fusion Energy Sciences, using the DIII-D National Fusion Facility, a DOE Office of Science user facility, under contract DE-FC02-04ER54698.  

\section*{Disclaimer} This report was prepared as an account of work sponsored by an agency of the United States Government. Neither the United States Government nor any agency thereof, nor any of their employees, makes any warranty, express or implied, or assumes any legal liability or responsibility for the accuracy, completeness, or usefulness of any information, apparatus, product, or process disclosed, or represents that its use would not infringe privately owned rights. Reference herein to any specific commercial product, process, or service by trade name, trademark, manufacturer, or otherwise does not necessarily constitute or imply its endorsement, recommendation, or favoring by the United States Government or any agency thereof. The views and opinions of authors expressed herein do not necessarily state or reflect those of the United States Government or any agency thereof. 
 
\section*{Data Availabity Statement}
The data that support the findings of this study are available from the corresponding author upon reasonable request.
 
\section*{References}
\begin{thebibliography}{99}\baselineskip 5ex

\bibitem{rfa} R.~Fitzpatrick, Nucl.\ Fusion {\bf 33}, 1049 (1993).

\bibitem{fkr} H.P.~Furth,  J.~Killeen, and M.N.~Rosenbluth,  Phys.\ Fluids {\bf 6}, 459 (1963).

\bibitem{am1} R.~Fitzpatrick, R.J.~Hastie, T.J.~Martin, and C.M.~Roach, Nucl.\ Fusion {\bf 33}, 1533 (1993).

\bibitem{ruth} P.H.~Rutherford,  Phys.\ Fluids  {\bf 16}, 1903 (1973).

\bibitem{rfb} R.~Fitzpatrick, Phys.\ Plasmas {\bf 5}, 3325 (1998).

\bibitem{rfc} R.~Fitzpatrick, Phys.\ Plasmas {\bf 21}, 092513 (2014).

\bibitem{rf1} R.~Fitzpatrick, Phys.\ Plasmas {\bf 27}, 042506 (2020).

\bibitem{rffig} R.~Fitzpatrick, Phys.\ Plasmas {\bf 10}, 2304 (2003).

\bibitem{ruth1} P.H.~Rutherford, in  {\it Basic Physical Processes of
Toroidal Fusion Plasmas}, Proc.\ Course and Workshop, Varenna, 1985. (Commission of the European Communities, Brussels, 1985.) Vol.~2, p.~531.

\bibitem{slip1} L.G.~Eliseev, N.V.~Ivanov, A.M.~Kakurin, A.V.~Melnikov, and S.V.~Perfilov, Phys.\ Plasmas {\bf 22}, 052504 (2015). 

\bibitem{slip2} W.~Huang, and P.~Zhu, Phys.\ Plasmas {\bf 27}, 022514 (2020).

\bibitem{rfisland} R.~Fitzpatrick, Phys.\ Plasmas {\bf 25}, 082513 (2018);

\bibitem{slip3} A.N.~Chudnovskiy, Y.V.~Gvozdkov, N.V.~Ivanov, A.M.~Kakurin,
A.A.~Medvedev, I.I.~Orlovskiy, Y.D.~Pavlov, V.V.~Piterskiy, V.D.~Pustovitov,
M.B.~Safonova, V.V.~Volkov, and the T-10 Team, Nucl.\ Fusion {\bf 43} 681 (2003).

\bibitem{rftor} R.~Fitzpatrick, and A.O.~Nelson, Phys.\ Plasmas {\bf 27}, 072501 (2020).

\end{thebibliography}

%\begin{figure}
%\includegraphics[height=5in]{fig1.pdf}
%\caption{Overview of DIII-D discharge \#145380.
%(a) Safety factor at  ${\mit\Psi}_N=0.95$. 
%(b) $D_\alpha$ (i.e., Deuterium Balmer-alpha) signal, as well as $n=3$ current flowing in upper and lower sections of I-coil. 
%(c) Pedestal (i.e., ${\mit\Psi}_N=0.94$)
%electron pressure. (The red curve is the running average over 10 ms.) (d) Pedestal electron number density. The common vertical yellow bands indicate the ELM-suppression/mitigation windows.}\label{fig1}
%\end{figure}

\end{document}

