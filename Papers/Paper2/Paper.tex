\documentclass[12pt,prb,aps]{revtex4-1}
\usepackage {amsmath}
\pdfoutput = 1 
\usepackage {graphicx}
\newcommand {\bomega}{\mbox{\boldmath$\omega$}}
\newcommand {\bpi}{\mbox{\boldmath$\pi$}}

\begin{document}

\title {Modelling $q_{95}$ Windows for the Suppression of Edge Localized Modes by Resonant Magnetic Perturbations in the DIII-D Tokamak: Relaxation of the No-Slip
Constraint}

\author{R.~Fitzpatrick\,\footnote{rfitzp@farside.ph.utexas.edu}}
\affiliation{Institute for Fusion Studies,  Department of Physics,  University of Texas at Austin,  Austin TX, 78712, USA}

%\begin{abstract}

%\end{abstract}
 
\maketitle

\section{Introduction}

\section{Cylindrical Model}

\section{Toroidal Model}

\section{Discussion and Conclusions}

\section*{Acknowledgements}
This research was directly funded by the U.S.\ Department of Energy, Office of Science, Office of Fusion Energy Sciences,  under contract DE-FG02-04ER54742, and
incorporates work funded by the U.S.\ Department of Energy, Office of Science, Office of Fusion Energy Sciences, using the DIII-D National Fusion Facility, a DOE Office of Science user facility, under contract DE-FC02-04ER54698.  

\section*{Disclaimer} This report was prepared as an account of work sponsored by an agency of the United States Government. Neither the United States Government nor any agency thereof, nor any of their employees, makes any warranty, express or implied, or assumes any legal liability or responsibility for the accuracy, completeness, or usefulness of any information, apparatus, product, or process disclosed, or represents that its use would not infringe privately owned rights. Reference herein to any specific commercial product, process, or service by trade name, trademark, manufacturer, or otherwise does not necessarily constitute or imply its endorsement, recommendation, or favoring by the United States Government or any agency thereof. The views and opinions of authors expressed herein do not necessarily state or reflect those of the United States Government or any agency thereof. 
 
\section*{Data Availabity Statement}
The data that support the findings of this study are available from the corresponding author upon reasonable request.
 
\section*{References}
\begin{thebibliography}{99}\baselineskip 5ex

%\bibitem{rf1} R.~Fitzpatrick, Phys.\ Plasmas {\bf 27}, 042506 (2020).

%\bibitem{rf2} R.~Fitzpatrick, and A.O.~Nelson, Phys.\ Plasmas {\bf 27}, 072501 (2020).

\bibitem{rfa} R.~Fitzpatrick, Nucl.\ Fusion {\bf 33}, 1049 (1993).

%\bibitem{rfb} R.~Fitzpatrick, Phys.\ Plasmas {\bf 5}, 3325 (1998).

%\bibitem{nl1} R.~Fitzpatrick, and F.L.~Waelbroeck, Phys.\ Plasmas {\bf 12}, 022307 (2005).

%\bibitem{nl2} R.~Fitzpatrick, Phys.\ Plasmas {\bf 25}, 042503 (2018).

%\bibitem{nl3} R.~Fitzpatrick, Phys.\ Plasmas {\bf 25}, 112505 (2018).

%\bibitem{hender} R.~Fitzpatrick, and T.C.~Hender, Phys.\ Fluids B {\bf 3}, 644 (1991).

%\bibitem{cole} A.~Cole, and R.~Fitzpatrick, Phys.\ Plasmas {\bf 13}, 032503 (2006).

\bibitem{fkr} H.P.~Furth,  J.~Killeen, and M.N.~Rosenbluth,  Phys.\ Fluids {\bf 6}, 459 (1963).

%\bibitem{am1} R.~Fitzpatrick, R.J.~Hastie, T.J.~Martin, and C.M.~Roach, Nucl.\ Fusion {\bf 33}, 1533 (1993).

%\bibitem{am3} R.~Fitzpatrick, Phys.\ Plasmas {\bf 24}, 072506 (2017). 

%\bibitem{flat} R.~Fitzpatrick,  Phys.\ Plasmas {\bf 2}, 825 (1995).

\bibitem{ruth} P.H.~Rutherford,  Phys.\ Fluids  {\bf 16}, 1903 (1973).

%\bibitem{ara} G.~Ara,  B.~Basu, B.~Coppi, G.~Laval, M.N.~Rosenbluth, and B.V.~Waddell, Ann.\ Phys.\ (N.Y.) {\bf 112}, 443 (1978). 

%\bibitem{wat} R.~Fitzpatrick, P.G.~Watson, and F.L.~Waelbroeck, Phys.\ Plasmas {\bf 12}, 082510 (2005).

%\bibitem{rffig} R.~Fitzpatrick, Phys.\ Plasmas {\bf 10}, 2304 (2003).

\end{thebibliography}

%\begin{figure}
%\includegraphics[height=5in]{fig1.pdf}
%\caption{Overview of DIII-D discharge \#145380.
%(a) Safety factor at  ${\mit\Psi}_N=0.95$. 
%(b) $D_\alpha$ (i.e., Deuterium Balmer-alpha) signal, as well as $n=3$ current flowing in upper and lower sections of I-coil. 
%(c) Pedestal (i.e., ${\mit\Psi}_N=0.94$)
%electron pressure. (The red curve is the running average over 10 ms.) (d) Pedestal electron number density. The common vertical yellow bands indicate the ELM-suppression/mitigation windows.}\label{fig1}
%\end{figure}

\end{document}

