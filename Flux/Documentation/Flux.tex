\documentclass[12pt]{article}
\usepackage{amsmath}
\usepackage{fullpage}
\usepackage{graphicx}
\allowdisplaybreaks

\title{\bf Program FLUX}
\date{\today}
\author{Richard Fitzpatrick}

\begin{document}
\maketitle

\section{Preliminarly Analysis}
\subsection{Coordinate Systems}
\begin{itemize}
\item Cartesian coordinates: $X$, $Y$, $Z$.
\item Cylindrical coordinates: $R\equiv (X^{\,2}+Y^{\,2})^{1/2}$, $\phi\equiv \tan^{-1}(Y/X)$, $Z$,
so
\begin{equation}
|\nabla\phi| = \frac{1}{R}.
\end{equation}
\item Flux coordinates: $r(R,Z)$, $\theta(R,Z)$, $\phi$, where
\begin{align}\label{e1}
(\nabla r\cdot\nabla\theta\times \nabla\phi)^{-1} &= {\cal J},\\[0.5ex]
{\cal J} &=\frac{r\,R^{\,2}}{R_0},\label{e2}
\end{align}
where $R_0$ is a convenient scale major radius, flux-surface label $r$ has units of length, and $\theta$ is
a poloidal angle. $\theta=0$ on inboard midplane. $\theta>0$ above midplane. 
\end{itemize}

\subsection{Useful Identities}\label{s2}
Easily demonstrated that:
\begin{align}
{\bf A} &= A^{\,r}\,{\cal J}\,\nabla\theta\times\nabla\phi
+  A^{\,\theta}\,{\cal J}\,\nabla\phi\times\nabla r
+  A^{\,\phi}\,{\cal J}\,\nabla r\times\nabla\theta,\\[0.5ex]
{\bf A} &= A_r\,\nabla r+A_\theta\,\nabla\theta+A_\phi\,\nabla\phi,\\[0.5ex]
{\bf A}\cdot{\bf B} &= A_r\,B^{\,r}+A_\theta \,B^{\,\theta}+A_\phi\,B^{\,\phi}=A^{\,r}\,B_r+A^{\,\theta}\,B_{\theta}+A^{\,\phi}\,B_\phi,\\[0.5ex]
({\bf A}\times {\bf B})_r &= {\cal J}\,(A^{\,\theta}\,B^{\,\phi}-A^{\,\phi}\,B^{\,\theta}),\\[0.5ex]
({\bf A}\times {\bf B})_\theta &= {\cal J}\,(A^{\,\phi}\,B^{\,r}-A^{\,r}\,B^{\,\phi}),\\[0.5ex]
({\bf A}\times {\bf B})_\phi &= {\cal J}\,(A^{\,r}\,B^{\,\theta}-A^{\,\theta}\,B^{\,r}),\\[0.5ex]
{\cal J}\,({\bf A}\times {\bf B})^{\,r} &= A_{\theta}\,B_{\phi}-A_{\phi}\,B_{\theta},\\[0.5ex]
{\cal J}\,({\bf A}\times {\bf B})^{\,\theta} &= A_{\phi}\,B_{r}-A_{r}\,B_{\phi},\\[0.5ex]
{\cal J}\,({\bf A}\times {\bf B})^{\,\phi} &=A_{r}\,B_{\theta}-A_{\theta}\,B_{r},\\[0.5ex]
{\cal J}\,\nabla \cdot{\bf A} &= \frac{\partial\,({\cal J}\,A^{\,r})}{\partial r} + 
\frac{\partial\,({\cal J}\,A^{\,\theta})}{\partial \theta}+ 
\frac{\partial\,({\cal J}\,A^{\,\phi})}{\partial \phi},\\[0.5ex]
{\cal J}\,(\nabla\times {\bf A})^{\,r}&= \frac{\partial A_\phi}{\partial \theta}
-\frac{\partial A_\theta}{\partial \phi},\\[0.5ex]
{\cal J}\,(\nabla\times {\bf A})^{\,\theta}&= \frac{\partial A_r}{\partial \phi}
-\frac{\partial A_\phi}{\partial r},\\[0.5ex]
{\cal J}\,(\nabla\times {\bf A})^{\,\phi}&= \frac{\partial A_\theta}{\partial r}
-\frac{\partial A_r}{\partial \theta}.
\end{align}

\subsection{Equilibrium Magnetic Field}
Equilibrium magnetic field:
\begin{align}
{\bf B}& =B_0\,R_0\left[ f(r)\,\nabla\phi\times \nabla r + g(r)\,\nabla\phi\right],\\[0.5ex]
q(r)&= \frac{r\,g}{R_0\,f},
\end{align}
where $B_0$ is a convenient scale magnetic field strength, and $g$ and $f$ are dimensionless.
So,
\begin{align}
B^{\,r} &=0,\\[0.5ex]
B^{\,\theta}&=B_0\,R_0^{\,2}\, \frac{f}{r\,R^{\,2}},\\[0.5ex]
B^{\,\phi} &= B_0\,R_0^{\,2}\,\frac{q\,f}{r\,R^{\,2}}= B_0\,B_0\,\frac{g}{R^{\,2}},\\[0.5ex]
B_r&= - B_0\,r\,f\,\nabla r\cdot\nabla\theta,\\[0.5ex]
B_\theta &= B_0\,r\,f\,|\nabla r|^{\,2},\\[0.5ex]
B_\phi &= B_0\,R_0^{\,2}\,\frac{q\,f}{r} = B_0\,R_0\,g,
\end{align}
and
\begin{equation}
{\bf B}\cdot\nabla = B_0\,R_0^{\,2}\,
\frac{f}{r\,R^{\,2}}\left(\frac{\partial}{\partial \theta}+q\,\frac{\partial}{\partial\phi}\right).
\end{equation}

\subsection{Equilibrium Plasma Current}
The relation $\mu_0\,{\bf J} = \nabla\times{\bf B}$ yields 
\begin{align}
\mu_0\,{\cal J}\,J^{\,r}&= 0,\\[0.5ex]
\mu_0\,{\cal J}\,J^{\,\theta} &= - B_0\,R_0\,\frac{dg}{dr},\\[0.5ex]
\mu_0\,{\cal J}\,J^{\,\phi} &= B_0\,\frac{\partial}{\partial r}\!\left(r\,f\,|\nabla r|^{\,2}\right) +B_0\,\frac{\partial}{\partial\theta}\!\left(r\,f\,\nabla r\cdot\nabla\theta\right).
\end{align} 

\subsection{Grad-Shafranov Equation}
The equilibrium force balance relation
\begin{equation}
{\bf J}\times{\bf B} = \nabla P
\end{equation}
yields
\begin{equation}
{\cal J}\,(J^{\,\theta}\,B^{\,\phi}- J^{\,\phi}\,B^{\,\theta})= \frac{dP}{dr},
\end{equation}
which reduces to the Grad-Shafranov equation, 
\begin{equation} 
\frac{f}{r}\,\frac{\partial}{\partial r}\!\left(r\,f\,|\nabla r|^{\,2}\right) + \frac{f}{r}\,\frac{\partial}{\partial \theta}\!\left(r\,f\,\nabla r\cdot\nabla\theta\right) + g\,\frac{dg}{dr} + \frac{\mu_0}{B_0^{\,2}}\left(\frac{R}{R_0}\right)^2\frac{dP}{dr} = 0.
\end{equation}

\subsection{Inhomogeneous Tearing Mode Dispersion Relation}
\subsection{High-$q$ Limit}
Now,
\begin{equation}
\nabla \cdot \delta{\bf B}=0
\end{equation}
yields
\begin{equation}\label{e38}
\frac{\partial ({\cal J}\,\delta B^{\,r})}{\partial r}+ \frac{\partial({\cal J}\,\delta B^{\,\theta})}{\partial\theta} +\frac{\partial({\cal J}\,\delta B^{\,\phi})}{\partial\phi} =0.
\end{equation}

Furthermore,
\begin{equation}
\delta {\bf B} = \delta B_r\,\nabla r + \delta B_\theta\,\nabla\theta
+\delta B_\phi\,\nabla\phi.
\end{equation}
So,
\begin{align}
\delta B^{\,r} &=\delta {\bf B}\cdot \nabla r = |\nabla r|^{\,2}\,\delta  B_r+
(\nabla r\cdot\nabla\theta)\,\delta B_\theta,\\[0.5ex]
\delta B^{\,\theta} &= \delta {\bf B}\cdot\nabla\theta = (\nabla r\cdot\nabla\theta)\,\delta B_r
+|\nabla \theta|^{\,2}\,\delta B_\theta,\\[0.5ex]
\delta B^{\,\phi}&=\delta {\bf B}\cdot\nabla\phi = \frac{\delta B_\phi}{R^{\,2}}.
\end{align}
Follows that
\begin{equation}\label{e43}
\delta B_r = \left(\frac{1}{|\nabla r|^{\,2}}\right)\delta B^{\,r}-\left(
\frac{\nabla r\cdot\nabla\theta}{|\nabla r|^{\,2}}\right)\delta B_\theta,
\end{equation}
and
\begin{equation}\label{e44}
\delta B^{\,\theta}=\left(\frac{\nabla r\cdot\nabla\theta}{|\nabla r|^{\,2}}\right)
\delta B^{\,r}+ \left[|\nabla\theta|^{\,2}-\frac{(\nabla r\cdot\nabla\theta)^{\,2}}{|\nabla r|^{\,2}}\right]\delta B_\theta.
\end{equation}
But, Eqs.~(\ref{e1}) and (\ref{e2}) imply that
\begin{equation}
|\nabla r|^{\,2}\,|\nabla\theta|^{\,2}-(\nabla r\cdot\nabla\theta)^{\,2} = \frac{R_0^{\,2}}{r^{\,2}\,R^{\,2}}.
\end{equation}
Hence,
\begin{equation}\label{e46}
\delta B^{\,\theta} = \left(\frac{\nabla r\cdot\nabla\theta}{|\nabla r|^{\,2}}\right)
\delta B^{\,r}+ \left(\frac{R_0^{\,2}}{r^{\,2}\,R^{\,2}\,|\nabla r|^{\,2}}\right)\delta B_\theta.
\end{equation}

Assume that high-$q$ limit is equivalent to
 $\delta {\bf B}$ being curl-free.
Follows that
\begin{align}
\frac{\partial\,\delta B_\phi}{\partial\theta}&=\frac{\partial \,\delta B_\theta}{\partial\phi},\label{e52}\\[0.5ex]
 \frac{\partial\,\delta B_r}{\partial\phi}&=\frac{\partial \,\delta B_\phi}{\partial r},\\[0.5ex]
 \frac{\partial\,\delta B_\theta}{\partial r}&= \frac{\partial \,\delta B_r}{\partial \theta}.\label{e51}
\end{align}

Previous three equations imply that
\begin{equation}
\delta B_\phi \sim \frac{n}{m}\,\delta B_\theta,\, \frac{n}{m}\,r\,\delta B_r
\end{equation}
and, hence, that 
\begin{equation}
\delta B^{\,\phi} \sim \frac{n}{m}\,\left(\frac{r}{R}\right)^2 \delta B^{\,\theta},\, \frac{n}{m}\,\left(\frac{r}{R}\right)^2 \frac{\delta B^{\,r}}{r}, 
\end{equation}
Consequently,  final term on right-hand side of (\ref{e38}) is of order $(n/m)^2\,(r/R)^2$ smaller than other two terms, and, therefore, negligible. 
Thus, we get 
\begin{equation}
\frac{\partial ({\cal J}\,\delta B^{\,r})}{\partial r}+ \frac{\partial({\cal J}\,\delta B^{\,\theta})}{\partial\theta}  =0.
\end{equation}
Follows from (\ref{e46}) and previous equation that
\begin{equation}
r\,\frac{\partial}{\partial r}\left(\frac{r\,R^{\,2}\,\delta B^{\,r}}{R_0^{\,2}}\right)
=-\frac{\partial}{\partial\theta}\!\left[\left(\frac{r\,\nabla r\cdot\nabla \theta}{|\nabla r|^{\,2}}\right)\left(\frac{r\,R^{\,2}\,\delta B^{\,r}}{R_0^{\,2}}\right) +\left(\frac{1}{|\nabla r|^{\,2}}\right)\delta B_\theta\right].
\end{equation}
Follows from (\ref{e2}), (\ref{e43}) and (\ref{e51}) that
\begin{equation}
r\,\frac{\partial\,\delta B_\theta}{\partial r} =\frac{\partial}{\partial\theta}\!\left[\left(\frac{R_0^{\,2}}{R^{\,2}\,|\nabla r|^{\,2}}\right)\left(\frac{r\,R^{\,2}\,\delta B^{\,r}}{R_0^{\,2}}\right)
-\left(\frac{r\,\nabla r\cdot\nabla\theta}{|\nabla r|^{\,2}}\right)\delta B_\theta\right].
\end{equation}
Let
\begin{align}
\frac{r\,R^{\,2}\,\delta B^{\,r}(r,\theta,\phi)}{R_0^{\,2}}&= \sum_j \delta\hat{B}_j^{\,r}(r)\,{\rm e}^{\,{\rm i}\,(m_j\,\theta-n\phi)},\\[0.5ex]
\delta B_\theta (r,\theta,\phi)&=\sum_j \delta\hat{B}_{\theta\,j}(r)\,{\rm e}^{\,{\rm i}\,(m_j\,\theta-n\,\phi)}.
\end{align}
Follows that
\begin{align}
\sum_{j'} r\,\frac{d\,\delta \hat{B}^{\,r}_{j'}}{dr}\,{\rm e}^{\,{\rm i}\,m_{j'}\,\theta}&=-\frac{\partial}{\partial\theta}\sum_{j'} \left[\left(\frac{r\,\nabla r\cdot\nabla \theta}{|\nabla r|^{\,2}}\right)\,{\rm e}^{\,{\rm i}\,m_{j'}\,\theta}\,\delta \hat{B}^{\,r}_{j'} +\left(\frac{1}{|\nabla r|^{\,2}}\right)\,{\rm e}^{\,{\rm i}\,m_{j'}\,\theta}\,\delta \hat{B}_{\theta\,j'}\right],\\[0.5ex]
\sum_{j'} r\,\frac{d\,\delta \hat{B}_{\theta\,j'}}{dr}\,{\rm e}^{\,{\rm i}\,m_{j'}\,\theta}
&= \frac{\partial}{\partial\theta}\sum_{j'}\left[\left(\frac{R_0^{\,2}}{R^{\,2}\,|\nabla r|^{\,2}}\right)\,{\rm e}^{\,{\rm i}\,m_{j'}\,\theta}\,\delta \hat{B}^{\,r}_{j'}
-\left(\frac{r\,\nabla r\cdot\nabla\theta}{|\nabla r|^{\,2}}\right){\rm e}^{\,{\rm i}\,m_{j'}\,\theta}\,\delta \hat{B}_{\theta\,j'}\right].
\end{align}
Operating with $\oint(\cdots)\, {\rm e}^{-{\rm i}\,m_j\,\theta}\,d\theta/(2\pi)$, we
get
\begin{align}
r\,\frac{d\,\delta \hat{B}_j^{\,r}}{dr}
&= -{\rm i}\,m_j\sum_{j'}\left(-{\rm i}\,c_{jj'}\,\delta \hat{B}_{j'}^{\,r}
+a_{jj'}\,\delta \hat{B}_{\theta\,j'}\right),\\[0.5ex]
r\,\frac{d\,\delta \hat{B}_{\theta\,j}}{dr}
&= -{\rm i}\,m_j\sum_{j'}\left(-{\rm i}\,c_{jj'}\,\delta \hat{B}_{\theta\,j'}
-b_{jj'}\,\delta \hat{B}_{j'}^{\,r}\right),
\end{align}
where
\begin{align}
a_{jj'} &= \oint \frac{1}{|\nabla r|^{\,2}}\,
{\rm e}^{-{\rm i}\,(m_j-m_{j'})\,\theta}
\,\frac{d\theta}{2\pi},\\[0.5ex]
b_{jj'}& = \oint \frac{R_0^{\,2}}{R^{\,2}\,|\nabla r|^{\,2}}\,
{\rm e}^{-{\rm i}\,(m_j-m_{j'})\,\theta}
\,\frac{d\theta}{2\pi},\\[0.5ex]
c_{jj'} &= \oint \frac{{\rm i}\,r\,\nabla r\cdot\nabla \theta}{|\nabla r|^{\,2}}\,
{\rm e}^{-{\rm i}\,(m_j-m_{j'})\,\theta}
\,\frac{d\theta}{2\pi}.
\end{align}
Let
\begin{align}
\delta \hat{B}_{j}^{\,r}(r) &={\rm i}\,\psi_j(r),\\[0.5ex]
\delta \hat{B}_{\theta\,j}(r)&= -\chi_j(r).
\end{align}
It follows that
\begin{align}
r\,\frac{d\psi_j}{dr} &=m_j\sum_{j'}\left(-c_{jj'}\,\psi_{j'} + a_{jj'}\,\chi_{j'}\right),\\[0.5ex]
r\,\frac{d\chi_j}{dr} &=m_j\sum_{j'}\left(-c_{jj'}\,\chi_{j'} + b_{jj'}\,\psi_{j'}\right).
\end{align}
Hence,
\begin{align}
\frac{r\,R^{\,2}\,\delta B^{\,r}(r,\theta,\phi)}{R_0^{\,2}}&= {\rm i}\sum_j \psi_j(r)\,{\rm e}^{\,{\rm i}\,(m_j\,\theta-n\phi)},\label{e86}\\[0.5ex]
\frac{r\,R^{\,2}\,\delta B^{\,\theta}(r,\theta,\phi)}{R_0^{\,2}}&=- \sum_j \frac{1}{m_j}\,\frac{d\psi_j}{dr}\,{\rm e}^{\,{\rm i}\,(m_j\,\theta-n\,\phi)},\label{e90}\\[0.5ex]
R^{\,2}\,\delta B^{\,\phi}(r,\theta,\phi)&= n\sum_j\frac{\chi_j(r)}{m_j}\,{\rm e}^{\,{\rm i}\,(m_j\,\theta-n\,\phi)}.\label{e91}
\end{align}
where use has been made of (\ref{e52}). Also have
\begin{align}
\delta B_r(r,\theta,\phi)&= {\rm i}\sum_j \frac{1}{m_j}\,\frac{d\chi_j}{dr}\,{\rm e}^{\,{\rm i}\,(m_j\,\theta-n\,\phi)},\\[0.5ex]
\delta B_\theta (r,\theta,\phi)&=-\sum_j\chi_j(r)\,{\rm e}^{\,{\rm i}\,(m_j\,\theta-n\,\phi)},\\[0.5ex]
\delta B_\phi(r,\theta,\phi)&=n\sum_j\frac{\chi_j(r)}{m_j}\,{\rm e}^{\,{\rm i}\,(m_j\,\theta-n\,\phi)},
\end{align}

We can write
\begin{align}
{\cal J}\,\mu_0\,\delta J^{\,r} = \frac{\partial\,\delta B_\phi}{\partial\theta} -\frac{\partial \,\delta B_\theta}{\partial\phi},\\[0.5ex]
{\cal J}\,\mu_0\,\delta J^{\,\theta} = \frac{\partial\,\delta B_r}{\partial\phi} -\frac{\partial \,\delta B_\phi}{\partial r},\\[0.5ex]
{\cal J}\,\mu_0\,\delta J^{\,\phi} = \frac{\partial\,\delta B_\theta}{\partial r} -\frac{\partial \,\delta B_r}{\partial\theta}.
\end{align}
Normally, all three components are zero. However, at $k$th resonant 
surface (at which $r=r_k$, where $q(r_k)=m_k/n$) $\psi_k$, $\psi_{j\neq k}$, and $\chi_{j\neq k}$ are continuous, whereas $\chi_k$ is discontinuous. Hence,
\begin{align}
{\cal J}\,\mu_0\,\delta J^{\,r}(r,\theta,\phi) &= 0,\\[0.5ex]
{\cal J}\,\mu_0\,\delta J^{\,\theta}(r,\theta,\phi) &=-\sum_k \frac{n}{m_k}\,\left[\chi_k
\right]_{r_{k-}}^{r_{k+}}\,\delta(r-r_k)\,{\rm e}^{\,{\rm i}\,(m_k\,\theta-n\,\phi)},\\[0.5ex]
{\cal J}\,\mu_0\,\delta J^{\,\phi}(r,\theta,\phi) &=-\sum_k \left[\chi_k\right]_{r_{k-}}^{r_{k+}}\,\delta(r-r_k)\,{\rm e}^{\,{\rm i}\,(m_k\,\theta-n\,\phi)}.\label{e97}
\end{align}

Now,
\begin{align}
(\mu_0\,\delta{\bf J}\times \delta {\bf B})_\theta&= \frac{1}{4}\left({\cal J}\,\mu_0\,\delta J^{\,\phi}\,\delta B^{\,r\,\ast}+{\rm c.c.}\right),\\[0.5ex]
(\mu_0\,\delta{\bf J}\times \delta {\bf B})_\phi&= \frac{1}{4}\left(-{\cal J}\,{\mu_0}\,\delta J^{\,\theta}\,\delta B^{\,r\,\ast}+{\rm c.c.}\right),
\end{align}
which implies that
\begin{align}
{\cal J}\,(\mu_0\,\delta{\bf J}\times \delta {\bf B})_\theta&=\frac{R_0}{2}\sum_j\sum_k {\rm Re}\left\{{\rm i}\,[\chi_k]_{r_{k-}}^{r_{k+}}\,\psi_j^{\,\ast}(r_k)\,{\rm e}^{\,{\rm i}\,(m_k-m_j)\,\theta}\right\}\delta(r-r_k),\\[0.5ex]
{\cal J}\,(\mu_0\,\delta {\bf J}\times \delta {\bf B})_\phi&= -\frac{R_0}{2}\sum_j\sum_k \frac{n}{m_k}\, {\rm Re}\left\{{\rm i}\,[\chi_k]_{r_{k-}}^{r_{k+}}\,\psi_j^{\,\ast}(r_k)\,{\rm e}^{\,{\rm i}\,(m_k-m_j)\,\theta}\right\}\delta(r-r_k).
\end{align}
Let
\begin{align}
{\mit\Psi}_k      &= \frac{\psi_k(r_k)}{m_k},\label{e102}\\[0.5ex]
{\mit\Delta\Psi}_k&=\left[\chi_k\right]_{r_{k-}}^{r_{k+}},\label{e103}\\[0.5ex]
\delta T_{\theta\,k} &\equiv \delta {\bf T}\cdot{\bf e}_\theta= \int_{r_{k-}}^{r_{k+}}\oint\oint (\delta {\bf J}\times \delta {\bf B})_\theta\,{\cal J}\,dr\,d\theta\,d\phi,\\[0.5ex]
\delta T_{\phi\,k} &\equiv \delta {\bf T}\cdot{\bf e}_\phi= \int_{r_{k-}}^{r_{k+}}\oint\oint (\delta {\bf J}\times \delta {\bf B})_\phi\,{\cal J}\,dr\,d\theta\,d\phi,
\end{align}
where $\delta{\bf T}(r)\,dr$ is the net electromagnetic torque acting on the plasma between $r$ and $r+dr$.  Here, ${\bf e}_\theta= (R^{\,2}/R_0)\,\nabla\phi\times\nabla r$ and
${\bf e}_\phi = R\,\nabla\phi$. 
Follows that
\begin{align}
\delta T_{\theta\,k} &= -\frac{2\pi^{\,2}\,R_0\,m_k}{\mu_0}\,{\rm Im}\left({\mit\Psi}_k^{\,\ast}\,{\mit\Delta\Psi}_k\right),\label{et1}\\[0.5ex]
\delta T_{\phi\,k} &= \frac{2\pi^{\,2}\,R_0\,n}{\mu_0}\,{\rm Im}\left({\mit\Psi}_k^{\,\ast}\,{\mit\Delta\Psi}_k\right).\label{et2}
\end{align}

\subsection{Homogeneous Solution}
We can write
\begin{equation}
\delta {\bf B} = \nabla \times \delta {\bf A}.
\end{equation}
Suppose that $r\,\delta A_r$, $\delta A_\theta$ are negligible with respect to
$\delta A_\phi$. In fact, it is easily demonstrated from $\nabla\cdot\delta {\bf A}=0$ that $r\,\delta A_r$, $\delta A_\theta\sim (n/m)\,(r/R)^{\,2}\,\delta A_\phi$. 
 It follows that
\begin{align}
{\cal  J}\,\delta B^{\,r} &\simeq \frac{\partial\,\delta A_\phi}{\partial\theta},\\[0.5ex]
{\cal J}\,\delta B^{\,\theta} &\simeq - \frac{\partial\,\delta A_\phi}{\partial r},\\[0.5ex]
{\cal J}\,\delta B^{\,\phi} &\simeq 0.
\end{align}
The neglected terms in the previous three equations are $(n/m)^2\,(r/R)^2$ smaller than the dominant terms. 
The previous three expressions are consistent with (\ref{e86}), (\ref{e90}), and (\ref{e91}) provided
\begin{equation}
\delta A_\phi(r,\theta,\phi) \simeq R_0\sum_j\frac{\psi_j(r)}{m_j} \,{\rm e}^{\,{\rm i}\,(m_j\,\theta-n\,\phi)}.\label{e112}
\end{equation}

According to the Biot-Savart law:
\begin{equation}
\delta A_{\,\phi}({\bf x}) = \frac{1}{4\pi}\int \frac{R\,R'\,\mu_0\,\delta {\bf J}({\bf x'})\cdot\nabla\phi}{|{\bf x}-{\bf x}'|}\,d^{\,3}{\bf x}'.
\end{equation}
Assume that
\begin{equation}
\delta A_\phi (R,\phi,Z) = \delta A_\phi(R,0,Z)\,{\rm e}^{-{\rm i}\,n\,\phi}.
\end{equation}
Hence, can evaluate integral at $\phi=0$ without loss of generality. 
Follows that
\begin{align}
{\bf x} &= (R,\,0,\,Z),\\[0.5ex]
{\bf x'}&= (R'\,\cos\phi',\,R'\,\sin\phi',\,Z'),
\end{align}
and
\begin{equation}
|{\bf x}-{\bf x}'| = \left[R^{\,2}+R'^{\,2} +(Z-Z')^{\,2} -2\,R\,R'\,\cos\phi'\right]^{\,1/2}.
\end{equation}
Now,
\begin{equation}
\delta {\bf J}(R',\phi',Z')\cdot\nabla\phi= \delta J^{\,\phi}(R',0,Z')\,{\rm e}^{-{\rm i}\,n\,\phi'}\,\cos\phi',
\end{equation}
so
\begin{equation}
\delta A_\phi(R,0,Z)= \frac{1}{4\pi}\int_0^\infty
\oint\oint \frac{R\,R'\,\mu_0\,\delta J^{\,\phi}(R',0,Z')\,{\rm e}^{-{\rm i}\,n\,\phi'}\,\cos\phi'\,{\cal J}'\,dr'\,d\theta'\,d\phi'}{\left[R^{\,2}+R'^{\,2} +(Z-Z')^{\,2} -2\,R\,R'\,\cos\phi'\right]^{\,1/2}},
\end{equation}
which can be written
\begin{equation}
\delta A_\phi(R,0,Z)= \frac{1}{4\pi}\int_0^\infty
\oint R\,R'\,\mu_0\,\delta J^{\,\phi}(R',0,Z')\,G(R,Z;R',Z')\,{\cal J}'\,dr'\,d\theta',
\end{equation}
where
\begin{equation}
G(R,Z;R',Z') = \oint \frac{\cos \phi'\,\cos(n\,\phi')\,d\phi'}{\left[R^{\,2}+R'^{\,2} +(Z-Z')^{\,2} -2\,R\,R'\,\cos\phi'\right]^{\,1/2}},
\end{equation}
or
\begin{equation}
G(R,Z;R',Z') = \frac{1}{2}\oint \frac{\left(\cos[(n-1)\,\phi']+\cos[(n+1)\,\phi']\right)\,d\phi'}{\left[R^{\,2}+R'^{\,2} +(Z-Z')^{\,2} -2\,R\,R'\,\cos\phi'\right]^{\,1/2}}.
\end{equation}

Now, 
\begin{align}
P^{\,n}_{-1/2}(\cosh\eta) &= \frac{(-1)^{\,n}}{2\pi}\,\frac{\Gamma(1/2)}{\Gamma(1/2-n)}\oint \frac{\cos(n\,\varphi)\,d\varphi}{(\cosh\eta+\sinh\eta\,\cos\varphi)^{\,1/2}}\nonumber\\[0.5ex]
 &=\frac{1}{2\pi}\,\frac{\Gamma(1/2)}{\Gamma(1/2-n)}\oint \frac{\cos(n\,\varphi)\,d\varphi}{(\cosh\eta-\sinh\eta\,\cos\varphi)^{\,1/2}}\nonumber\\[0.5ex]
 &=\frac{(-1)^{\,n}\,\Gamma(1/2)\,\Gamma(1/2+n)}{2\pi^{\,2}}\oint
 \frac{\cos(n\,\varphi)\,d\varphi}{(\cosh\eta-\sinh\eta\,\cos\varphi)^{\,1/2}}.
\end{align}
Let
\begin{align}
\alpha\,\cosh\eta &= R^{\,2} +R'^{\,2} + (Z-Z')^{\,2},\\[0.5ex]
\alpha\,\sinh\eta &= 2\,R\,R'.
\end{align}
Hence,
\begin{equation}
\eta = \tanh^{-1}\left[\frac{2\,R\,R'}{R^{\,2}+R'^{\,2}+(Z-Z')^{\,2}}\right],
\end{equation}
and
\begin{equation}
\alpha = \frac{R^{\,2} +R'^{\,2} + (Z-Z')^{\,2}}{\cosh\eta}.
\end{equation}
It follows that
\begin{align}
G(R,Z;R',Z')&=\pi\left[\frac{\cosh\eta}{R^{\,2}+R'^{\,2}+(Z-Z')^{\,2}}\right]^{1/2}\nonumber\\[0.5ex]
&
\times\left[\frac{\Gamma(3/2-n)}{\Gamma(1/2)}\,P_{-1/2}^{\,n-1}(\cosh\eta)
+ \frac{\Gamma(-1/2-n)}{\Gamma(1/2)}\,P_{-1/2}^{\,n+1}(\cosh\eta)\right].
\end{align}
However,
\begin{align}
{\Gamma}(3/2-n)  &=\frac{(-1)^{\,n+1}\,\pi\,(n-1/2)}{\Gamma(n+1/2)},\\[0.5ex]
{\Gamma}(-1/2-n)&= \frac{(-1)^{\,n+1}\,\pi}{\Gamma(n+1/2)\,(n+1/2)},
\end{align}
so
\begin{align}
G(R,Z;R',Z')&= \frac{(-1)^{\,n+1}\,\pi^{\,2}}{\Gamma(1/2)\,\Gamma(n+1/2)}
\left[\frac{\cosh\eta}{R^{\,2}+R'^{\,2}+(Z-Z')^{\,2}}\right]^{1/2}\nonumber\\[0.5ex]
&\times\left[(n-1/2)\,P_{-1/2}^{\,n-1}(\cosh\eta)+
\frac{P_{-1/2}^{\,n+1}(\cosh\eta)}{n+1/2}\right].
\end{align}

Now, from (\ref{e97}) and (\ref{e103}), 
\begin{equation}
{\cal J}\,\mu_0\,\delta J^{\,\phi}(r,\theta,0)=-\sum_k {\mit\Delta\Psi}_k\,\delta (r-r_k)\,{\rm e}^{\,{\rm i}\,m_k\,\theta}.
\end{equation}
Furthermore, from (\ref{e102}) and (\ref{e112}), 
\begin{equation}
{\mit\Psi}_k = \frac{1}{R_0}\oint \delta A_{\phi}(r_k,\theta,0)\,{\rm e}^{-{\rm i}\,m_k\,\theta}\,\frac{d\theta}{2\pi}.
\end{equation}
Hence, we obtain the homogeneous tearing mode dispersion relation, 
\begin{equation}
{\mit\Psi}_k = \sum_{k'} F_{kk'}\,{\mit\Delta\Psi}_{k'},
\end{equation}
where
\begin{equation}
F_{kk'} = \oint\oint 
{\cal G}(R_k,Z_k;R_k',Z_k')\,{\rm e}^{-{\rm i}\,(m_k\,\theta-m_{k'}\,\theta')}\,\frac{d\theta}{2\pi}\,\frac{d\theta'}{2\pi},
\end{equation}
and
\begin{align}
{\cal G}(R,Z;R',Z')&= \frac{(-1)^{\,n}\,\pi^{\,2}\,R\,R'/R_0}{2\,\Gamma(1/2)\,\Gamma(n+1/2)}
\left[\frac{\cosh\eta}{R^{\,2}+R'^{\,2}+(Z-Z')^{\,2}}\right]^{1/2}\nonumber\\[0.5ex]
&\times\left[(n-1/2)\,P_{-1/2}^{\,n-1}(\cosh\eta)+
\frac{P_{-1/2}^{\,n+1}(\cosh\eta)}{n+1/2}\right].
\end{align}
Here, $R_k$, $Z_k$ are the $R$. $Z$ coordinates of the $k$th resonant surface in the plane $\phi=0$. 
Note that
\begin{equation}
{\cal G} (R',Z';R,Z) = {\cal G}(R,Z;R',Z'),
\end{equation}
which implies that
\begin{equation}
F_{k'k} = F_{kk'}^{\,\ast}.
\end{equation}

\subsection{Inhomogeneous Solution}
Suppose that there are currents flowing in a number of external poloidal field
coils. Let $I_l$, $R_l$, and $Z_l$ be the peak current, and coordinates of the
$l$th field coil. The currents are assumed to modulate like ${\rm e}^{-{\rm i}\,n\,\phi}$. It follows that
\begin{equation}
\delta J^{\,\phi}_{\rm ext}(R,0,Z)=\sum_j\frac{I_l}{R_l}\,\delta(R-R_l)\,\delta (Z-Z_l).
\end{equation}
Hence,
\begin{equation}
{\mit\Psi}_k = \sum_{k'} F_{kk'}\,{\mit\Delta\Psi}_k - \sum_lg_{kl}\,\mu_0\,I_l,
\end{equation}
where
\begin{equation}
g_{kj} = \frac{1}{2\pi}\oint {\cal G}(r_k,\theta;R_j,Z_j)\,{\rm e}^{-{\rm i}\,m_k\,\theta}\,\frac{d\theta}{2\pi}.
\end{equation}
Thus, we obtain the inhomogeneous tearing mode dispersion relation, 
\begin{equation}
{\mit\Delta\Psi}_k =\sum_{k'} E_{kk'}\,{\mit\Psi}_{k'}+ |E_{kk}|\,\chi_k, \label{e68}
\end{equation}
where $E_{kk'}$ is the inverse of the $F_{kk'}$ matrix, 
and 
\begin{align}
\chi_k&=\frac{1}{|E_{kk}|} \sum_j h_{kl}\,\mu_0\,I_l,\\[0.5ex]
h_{kl} &= \sum_{k'} E_{kk'} \,g_{k'l}. 
\end{align}

\subsection{Perturbed Poloidal Magnetic Field}
The perturbed poloidal magnetic field generated by the
current sheets inside the plasma and the currents flowing
in the external field coils can be expressed as 
\begin{equation}
\delta {\bf B}_p = \nabla \delta A_\phi\times\nabla\phi,
\end{equation}
where
\begin{equation}
\frac{\delta A_\phi(R,\phi,Z)}{R_0} = {\rm Re}\left[
\sum_{k=1,K} e_k(R,Z)\,{\mit\Delta\Psi}_k\,{\rm e}^{-{\rm i}\,n\,\phi}
-\sum_{l=1,L} f_l(R,Z)\,\mu_0\,I_l\,{\rm e}^{-{\rm i}\,n\,\phi}\right],
\end{equation}
and
\begin{align}
e_k(R,Z) &= \oint {\cal G}(R,Z;R_k;Z_k)\,{\rm e}^{\,{\rm i}\,m_k\,\theta_k}\,\frac{d\theta_k}{2\pi},\label{e65a}\\[0.5ex]
f_l(R,Z) &=  \frac{1}{2\pi}\,{\cal G}(R,Z;R_l,Z_l).
\end{align}
As before, $k$ indexes the resonant surfaces within the plasma, whereas $l$ indexes the external current filaments.  Moreover, the integral in Eq.~(\ref{e65a}) is taken around the
$k$th resonant surface.

\subsection{Electromagnetic Torques in Presence of External Currents}
Suppose that
\begin{align}
{\mit\Psi}_k &= B_0\,R_0\,\hat{\mit\Psi}_k\,{\rm e}^{-{\rm i}\,\varphi_k},\\[0.5ex]
\chi_k &= B_0\,R_0\,\hat{\chi}_{k}\,{\rm e}^{-{\rm i}\,\zeta_k},\\[0.5ex]
E_{kk'} &= \hat{E}_{kk'}\,{\rm e}^{-{\rm i}\,\xi_{kk'}},
\end{align}
where $\hat{\mit\Psi}_k$, $\hat{\chi}_{k}$, and $\hat{E}_{kk'}$
are real and positive, whereas $\varphi_k$, $\zeta_k$, and
$\xi_{kk'}$ are real. 
(Note that all hatted quantities in this report are dimensionless.)
It follows from Eqs.~(\ref{et1}), (\ref{et2}), and (\ref{e68})
that
\begin{align}
\delta T_{\theta\,k} &=-\left(\frac{2\pi^{\,2}\,B_0^{\,2}\,R_0^{\,3}}{\mu_0}\right)m_k\,\delta \hat{T}_k,\\[0.5ex]
\delta T_{\phi\,k}&= \left(\frac{2\pi^{\,2}\,B_0^{\,2}\,R_0^{\,3}}{\mu_0}\right)n\,\delta \hat{T}_k,
\end{align}
where
\begin{equation}
\delta \hat{T}_k= \sum_{k'=1,K}\hat{E}_{kk'}\,\hat{\mit\Psi}_k\,\hat{\mit\Psi}_{k'}\,\sin(\varphi_k-\varphi_{k'}-\xi_{kk'}) + \hat{E}_{kk}\,\hat{\mit\Psi}_k\,\hat{\chi}_{k}\,\sin(\varphi_k-\zeta_k).
\end{equation}

\subsection{GPEC Coupling}
\subsubsection{PRL Derivation}
We have
\begin{align}
\frac{d\psi_p}{dr} &= B_0\,R_0\,f(r),\\[0.5ex]
{\bf B}\cdot\nabla \phi &= \frac{B_0\,R_0}{R^{\,2}}\,g(r),\\[0.5ex]
\delta B^{\,r}&= \frac{{\rm i}}{r}\left(\frac{R_0}{R}\right)^2\sum_j \psi_j\,{\rm e}^{\,{\rm i}\,(m_j\,\theta-n\,\phi)}.
\end{align}
According to PRL {\bf 99}, 195003 (2007),
\begin{equation}
{\mit\Delta}_j\,{\rm e}^{\,{\rm i}\,(m_j\,\theta-n\,\phi)} =\left[\frac{\partial}{\partial\psi_p}\,\frac{\delta {\bf B}\cdot\nabla\psi_p}{{\bf B}\cdot\nabla\phi}\right]_{r_j}=
\left[\frac{\partial}{\partial r}\,\frac{\delta {\bf B}\cdot\nabla r}{{\bf B}\cdot\nabla\phi}\right]_{r_j} = \left[\frac{\partial}{\partial r}\,\frac{\delta B^{\,r}}{{\bf B}\cdot\nabla\phi}\right]_{r_j} .
\end{equation}
It follows that
\begin{equation}
{\mit\Delta}_j = \frac{{\rm i}}{r_j}\left(\frac{R_0}{R}\right)^2\frac{R^{\,2}}{B_0\,R_0\,g_j}\left[\frac{d\psi_j}{dr}\right]_{r_j},
\end{equation}
But,
\begin{equation}
\left[r\,\frac{d\psi_j}{dr}\right]_{r_j} = m_j\,a_{jj}\,[\chi_j]_{r_j} =  m_j\,a_{jj}\,{\mit\Delta\Psi}_j,
\end{equation}
and
\begin{equation}
\chi_j =\frac{{\mit\Delta\Psi}_j}{|E_{jj}|},
\end{equation}
which implies that
\begin{align}
{\mit\Delta}_j ={\rm i}\left(\frac{R_0}{r_j}\right)^2 \frac{m_j\,a_{jj}}{g_j}\,\frac{{\mit\Delta\Psi}_j}{B_0\,R_0},
\end{align}
or
\begin{align}
\frac{\chi_j}{R_0\,B_0} = -{\rm i}\left(\frac{r_j}{R_0}\right)^2 \frac{g_j}{m_j\,a_{jj}}\,\frac{{\mit\Delta}_j}{|E_{jj}|}.
\end{align}
Here, the ${\mit\Delta}_j$ values can be determined from the GPEC code. 

\subsubsection{PoP Derivation}
Now, $d\psi_p/dr = B_0\,R_0\,f$. It follows that
\begin{equation}
(\nabla\psi_p\times\nabla\theta\cdot\nabla\phi)^{-1}=\frac{R^{\,2}\,q}{B_0\,R_0\,g}.
\end{equation}
Hence,
\begin{equation}
{\bf B} = \nabla\phi\times\nabla\psi_p + B_0\,R_0\,g\,\nabla\phi = \nabla\phi\times\nabla\psi_p + q\,\nabla\psi_p\times\nabla\theta.
\end{equation}
Let ${\mit\Psi}_p = 2\pi\,\psi_p$ and $d{\mit\Psi}_p/d{\mit\Psi}_t = 1/q$. 
Hence,
\begin{align}
\frac{d{\mit\Psi}_p}{dr} &= 2\pi\,B_0\,R_0\,f,\\[0.5ex]
\frac{d{\mit\Psi}_t}{dr} &= 2\pi\,B_0\,r\,g,
\end{align}
and [cf.\ PoP {\bf 13}, 102501 (2006), Eq.~(41)]
\begin{equation}
2\pi\,{\bf B} = q^{-1}\,\nabla\phi\times\nabla{\mit\Psi}_t + \nabla{\mit\Psi}_t\times\nabla\theta.
\end{equation}

We have
\begin{align}
\delta J^{\,r} &= 0,\\[0.5ex]
{\cal J}\,\mu_0\,\delta J^{\,\theta} &= -\sum_j \frac{{\mit\Delta\Psi}_j}{q_j}\,{\rm e}^{\,{\rm i}\,(m_j\,\theta-n\,\phi)}\,\delta(r-r_j),\\[0.5ex]
{\cal J}\,\mu_0\,\delta J^{\,\phi} &= -\sum_j {\mit\Delta\Psi}_k\,{\rm e}^{\,{\rm i}\,(m_j\,\theta-n\,\phi)}\,\delta(r-r_j),
\end{align}
which implies that
\begin{equation}
\mu_0\,\delta{\bf J} = -2\pi\sum_j{\mit\Delta\Psi}_j,{\rm e}^{\,{\rm i}\,(m_j\,\theta-n\,\phi)}\,\delta(\psi_t-\psi_{t\,j})\,{\bf B}.
\end{equation}
However, according to PoP {\bf 13}, 102501 (2006),
\begin{equation}
{\mit\Delta\Psi}_j =- {\rm i}\,\frac{\mu_0\,J_c\,{\mit\Delta}_j}{2\pi\,m_j},
\end{equation}
where
\begin{equation}
\frac{1}{\mu_0\,J_c} = \left(\oint \frac{B^{\,2}}{|\nabla\psi_t|^{\,2}}\,\frac{d\theta\,d\phi}{2\pi\,{\bf B}\cdot\nabla\phi}\right)_{r_j}.
\end{equation}
It is easily demonstrated that
\begin{equation}
\frac{1}{\mu_0\,J_c} = \left(\frac{R_0}{r_j}\right)^2\,\frac{1}{2\pi\,B_0\,R_0\,g_j}\left[a_{jj} + \left(\frac{r_k}{R_0\,q_j}\right)^2\right].
\end{equation}
Hence,
\begin{equation}
\frac{{\mit\Delta\Psi}_j}{B_0\,R_0} = -{\rm i}\,{\mit\Delta}_j\left(\frac{r_j}{R_0}\right)^2\frac{g_j}{m_j\,[a_{jj}+ (r_k/R_0\,q_j)^2]},
\end{equation}
which implies that
\begin{equation}
\frac{\chi_j}{B_0\,R_0} = -{\rm i}\,\frac{{\mit\Delta}_j}{|E_{jj}|}\left(\frac{r_j}{R_0}\right)^2\frac{g_j}{m_j\,[a_{jj}+ (r_k/R_0\,q_j)^2]}.
\end{equation}
(Note: This is what is actually implemated in EPEC.)
As before, the ${\mit\Delta}_j$ values can be determined from the GPEC code. 

\subsection{Magnetic Island Width}
\subsubsection{Island Width in $r$}
We have
\begin{align}
{\cal J}\,\delta B^{\,r} &\simeq \frac{\partial \,\delta A_\phi}{\partial\theta},\\[0.5ex]
{\cal J}\,\delta B^{\,\theta} &\simeq -\frac{\partial\,\delta A_\phi}{\partial r},\\[0.5ex]
{\cal J}\,\delta B^{\,\phi} &\simeq 0.
\end{align}
It follows that
\begin{equation}
\delta {\bf B}\cdot\nabla\delta A_\phi = \delta B^{\,r}\,\frac{\partial\,\delta A_\phi}{\partial r}+  \delta B^{\,\theta}\,\frac{\partial\,\delta A_\phi}{\partial \theta}+  \delta B^{\,\phi}\,\frac{\partial\,\delta A_\phi}{\partial \phi}\simeq 0. 
\end{equation}
We have
\begin{equation}
\delta A_\phi (r,\theta,\phi)\simeq R_0\,{\mit\Psi}_k\,{\rm e}^{\,{\rm i}\,(
m_k\,\theta - n\,\phi)}
\end{equation}
in the vicinity of the $k$th resonant surface. It follows that
\begin{equation}
{\bf B}\cdot\nabla \delta A_\phi = {\rm i}\,B_0\,R_0^{\,2}\,\frac{f}{r\,R^{\,2}}\,(m_k-n\,q)\,R_0\,{\mit\Psi}_k\,{\rm e}^{\,{\rm i}\,(
m_k\,\theta - n\,\phi)}.
\end{equation}
Let us search for a function,
\begin{equation}
F(r,\theta,\phi)= F_0(r) + \delta A_\phi,
\end{equation}
which is such that
\begin{equation}
({\bf B}+\delta {\bf B})\cdot\nabla F = 0.
\end{equation}
It follows that
\begin{equation}
\delta B^{\,r}\,\frac{dF_0}{dr} +{\rm i}\,B_0\,R_0^{\,2}\,\frac{f}{r\,R^{\,2}}\,(m_k-n\,q)\,R_0\,{\mit\Psi}_k\,{\rm e}^{\,{\rm i}\,(m_k\,\theta - n\,\phi)}=0.
\end{equation}
However,
\begin{equation}
\delta B^{\,r} = \frac{R_0^{\,2}}{r\,R^{\,2}}\,{\rm i}\,m_k\,{\mit\Psi}_k\,{\rm e}^{\,{\rm i}\,(m_k\,\theta - n\,\phi)},
\end{equation}
so
\begin{equation}
{\rm i}\,m_k\,\frac{R_0}{r\,R^{\,2}}\,R_0\,{\mit\Psi}_k\,\frac{dF_0}{dr}+{\rm i}\,B_0\,R_0^{\,2}\,\frac{f}{r\,R^{\,2}}\,(m_k-n\,q)\,R_0\,{\mit\Psi}_k=0,
\end{equation}
which implies that
\begin{equation}
\frac{dF_0}{dr} = -\frac{B_0\,R_0\,f}{m_k}\,(m_k-n\,q),
\end{equation}
or
\begin{equation}
F_0(r) \simeq \frac{B_0}{2}\left(\frac{g\,s}{q}\right)_{r_k} (r-r_k)^{\,2},
\end{equation}
where $s=r\,q'/q$. 
Hence,
\begin{equation}
F(r,\theta,\phi) = \frac{B_0}{2}\left(\frac{g\,s}{q}\right)_{r_k}(r-r_k)^{\,2}
+R_0\,{\mit\Psi}_k\,\cos(m_k\,\theta-n\,\phi)
\end{equation}
is a flux surface function in the island region. 
Thus, 
\begin{equation}
\frac{F}{R_0\,|{\mit\Psi}_k|}= 2\,X^{\,2} + \cos(m_k\,\theta-n\,\phi),
\end{equation}
where
\begin{equation}
X = \frac{2\,(r-r_k)}{W_k},
\end{equation}
and
\begin{equation}
\frac{W_k}{4\,R_0} = \left[\left(\frac{q}{g\,s}\right)_{r_k}\,\frac{|{\mit\Psi}_k|}{B_0\,R_0}\right]^{1/2}.
\end{equation}
It follows that $W_k$ is the full radial island width  in $r$. Moreover, $W_k$ has no dependence on $\theta$.

\subsubsection{Island Width in ${\mit\Psi}_N$}
We have
\begin{align}
\frac{d{\mit\Psi}_N}{dr} &= \frac{f}{R_0\,|\psi_c|},\\[0.5ex]
\frac{dF_0}{dr} &= -\frac{B_0\,R_0}{m_k}\,f\,(m_k-n\,q).
\end{align}
Hence,
\begin{equation}
\frac{dF_0}{d{\mit\Psi}_N} = -B_0\,R_0^{\,2}\,|\psi_c|\,(1-q/q_k).
\end{equation}

Let
\begin{equation}
q = q_k + q_k'\,({\mit\Psi}_N-{\mit\Psi}_{N\,k})+ \frac{1}{2}\,q_k''\,({\mit\Psi}_N-{\mit\Psi}_{N\,k})^2+ \frac{1}{6}\,q_k'''\,({\mit\Psi}_N-{\mit\Psi}_{N\,k})^3,
\end{equation}
where $'\equiv d/d{\mit\Psi}_N$. Follows that
\begin{equation}
F_0({\mit\Psi}_N) = B_0\,R_0^{\,2}\,|\psi_c|\left\{\frac{1}{2}\,\frac{q_k'}{q_k}({\mit\Psi}_N-{\mit\Psi}_{N\,k})^2+  \frac{1}{6}\,\frac{q_k''}{q_k}\,({\mit\Psi}_N-{\mit\Psi}_{N\,k})^3+ \frac{1}{24}\,
\frac{q_k'''}{q_k}\,({\mit\Psi}_N-{\mit\Psi}_{N\,k})^4\right\}.
\end{equation}
Hence, the island extends from  ${\mit\Psi}_{N\,k}+X_{k-}$ to ${\mit\Psi}_{N\,k}+X_{k+}$, where $X_{k-}$ and $X_{k+}$ are the
positive and negative roots, respectively, of 
\begin{equation}
X_k^{\,2}+  A_k^{(2)}\,X_k^{\,3}= \frac{\overline{W}_k^{\,2}}{4},
\end{equation}
where
\begin{equation}
\overline{W}_k=4\left(A_k^{(1)}\,\frac{|{\mit\Psi}_k|}{R_0\,B_0}\right)^{1/2},
\end{equation}
and
\begin{align}
A_k^{(1)} &= \frac{q_k}{q_k'\,|\psi_c|},\\[0.5ex]
A_k^{(2)}& =  \frac{1}{3}\,\frac{q_k''}{q_k'}.
\end{align}
It follows that
\begin{equation}
X_{k\pm} \simeq \pm \frac{\overline{W}_k}{2} - A_k^{(2)}\,\frac{\overline{W}_k^{\,2}}{8}.
\end{equation}

Alternatively, can say that island extends from  ${\mit\Psi}_{N\,k}+X_{k-} + {\mit\Delta\Psi}_{N\,k}$ to ${\mit\Psi}_{N\,k}+X_{k+}+{\mit\Delta\Psi}_{N\,k}$,
where $X_{k\pm} = \pm{\mit\Delta}_{k\pm}\,x_{k\pm}$,
and
\begin{align}
{\mit\Delta}_{k+} &= \chi_{\rm max}({\mit\Psi}_{N\,k+1}-{\mit\Psi}_{N\,k}),\\[0.5ex]
{\mit\Delta}_{k-} &=\chi_{\rm max}( {\mit\Psi}_{N\,k}-{\mit\Psi}_{N\,k-1}),\\[0.5ex]
F_{k\pm}(x_{k\pm}) &= \frac{2\,A_k^{(1)}}{{\mit\Delta}_{k\pm}^{\,2}}\,\frac{|{\mit\Psi}_k|}{R_0\,B_0},\\[0.5ex]
F_{k\pm}(x)&\equiv - x -\ln(1-x).
\end{align}
Here,  $\chi_{\rm max}$ is the maximum allowable Chirikov parameter.
Moreover,
\begin{equation}
{\mit\Delta\Psi}_{N\,k } = -\frac{A_k^{(2)}}{8}\,(X_{k+}-X_{k-})^{\,2}.
\end{equation}

\section{Technical Details}
\subsection{Flux Coordinate System}
Let all lengths be normalized to $R_0$, and all magnetic field-strengths to $B_0$. 
We have
\begin{equation}
{\bf B} =\nabla\phi\times \nabla\psi_p +g(\psi_p)\,\nabla\phi,
\end{equation}
and
\begin{equation}
\nabla\psi_p\times \nabla\theta\cdot\nabla\phi = \frac{g}{q\,R^{\,2}},
\end{equation}
where $q=q(\psi_p)$. 

Let ${\mit\Psi}=\psi_p/\psi_c = 1-{\mit\Psi}_N$, where $\psi_c$ is the value
of $\psi_p$ on the magnetic axis. (It is assumed that $\psi_p=0$ on the plasma boundary.) 
The previous equation implies that
\begin{equation}
\frac{d\theta}{dl} = \frac{g}{q}\,\frac{1}{|\psi_c|\,R\,\sqrt{{\mit\Psi}_R^{\,2}+{\mit\Psi}_Z^{\,2}}},
\end{equation}
where $dl$ is an element of poloidal path length on a magnetic flux-surface, and ${\mit\Psi}_R\equiv \partial{\mit\Psi}/\partial R$, etc. Furthermore,
\begin{align}
dR &= -\frac{{\mit\Psi}_Z\,dl}{\sqrt{{\mit\Psi}_R^{\,2}+{\mit\Psi}_Z^{\,2}}},\\[0.5ex]
dZ&= \frac{{\mit\Psi}_R\,dl}{\sqrt{{\mit\Psi}_R^{\,2}+{\mit\Psi}_Z^{\,2}}}.
\end{align}
It follows that
\begin{equation}
\frac{q({\mit\Psi})}{g({\mit\Psi})} = \frac{1}{2\pi\,|\psi_c|}\oint \frac{dl}{R\,\sqrt{{\mit\Psi}_R^{\,2}+{\mit\Psi}_Z^{\,2}}}.
\end{equation}

If we define
\begin{equation}
\tan\zeta = \frac{Z-Z_{\rm axis}}{R_{\rm axis}-R}
\end{equation}
then
\begin{align}
\frac{dR}{d\zeta} &= -{\mit\Psi}_Z\,F,\\[0.5ex]
\frac{dZ}{d\zeta} &= {\mit\Psi}_R\,F,\\[0.5ex]
\frac{q({\mit\Psi})}{g({\mit\Psi})} &= \frac{1}{2\pi\,|\psi_c|}\oint \frac{F}{R}\,d\zeta,
\end{align}
where
\begin{equation}
F = \frac{(R_{\rm axis}-R)^{\,2} + (Z-Z_{\rm axis})^{\,2}}{-(Z-Z_{\rm axis})\,{\mit\Psi}_Z + (R_{\rm axis}-R)\,{\mit\Psi}_R}.
\end{equation}

It is helpful to define the length-like flux-surface coordinate $r$, according to
\begin{equation}
\nabla r\times \nabla\theta\cdot\nabla \phi = \frac{1}{r\,R^{\,2}}.
\end{equation}
It follows that
\begin{equation}
r({\mit\Psi}) = \left[2\,|\psi_c|\int_{{\mit\Psi}}^1\frac{q({\mit\Psi}')}
{g({\mit\Psi}')}\,d{\mit\Psi}'\right]^{1/2}.
\end{equation}
Let
\begin{equation}
a = r (0).
\end{equation}

We can calculate $R(r,\theta)$ and $Z(r,\theta)$ from 
\begin{align}
\frac{dR}{d\theta}&=-|\psi_c|\,\frac{q}{g}\,R\,{\mit\Psi}_Z,\\[0.5ex]
\frac{dZ}{d\theta} &= |\psi_c|\,\frac{q}{g}\,R\,{\mit\Psi}_R.
\end{align}

Now,
\begin{equation}
r\,\frac{dr}{d{\mit\Psi}} = -|\psi_c|\,\frac{q(r)}{g(r)}.
\end{equation}
So
\begin{equation}
\nabla r = \frac{dr}{d{\mit\Psi}}\,\nabla{\mit\Psi} = -|\psi_c|\,\frac{q(r)}{r\,g(r)}\,\nabla{\mit\Psi}.
\end{equation}
Hence,
\begin{equation}
a_{jj}  =\left(\oint \frac{1}{|\nabla r|^{\,2}}\,\frac{d\theta}{2\pi}\right)_{r_j} = \left(\frac{r\,g}{|\psi_c|\,q}\right)^2_{r_j}\oint
\frac{1}{{\mit\Psi}_R^{\,2}+{\mit\Psi}_Z^{\,2}}\,\frac{d\theta}{2\pi}.
\end{equation}

Note that
\begin{equation}
\frac{d{\mit\Psi}_N}{dr} =\frac{r\,g(r)}{|\psi_c|\,q(r)}.
\end{equation}
Hence, if $\overline{W}_k$ is the full magnetic island width in ${\mit\Psi}_N$ at the 
$k$th resonant surface then
\begin{equation}
\overline{W}_k = \frac{W_k}{R_0}\,\frac{d{\mit\Psi}_N(r_k)}{dr},
\end{equation}
where $W_k$ is the full island width in $r$. 

\subsection{Neoclassical Coordinate System}
It is also helpful to define the geometric poloidal angle
\begin{equation}
{\bf b}\cdot\nabla {\mit\Theta} = \gamma(r).
\end{equation}
It follows that
\begin{equation}
\frac{d{\mit\Theta}}{dl} = \frac{\gamma\,B\,R}{|\psi_c|\,\sqrt{{\mit\Psi}_R^{\,2}+{\mit\Psi}_Z^{\,2}}},
\end{equation}
where
\begin{equation}
B\,R = \left[g^{\,2} + |\psi_c|^{\,2}\,({\mit\Psi}_R^{\,2}+{\mit\Psi}_Z^{\,2})\right]^{1/2}.
\end{equation}
Hence,
\begin{equation}
\frac{1}{\gamma(r)} = \frac{1}{2\pi\,|\psi_c|}\oint\frac{B\,R\,dl}{\sqrt{{\mit\Psi}_R^{\,2}+{\mit\Psi}_Z^{\,2}}}= \frac{1}{2\pi\,|\psi_c|}\oint B\,R\,F\,d\zeta.
\end{equation}
We can calculate $R(r,{\mit\Theta})$ and $Z(r,{\mit\Theta})$ from 
\begin{align}
\frac{dR}{d{\mit\Theta}} &=-|\psi_c|\,\frac{{\mit\Psi}_Z}{\gamma\,B\,R},\\[0.5ex]
\frac{dZ}{d{\mit\Theta}} &=|\psi_c|\,\frac{{\mit\Psi}_R}{\gamma\,B\,R}.
\end{align}
Note that
\begin{equation}
\frac{d{\mit\Theta}}{d\theta} = \left(\frac{\gamma\,q}{g}\right)B\,R^{\,2}.
\end{equation}
Thus,
\begin{equation}
\frac{1}{\gamma} = \frac{q}{g}\oint B\,R^{\,2}\,\frac{d\theta}{2\pi}.
\end{equation}

\subsection{Neoclassical Parameters}
The flux-surface average operator has the following
properties:
\begin{align}
\langle 1\rangle &= 1,\\[0.5ex]
\langle {\bf B}\cdot\nabla A\rangle &= 0.
\end{align}
It follows that
\begin{equation}\label{e140}
\langle A\rangle =\left.\oint R^{\,2}\,A\,\frac{d\theta}{2\pi}\right/\oint R^{\,2}\,\frac{d\theta}{2\pi}=\left.\oint
\frac{A}{B}\,\frac{d{\mit\Theta}}{2\pi}\right/\oint
\frac{1}{B}\,\frac{d{\mit\Theta}}{2\pi}.
\end{equation}

Let
\begin{align}
I_0 &= \oint \frac{1}{B\,R^{\,2}}\,\frac{d{\mit\Theta}}{2\pi}=\frac{\gamma\,q}{g},\\[0.5ex]
I_1 &=\oint \frac{1}{B}\,\frac{d{\mit\Theta}}{2\pi},\\[0.5ex]
I_2 &=\oint B\,\frac{d{\mit\Theta}}{2\pi},\\[0.5ex]
I_3 &=\oint\left(\frac{\partial B}{\partial{\mit\Theta}}\right)^2\frac{1}{B}\,\frac{d{\mit\Theta}}{2\pi},\\[0.5ex]
I_{4,k}&= \sqrt{\frac{2}{k}}\oint \frac{\sin(k\,{\mit\Theta})}{B^{\,2}}\,\frac{\partial B}{\partial{\mit\Theta}}\,\frac{d{\mit\Theta}}{2\pi}=\oint\frac{\sqrt{2\,k}\,\cos(k\,{\mit\Theta})}{B}\,\frac{d{\mit\Theta}}{2\pi},\\[0.5ex]
I_{5,k} &= \sqrt{\frac{2}{k}}\oint \frac{\sin(k\,{\mit\Theta})}{B^{\,3}}\,\frac{\partial B}{\partial{\mit\Theta}}\,\frac{d{\mit\Theta}}{2\pi}=\oint\frac{\sqrt{2\,k}\,\cos(k\,{\mit\Theta})}{2\,B^{\,2}}\,\frac{d{\mit\Theta}}{2\pi},\\[0.5ex]
I_6(\lambda) &=\oint \frac{\sqrt{1-\lambda\,B/B_{\rm max}}}{B}\,\frac{d{\mit\Theta}}{2\pi},\\[0.5ex]
I_7 &= \oint\frac{R^{\,2}}{B}\,\frac{d{\mit\Theta}}{2\pi},\\[0.5ex]
I_8 &= \oint \frac{1}{B^{\,3}\,R^{\,2}}\,\frac{d{\mit\Theta}}{2\pi}.
\end{align}
It follows that
\begin{align}
\langle B\rangle &= \frac{1}{I_1},\\[0.5ex]
C_1=\left\langle \frac{1}{R^{\,2}}\right\rangle&= \frac{\gamma\,q}{I_1\,g},\\[0.5ex]\langle R^{\,2}\rangle &= \frac{I_7}{I_1},\\[0.5ex]
\langle B^{\,2}\rangle &= \frac{I_2}{I_1},\\[0.5ex]
C_2 = g^{\,2}\left\langle \frac{1}{B^{\,2}\,R^{\,2}}\right\rangle&= \frac {g^{\,2}\,I_8}{I_1},\\[0.5ex]
\left\langle \frac{|\nabla r|^{\,2}}{R^{\,2}}\right\rangle &= \frac{\gamma\,q\,a_{jj}}{I_1\,g},\\[0.5ex]
\langle ({\bf b}\cdot\nabla B)^{\,2}\rangle&= \gamma^{\,2}\,\frac{I_3}{I_1},\\[0.5ex]
|\langle {\bf B}\cdot \nabla \theta\rangle|&= \frac{g}{|q|}\,\frac{I_0}{I_1} = \frac{|\gamma|}{I_1},
\\[0.5ex]
\left\langle \sqrt{\frac{2}{k}}\,\sin(k\,{\mit\Theta}) \,({\bf b}\cdot\nabla \ln B)\right\rangle &= \gamma\,\frac{I_{4,k}}{I_1},\\[0.5ex]
\left\langle \sqrt{\frac{2}{k}}\,\sin(k\,{\mit\Theta}) \,\frac{({\bf b}\cdot\nabla \ln B)}{B}\right\rangle &= \gamma\,\frac{I_{5,k}}{I_1}.
\end{align}
Hence,
\begin{align}
L_c &= \frac{1}{|\gamma|}\,\frac{I_2^{\,2}}{I_1^{\,2}\,I_3}\sum_{k>0}
I_{4,k}\,I_{5,k},\\[0.5ex]
\omega_{t\,a}&\equiv \frac{v_{T\,a}}{L_c} = K_{t}\,|\gamma|\,v_{T\,a},\\[0.5ex]
\nu_{\ast\,a} &\equiv \frac{8}{3\pi}\,\frac{\langle B^{\,2}\rangle}{\langle
({\bf b}\cdot\nabla B)^{\,2}\rangle}\,\frac{g_t\,\omega_{t\,a}}{v_{T\,a}^{\,2}\,\tau_{aa}}= K_{\ast}\,\frac{g_t}{\omega_{t\,a}\,\tau_{aa}},\\[0.5ex]
f_c &=\frac{3}{4} \frac{I_2}{B_{\rm max}^{\,2}}\int_0^{1}\frac{\lambda\,d\lambda}{I_6(\lambda)},
\end{align}
where 
\begin{align}
K_{t} &= \frac{I_1^{\,2}\,I_3}{I_2^{\,2}\,\sum_{k>0}I_{4,k}\,I_{5,k}},\\[0.5ex]
K_{\ast}&= \frac{8}{3\pi}\,\frac{I_2}{I_3}\,K_{t}^{\,2}.
\end{align}
Also,
\begin{align}
\hat{q}^{\,2} &= \frac{q^{\,2}}{2\,r^{\,2}}\left.\left(\left\langle \frac{1}{R^{\,2}}\right\rangle - \frac{1}{\langle R^{\,2}\rangle}\right)\right/\left\langle\frac{|\nabla r|^{\,2}}{R^{\,2}}\right\rangle = \frac{q^{\,2}}{2\,r^{\,2}\,a_{jj}}\left(
1-\frac{I_1^{\,2}}{I_7}\,\frac{g}{\gamma\,q}\right),\\[0.5ex]
\frac{\langle B_T^{\,2}\rangle}{\langle B_p^{\,2}\rangle}&= \frac{q^{\,2}}{r^{\,2}\,a_{jj}}.
\end{align}

\end{document}