\documentclass[notitlepage,12pt]{article}
\pdfoutput = 1
\usepackage{amsmath}
\usepackage{fullpage}
\usepackage{graphicx}
\allowdisplaybreaks

\title{\bf Cylindrical Tearing Mode Analysis}
\date{\today}
\author{Richard Fitzpatrick}

\begin{document}
\maketitle

\section{Normalization}
All lengths are normalized to $R_0$. All magnetic field-strengths are normalized to $B_0$. All current densities are
normalized to $B_0/(R_0\,\mu_0)$. 

\section{Coordinates}
Adopt cylindrical coordinates $r$, $\theta$, $z$. 

\section{Plasma Equilibrium}
The equilibrium magnetic field is written
\begin{equation}
{\bf B} = \nabla{\mit\Psi}(r)\times{\bf e}_z + {\bf e}_z.
\end{equation}
So, 
\begin{align}
B_r &= 0,\\[0.5ex]
B_\theta &= -\frac{d{\mit\Psi}}{dr},\\[0.5ex]
B_z &= 1.
\end{align}
The safety-factor profile is
\begin{equation}
q(r) = \frac{r\,B_z}{B_\theta}= \frac{r}{B_\theta}.
\end{equation}
The equilibrium current density is written
\begin{equation}
{\bf J} = J_z(r)\,{\bf e}_z,
\end{equation}
where
\begin{equation}\label{e7}
J_z = \frac{1}{r}\,\frac{d(r\,B_\theta)}{dr} =\frac{1}{r}\,\frac{d}{dr}\!\left(\frac{r^{\,2}}{q}\right)=\frac{2-s}{q},
\end{equation}
and
\begin{equation}
s(r) = \frac{d\ln q}{dr}.
\end{equation}

\section{Cylindrical Tearing Mode Equation}
Let 
\begin{align}
{\bf b} &= \nabla\left[\psi(r)\,{\rm e}^{\,{\rm i}\,(m\,\theta-n\,z-\omega\,t)}\right]\times{\bf e}_z\,\\[0.5ex]
{\bf j} &\simeq -\left[\frac{1}{r}\,\frac{d}{dr}\!\left(r\,\frac{d\psi}{dr}\right)-\frac{m^{\,2}}{r^{\,2}}\,\psi\right]{\rm e}^{\,{\rm i}\,(m\,\theta-n\,z-\omega\,t)}\,{\bf e}_z,
\end{align}
be the perturbed magnetic field and current density, respectively. Here, we have assumed that  $r\ll 1$. Linearized  force balance
yields
\begin{equation}
({\bf J}\cdot\nabla)\,{\bf b} +({\bf j}\cdot\nabla)\,{\bf B} -({\bf B}\cdot\nabla)\,{\bf j} - ({\bf b}\cdot\nabla)\,{\bf J} = 0.
\end{equation}
The $z$ component of the previous equation gives the cylindrical tearing mode equation:
\begin{equation}
\frac{d^{\,2}\psi}{dr^{\,2}}+\frac{1}{r}\,\frac{d\psi}{dr} - \frac{m^{\,2}}{r^{\,2}}\,\psi - \frac{J_z'\,\psi}{r\,(1/q-n/m)} = 0,
\end{equation}
where $'\equiv d/dr$. In the following, $q(r)$ and $J_z(r)$ are taken from the EQDSK equilibrium, rather than being related
according to Eq.~(\ref{e7}). 

\section{Behavior in Vicinity of Rational Surface}
Suppose that $q(r_s)=m/n$. Let $\rho = r-r_s$. Expanding the previous equation about $r=r_s$, we obtain
\begin{equation}
\frac{d^{\,2}\psi}{d\rho^{\,2}} + \frac{1}{r_s}\left(1-\frac{\rho}{r_s}\right)\frac{d\psi}{d\rho} - \frac{m^{\,2}}{r_s^{\,2}}\left(1-\frac{2\,\rho}{r_s}\right)\psi - 
\left(\frac{\alpha}{\rho} + \beta\right)\psi = 0,
\end{equation}
where
\begin{align}
\alpha &= -\left(\frac{q\,J_z'}{s}\right)_{r=r_s},\\[0.5ex]
\beta &= \alpha\left(\frac{J_z''}{J_z'} +\frac{s-1}{r_s}-\frac{q''}{2\,q'}\right)_{r=r_s}.
\end{align}
We obtain
\begin{align}
\psi(\rho) &= {\mit\Psi}\left[1+\left(\frac{{\mit\Sigma}'\pm{\mit\Delta}'}{2}\right)\rho\right.\nonumber\\[0.5ex]
&\left.\phantom{===}+\left\{\alpha\left(\frac{{\mit\Sigma}'\pm{\mit\Delta}'}{4}\right)\left(1-\frac{1}{\alpha\,r_s}\right)+\frac{1}{2}\left[
\frac{m^{\,2}}{r_s^{\,2}} +\beta - \frac{\alpha^{\,2}}{2}\left(3-\frac{1}{\alpha\,r_s}\right)\right]\right\}\rho^{\,2}\right.\nonumber\\[0.5ex]
&\left.\phantom{===}+\alpha\left\{\rho + \frac{\alpha}{2}\left(1-\frac{1}{\alpha\,r_s}\right)\rho^{\,2}\right\}\ln|\rho|+{\cal O}(\rho^{\,3})
\right].
\end{align}
Here, the plus sign corresponds to $\rho>0$, whereas the minus sign corresponds to $\rho<0$.
It follows that
\begin{align}
\psi'(\rho) &= {\mit\Psi}\left[\frac{{\mit\Sigma}'\pm{\mit\Delta}'}{2}+\alpha\right.\nonumber\\[0.5ex]
&\left.\phantom{===}+\left\{\alpha\left(\frac{{\mit\Sigma}'\pm{\mit\Delta}'}{2}\right)\left(1-\frac{1}{\alpha\,r_s}\right)+
\frac{m^{\,2}}{r_s^{\,2}} +\beta - \frac{\alpha^{\,2}}{2}\left(3-\frac{1}{\alpha\,r_s}\right)
+\frac{\alpha^{\,2}}{2}\left(1-\frac{1}{\alpha\,r_s}\right)\right\}\rho\right.\nonumber\\[0.5ex]
&\left.\phantom{===}+\alpha\,\ln|\rho| + \alpha^{\,2}\left(1-\frac{1}{\alpha\,r_s}\right)\rho\,\ln|\rho|+{\cal O}(\rho^{\,2})
\right].
\end{align}
Now,
\begin{equation}
\lambda_+\equiv \left.\frac{\psi'}{\psi}\right|_{\rho =\delta} = \frac{a_{1+}+b_{1+}\,({\mit\Sigma}'+{\mit\Delta'})}{a_{2+}+ b_{2+}\,({\mit\Sigma}'+{\mit\Delta'})},
\end{equation}
where
\begin{align}
a_{1+}&= \alpha\,(1+\ln\delta)+\left\{
\frac{m^{\,2}}{r_s^{\,2}} +\beta - \frac{\alpha^{\,2}}{2}\left(3-\frac{1}{\alpha\,r_s}\right)
+\alpha^{\,2}\left(1-\frac{1}{\alpha\,r_s}\right)\left(\frac{1}{2}+\ln\delta\right)\right\}\delta,\\[0.5ex]
b_{1+} &=\frac{1}{2}+ \frac{\alpha}{2}\left(2-\frac{1}{\alpha\,r_s}\right)\delta,\\[0.5ex]
a_{2+} &= 1+\alpha\,\delta\,\ln\delta,\\[0.5ex]
b_{2+} &= \frac{\delta}{2}.
\end{align}
It follows that
\begin{equation}
{\mit\Sigma}'+ {\mit\Delta}' = \frac{a_{2+}\,\lambda_+ - a_{1+}}{b_{1+}-b_{2+}\,\lambda_+}.
\end{equation}
Likewise,
\begin{equation}
\lambda_-\equiv \left.\frac{\psi'}{\psi}\right|_{\rho =-\delta} = \frac{a_{1-}+b_{1-}\,({\mit\Sigma}'-{\mit\Delta'})}{a_{2-}+ b_{2-}\,({\mit\Sigma}'-{\mit\Delta'})},
\end{equation}
where
\begin{align}
a_{1-}&= \alpha\,(1+\ln\delta)-\left\{
\frac{m^{\,2}}{r_s^{\,2}} +\beta - \frac{\alpha^{\,2}}{2}\left(3-\frac{1}{\alpha\,r_s}\right)
+\alpha^{\,2}\left(1-\frac{1}{\alpha\,r_s}\right)\left(\frac{1}{2}+\ln\delta\right)\right\}\delta,\\[0.5ex]
b_{1-} &=\frac{1}{2}- \frac{\alpha}{2}\left(2-\frac{1}{\alpha\,r_s}\right)\delta,\\[0.5ex]
a_{2-} &= 1-\alpha\,\delta\,\ln\delta,\\[0.5ex]
b_{2-} &= -\frac{\delta}{2}.
\end{align}
It follows that
\begin{equation}
{\mit\Sigma}'- {\mit\Delta}' = \frac{a_{2-}\,\lambda_- - a_{1-}}{b_{1-}-b_{2-}\,\lambda_-}.
\end{equation}

Hence, 
\begin{align}
{\mit\Sigma}' &= \frac{1}{2}\left(\frac{a_{2+}\,\lambda_+ - a_{1+}}{b_{1+}-b_{2+}\,\lambda_+}+\frac{a_{2-}\,\lambda_- - a_{1-}}{b_{1-}-b_{2-}\,\lambda_-}\right),\\[0.5ex]
{\mit\Delta}'&=\frac{1}{2}\left(\frac{a_{2+}\,\lambda_+ - a_{1+}}{b_{1+}-b_{2+}\,\lambda_+}-\frac{a_{2-}\,\lambda_- - a_{1-}}{b_{1-}-b_{2-}\,\lambda_-}\right).
\end{align}

\section{Solution Launched from Magnetic Axis}
Let
\begin{equation}
\psi(r) = r^{\,m}\,f(r).
\end{equation}
It follows that
\begin{equation}
\psi'(r) = m\,r^{\,m-1}\,f + r^{\,m}\,f'.
\end{equation}
The cylindrical tearing mode equation transforms into
\begin{equation}
f'' + (1+2\,m)\,\frac{f'}{r} - \frac{J_z'\,f}{r\,(1/q-n/m)} = 0.
\end{equation}
The well-behaved solution launched from the magnetic axis is such that
\begin{equation}
f(r) = 1 +\left[\frac{c}{4\,(1+m)}\right]r^{\,2},
\end{equation}
where 
\begin{equation}
c = \left(\frac{J_z''}{1/q-n/m}\right)_{r=0}.
\end{equation}

\section{Solution Launched from Plasma Boundary}
Suppose that the plasma boundary lies at $r=a$. 
Let
\begin{equation}
\psi(r) = r^{-m}\,g(r).
\end{equation}
It follows that
\begin{equation}
\psi'(r) = -m\,r^{-m-1}\,g+ r^{-m}\,g'.
\end{equation}
The cylindrical tearing mode equation transforms into
\begin{equation}
g'' + (1-2\,m)\,\frac{g'}{r} - \frac{J_z'\,g}{r\,(1/q-n/m)} = 0.
\end{equation}
In the absence of a wall, or external currents flowing in non-axisymmetric field-coils, the well-behaved solution launched from the plasma boundary is such that
\begin{align}
g(a) &= 1,\\[0.5ex]
g'(a) &= 0.
\end{align}

\section{Resistive Wall}
Suppose that a thin resistive wall of radius $r_w>a$, uniform thickness $\delta_w$, and uniform resistivity $\eta_w$, is located outside the
plasma. We can now launch two different solutions from the plasma boundary.
The {\em no wall}\/ solution is such that
\begin{align}
g_{nw}(a) &= 1,\\[0.5ex]
g'_{nw}(a) &= 0.
\end{align}
Let the associated ${\mit\Sigma}'$ and ${\mit\Delta}'$ values be denoted ${\mit\Sigma}'_{nw}$ and ${\mit\Delta}'_{nw}$,
respectively. 
The {\em perfect wall}\/ solution is such that
\begin{align}
g_{pw}(a) &= 1,\\[0.5ex]
g'_{pw}(a) &= -\frac{2\,m/a}{(r_w/a)^{\,2m}-1}.
\end{align}
Let the associated ${\mit\Sigma}'$ and ${\mit\Delta}'$ values be denoted ${\mit\Sigma}'_{pw}$ and ${\mit\Delta}'_{pw}$,
respectively. 

Asymptotic matching in the presence of the wall yields
\begin{align}
{\mit\Delta\Psi}_s&= E_{ss}\,{\mit\Psi}_s + E_{sw}\,{\mit\Psi}_w,\\[0.5ex]
{\mit\Delta\Psi}_w &= E_{ww}\,{\mit\Psi}_w + E_{sw}\,{\mit\Psi}_s.
\end{align}
Here, ${\mit\Psi}_s\equiv \psi(r_s)$, ${\mit\Psi}_w\equiv \psi(r_w)$, and 
\begin{align}
{\mit\Delta\Psi}_s&\equiv\left[r\,\frac{d\psi}{dr}\right]_{r_s-}^{r_s+},\\[0.5ex]
{\mit\Delta\Psi}_w&\equiv\left[r\,\frac{d\psi}{dr}\right]_{r_w-}^{r_w+}.
\end{align}
 Moreover, 
$E_{ss}$, $E_{sw}$, and $E_{ww}$ are real parameters. We can
make the identifications
\begin{align}
\hat{\mit\Delta}'_{pw} &= \left.\frac{{\mit\Delta\Psi}_s}{{\mit\Psi}_s}\right|_{{\mit\Psi}_w=0}=E_{ss},\\[0.5ex]
\hat{\mit\Delta}'_{nw} &= \left.\frac{{\mit\Delta\Psi}_s}{{\mit\Psi}_s}\right|_{{\mit\Delta\Psi}_w=0}=E_{ss}- \frac{E_{sw}^{\,2}}{E_{ww}},\\[0.5ex]
E_{sw} &= \frac{2m\,(r_s/r_w)^m}{[1-(a/r_w)^{2m}]\,g_{pw}(r_s)},
\end{align}
where $\hat{\mit\Delta}'_{pw}\equiv r_s\,{\mit\Delta}'_{pw}$, et cetera. 
Hence,
\begin{align}
E_{ww} &= -\frac{E_{sw}^{\,2}}{\hat{\mit\Delta}'_{nw} -\hat{\mit\Delta}'_{pw}}.
\end{align}

Standard analysis reveals that 
\begin{align}
{\mit\Delta\Psi}_w &= - {\rm i}\,\omega\,\tau_w\,{\mit\Psi}_w,\\[0.5ex]
\int_{r_w-}^{r_w+} j_z\,dr& = \frac{{\rm i}\,\omega\,\tau_w\,{\mit\Psi}_w\,{\rm e}^{\,{\rm i}\,(m\,\theta-n\,z-\omega\,t)}}{\mu_0\,r_w},
\end{align}
where
\begin{equation}
\tau_w = \frac{\mu_0\,\delta_w\,r_w}{\eta_w}.
\end{equation}

Let
\begin{equation}
\hat{\mit\Delta}'(\omega)= {\rm Re}\left(\frac{{\mit\Delta\Psi}_s}{{\mit\Psi}_s}\right).
\end{equation}
It follows that
\begin{equation}
\hat{\mit\Delta}'(\omega) = \hat{\mit\Delta}'_{nw} - (\hat{\mit\Delta}'_{nw}-\hat{\mit\Delta}'_{pw})\left(\frac{\omega^2}{\omega^{\,2} + \omega_w^{\,2}}\right),
\end{equation}
where
\begin{equation}
\omega_w = -\frac{E_{ww}}{\tau_w}.
\end{equation}
The net poloidal torque acting on the rational surface is minus that acting on the wall. Hence,
\begin{equation}
T_\theta = -\frac{1}{2}\,r_w\,\int_{r_w-}^{r_w+}\oint\int_{0}^{2\pi\,R_0} j_z\, b_r^{\ast}\,r\,dr\,d\theta\,dz.
\end{equation}
It follows that
\begin{equation}
T_\theta =- \frac{2\pi^{\,2}\,R_0}{\mu_0}\,m\,({\mit\Delta}'_{nw}-{\mit\Delta}'_{pw})\left(\frac{\omega\,\omega_w}{\omega^{\,2}+
\omega_w^{\,2}}\right)|{\mit\Psi}_s|^{\,2}.
\end{equation}
\end{document}