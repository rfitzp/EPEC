\documentclass[notitlepage,12pt]{article}
\usepackage {amsmath}
\pdfoutput = 1 
\usepackage {graphicx}
\usepackage{fullpage}
\allowdisplaybreaks

\begin{document}
\title{\bf The EPEC Model}
\author{R.~Fitzpatrick\,\footnote{rfitzp@utexas.edu}\\
Institute for Fusion Studies,\\  Department of Physics,\\  University of Texas at Austin,\\  Austin TX 78712, USA}
\maketitle

\section{Plasma Response in Outer Region}
\subsection{Coordinates}\label{a1}
Let $R$, $\phi$, and $Z$ be right-handed cylindrical coordinates whose symmetry axis corresponds to
the toroidal symmetry axis of the plasma. Let $r$, $\theta$, and $\phi$ be right-handed flux coordinates
whose Jacobian is ${\cal J}\equiv (\nabla r \times \nabla\theta\cdot\nabla\phi)^{-1}=r\,R^{\,2}/R_0$. 
Here, $R_0$ is a convenient scale major radius, $r$ is a magnetic flux-surface label with dimensions of length,
and $\theta$ is an axisymmetric angular coordinate that increases by $2\pi$ radians for every poloidal circuit of the magnetic axis.
Let $r=0$ correspond to the magnetic axis, and let $r=r_{100}$ correspond to the LCFS. 
Let $\theta=0$ correspond to the inboard midplane, and let $0<\theta<\pi$ correspond to the region above the midplane.

\subsection{Equilibrium Magnetic Field}\label{a2}
The equilibrium magnetic field is written
${\bf B} = R_0\,B_0\left[f(r)\,\nabla\phi\times \nabla r + g(r)\,\nabla\phi\right]$, 
where $B_0$ is a convenient scale toroidal magnetic field-strength, and
$q(r) = r\,g/(R_0\,f)$
is the safety-factor profile. The equilibrium poloidal magnetic flux (divided by $2\pi$), ${\mit\Psi}_p(r)$,
satisfies $d{\mit\Psi}_p/dr = R_0\,B_0\,f(r)$, where, by convention, ${\mit\Psi}_p(r_{100})=0$. The
normalized poloidal magnetic flux, ${\mit\Psi}_N(r)$, is defined such that
${\mit\Psi}_N(r) = 1-{\mit\Psi}_p(r)/{\mit\Psi}_p(0)$. Hence, ${\mit\Psi}_N(0)=0$ and
${\mit\Psi}_N(r_{100})=1$. Finally, if ${\mit\Psi}_N(r_{95})=0.95$ then $q_{95}\equiv q(r_{95})$. 

\subsection{Perturbed Magnetic Field}
Consider the response of the plasma to an RMP with $n>0$ periods in the toroidal direction.
We can write the components of the perturbed magnetic field in the form
\begin{align}
\frac{r\,R^{\,2}\,\delta{\bf B}\cdot\nabla r}{R_0^{\,2}}&= {\rm i}\sum_j \psi_j(r)\,{\rm e}^{\,{\rm i}\,(m_j\theta-n\,\phi)},\\[0.5ex]
{\cal J} \,\delta{\bf B}\cdot\nabla\phi\times \nabla r &= -\sum_j{\mit\Xi}_j(r)\,{\rm e}^{\,{\rm i}\,(m_j\theta-n\,\phi)},\\[0.5ex]
R^{\,2}\,\delta{\bf B}\cdot\nabla\phi&= n\sum_j \frac{{\mit\Xi}_j(r)}{m_j}\,{\rm e}^{\,{\rm i}\,(m_j\,\theta-n\,\phi)},
\end{align}
where the sum is over all relevant poloidal harmonics of the perturbed magnetic field. 

Let there be $K$ resonant magnetic flux-surfaces in the plasma, labelled $1$ through $K$.
Consider the $k$th resonant surface, $r=r_k$, at which $n\,q(r_k)= m_k$, where $m_k$ is a positive integer. Let
${\mit\Psi}_k = \psi_k(r_k)/m_k$, and 
${\mit\Delta\Psi}_k=[{\mit\Xi}_k]_{r_{k-}}^{r_{k+}}$.
Here, ${\mit\Psi}_k$ is the (complex) reconnected helical magnetic flux (divided by $2\pi\,R_0$) at the $k$th resonant surface, whereas ${\mit\Delta\Psi}_k$ (which has the same dimensions as ${\mit\Psi}_k$) is a (complex) 
measure of the strength of the current sheet (consisting of filaments running parallel to the local equilibrium magnetic
field) at the same resonant surface. 

\subsection{Toroidal Tearing Mode Dispersion Relation}\label{tear}
In the presence of the RMP, the ${\mit\Psi}_k$ and the ${\mit\Delta\Psi}_k$ values are related according to the inhomogeneous 
toroidal tearing mode dispersion relation, which takes the form
\begin{equation}\label{dispersion}
{\mit\Delta\Psi}_k = \sum_{k'=1,K} E_{kk'}\,{\mit\Psi}_{k'} + |E_{kk}|\,\chi_k.
\end{equation}
Here, $E_{kk'}$  (for $k$, $k'= 1$, $K$) is the dimensionless, Hermitian, toroidal tearing mode stability matrix, whereas the $\chi_k$ (for $k=1,K$) parameterize the current sheets driven at the various
resonant surfaces when the plasma responds  to the applied RMP in accordance with the equations of linearized, marginally-stable, ideal-MHD. 

The EPEC model determines the elements of the $E_{kk'}$ matrix using a high-$q$ approximation. In fact, if $F_{kk'}$ is the inverse of the $E_{kk'}$ matrix then
\begin{equation}\label{e56}
F_{kk'} = \oint\oint 
G(R_k,Z_k;R_{k'},Z_{k'})\,{\rm e}^{-{\rm i}\,(m_k\,\theta_k-m_{k'}\,\theta_{k'})}\,\frac{d\theta_k}{2\pi}\,\frac{d\theta_{k'}}{2\pi},
\end{equation}
and
\begin{align}
 G(R_k,Z_k;R_{k'},Z_{k'})&= \frac{(-1)^{\,n}\,\pi^{\,2}\,R_k\,R_{k'}/R_0}{2\,{\mit\Gamma}(1/2)\,{\mit\Gamma}(n+1/2)}
\left[\frac{\cosh\eta_{kk'}}{R_k^{\,2}+R_{k'}^{\,2}+(Z_k-Z_{k'})^{\,2}}\right]^{1/2}\nonumber\\[0.5ex]
&\times\left[(n-1/2)\,P_{-1/2}^{\,n-1}(\cosh\eta_{kk'})+
\frac{P_{-1/2}^{\,n+1}(\cosh\eta_{kk'})}{n+1/2}\right],
\end{align}
with
\begin{equation}
\eta_{kk'} = \tanh^{-1}\left[\frac{2\,R_k\,R_{k'}}{R_k^{\,2}+R_{k'}^{\,2}+(Z_k-Z_{k'})^{\,2}}\right].
\end{equation}
Here, the double integral in Eq.~(\ref{e56}) is taken around the
$k$th resonant surface (cylindrical coordinates $R_k$, $0$, $Z_k$;
flux coordinates $r_k$, $\theta_k$, $0$, with $r_k$ constant; resonant poloidal mode number $m_k$) and the $k'$th resonant surface (cylindrical coordinates $R_{k'}$, $0$, $Z_{k'}$;
flux coordinates $r_{k'}$, $\theta_{k'}$, $0$, with $r_{k'}$ constant; resonant poloidal mode number $m_{k'}$). Finally, the ${\mit\Gamma}(z)$ and $P_\mu^{\,\nu}(z)$ are gamma functions and associated Legendre functions, respectively.

The (complex) $\chi_k$ parameters are determined from the GPEC code.
To be more exact, the GPEC code calculates the (complex) dimensionless ${\mit\Delta}_{m_k\,n}$ parameters which measure the strengths of the ideal current sheets that develop
at the various resonant magnetic flux-surfaces in the plasma in response to the applied RMP.  The ${\mit\Delta}_{m_k\,n}$ parameters are related to the $\chi_k$ parameters according to
\begin{equation}\label{a8}
\frac{\chi_k}{R_0\,B_0} = -{\rm i} \,\frac{{\mit\Delta}_{m_k\,n}}{|E_{kk}|}\left(\frac{r_k}{R_0}\right)^2
\frac{g(r_k)}{m_k\,[a_{kk}(r_k)+ (r_k/R_0\,q_k)^{\,2}]},
\end{equation}
where $q_k= m_k/n$, and $a_{kk}(r) = \oint |\nabla r|^{-2}\,d\theta/(2\pi)$. 

\subsection{Cylindrical Tearing Mode Equation}
The cylindrical tearing mode equation for the eigenfunction, $\psi(r)$,  of the $m_k$, $n$ tearing mode takes the form
\begin{equation}
\frac{d^{\,2}\psi}{dr^{\,2}}+ \frac{1}{r}\,\frac{d\psi}{dr}-\frac{m_k^{\,2}}{r^{\,2}}\,\psi - \frac{\hat{J}_\phi'\,\psi}{r\,(1/q-n/m_k)}=0.
\end{equation}
Here, 
\begin{equation}
\hat{J}_\phi(r) = - \frac{\mu_0\,R_0\,R}{B_0}\,\frac{dP}{d{\mit\Psi}_p} - R_0^{\,2}\,B_0\,g\,\frac{dg}{d{\mit\Psi}_p},
\end{equation}
where $P({\mit\Psi}_p)$ is the total plasma pressure,  the previous expression is evaluated on the outboard mid-plane, and
$\hat{J}_\phi'\equiv d\hat{J}_\phi/dr$. 

Suppose that the plasma is surrounded by a thin resistive wall of time constant
$\tau_w$, located at $r=r_w>r_{100}$. The quantities ${\mit\Delta}_k^{nw}$ and ${\mit\Sigma}_k^{nw}$
are defined as 
\begin{align}
{\mit\Delta}_k^{nw}&= \lim_{\delta\rightarrow 0}
\left[\left.\frac{r\,\psi'_{nw}}{\psi_{nw}}\right|_{r_k+\delta}
- \left.\frac{r\,\psi'_{nw}}{\psi_{nw}}\right|_{r_k-\delta}\right],\\[0.5ex]
{\mit\Sigma}_k^{nw} &= \lim_{\delta\rightarrow 0}\left[\left.\frac{r\,\psi'_{nw}}{\psi_{nw}}\right|_{r_k+\delta}+ \left.\frac{r\,\psi'_{nw}}{\psi_{nw}}\right|_{r_k-\delta}
-2\,A_k\,(1+\ln\delta)\right],
\end{align}
where 
\begin{align}
A_k &= -\left(\frac{r\,q\,\hat{J}_\phi'}{s}\right)_{r_k},\\[0.5ex]
s(r) &= \frac{d\ln q}{d\ln r}.
\end{align}
Here, $\psi_{nw}(r)$ is a solution of the cylindrical tearing mode equation that is well-behaved at $r=0$,
continuous at $r=r_k$, and satisfies 
\begin{equation}
\left.\frac{r\,\psi'_{nw}}{\psi_{nw}}\right|_{r_{100}} = -m.
\end{equation}
 Moreover, $\psi'\equiv d\psi/dr$. 

The quantities ${\mit\Delta}_k^{pw}$ and ${\mit\Sigma}_k^{\,pw}$
are defined as 
\begin{align}
{\mit\Delta}_k^{nw}&= \lim_{\delta\rightarrow 0}
\left[\left.\frac{r\,\psi'_{pw}}{\psi_{pw}}\right|_{r_k+\delta}
- \left.\frac{r\,\psi'_{pw}}{\psi_{pw}}\right|_{r_k-\delta}\right],\\[0.5ex]
{\mit\Sigma}_k^{\,pw} &= \lim_{\delta\rightarrow 0}\left[\left.\frac{r\,\psi'_{pw}}{\psi_{pw}}\right|_{r_k+\delta}+ \left.\frac{r\,\psi'_{pw}}{\psi_{pw}}\right|_{r_k-\delta}
-2\,A_k\,(1+\ln\delta)\right].
\end{align}
Here, $\psi_{pw}(r)$ is a solution of the cylindrical tearing mode equation that is well-behaved at $r=0$,
continuous at $r=r_k$, and satisfies 
\begin{equation}
\left.\frac{r\,\psi'_{pw}}{\psi_{pw}}\right|_{r_{100}} = -m\left[\frac{1+(r_{100}/r_w)^{\,2m}}{1-(r_{100}/r_w)^{\,2m}}\right].
\end{equation}
It is also helpful to define
\begin{align}
{\mit\Sigma}_{kw} &= \frac{2m\,(r_{100}/r_w)^{\,m}}{1-(r_{100}/r_w)^{\,2m}}\,\frac{\psi_{pw}(r_{100})}{\psi_{pw}(r_k)},\\[0.5ex]
{\mit\Delta}_{kw} &= -\left(\frac{{\mit\Sigma}_{kw}^{\,2}}{{\mit\Delta}^{nw}_k-{\mit\Delta}^{pw}_k}\right),\\[0.5ex]
B_k &= -\left(\frac{r^{\,2}\,q\,\hat{J}_\phi''}{s}\right)_{r_k},
\end{align}
where $\hat{J}_\phi'' \equiv d^{\,2}\hat{J}_\phi/dr^{\,2}$. 

\section{Neoclassical Physics}\label{appb}
\subsection{Plasma Species}\label{a3}
The plasma is assumed to consist of three (charged) species (with Maxwellian distribution functions); namely,  electrons ($e$), majority ions ($i$), and impurity ions
($I$).  The charges of the  three species are $e_e=-e$, $e_i= e$, and $e_I=Z_I\,e$, respectively, where
$e$ is the magnitude of the electron charge. 
Quasi-neutrality demands that
$n_e= n_i+ Z_I\,n_I$, where $n_a(r)$ is the species-$a$ number density. 
 Let 
$\alpha_I(r) =Z_I\,(Z_{\rm eff}-1)/(Z_I-Z_{\rm eff})$,
where 
$Z_{\rm eff}(r) = (n_i+Z_I^{\,2}\,n_I)/n_e$
is the effective ion charge number. It follows that
$n_i/n_e = (Z_I-Z_{\rm eff})/(Z_I-1)$
and
$n_I/n_e= (Z_{\rm eff}-1)/[Z_I\,(Z_I-1)]$. Finally, let $Z_{{\rm eff}\,i}= (Z_I-Z_{\rm eff})/(Z_I-1)$, and
$Z_{{\rm eff}\,I}=Z_I\,(Z_{\rm eff}-1)/(Z_I-1)$. Note that $Z_{\rm eff}= Z_{{\rm eff}\,i}+ Z_{{\rm eff}\,I}$. 

\subsection{Collisionality Parameters}
Consider an equilibrium magnetic flux-surface whose label is $r$. 
Let
\begin{equation}
\frac{1}{\gamma(r)} =\frac{q}{g}\oint\frac{ B\,R^{\,2}}{B_0\,R_0^{\,2}}\,\frac{d\theta}{2\pi},
\end{equation}
where $B=|{\bf B}|$. 
It is helpful to define a new poloidal angle ${\mit\Theta}$ such that
\begin{equation}
\frac{d{\mit\Theta}}{d\theta} = \frac{\gamma\,q}{g}\,\frac{B\,R^{\,2}}{B_0\,R_0^{\,2}}.
\end{equation}
Let
\begin{align}
I_1 &=\oint \frac{B_0}{B}\,\frac{d{\mit\Theta}}{2\pi},\\[0.5ex]
I_2 &=\oint \frac{B}{B_0}\,\frac{d{\mit\Theta}}{2\pi},\\[0.5ex]
I_3 &=\oint\left(\frac{\partial B}{\partial{\mit\Theta}}\right)^2\frac{1}{B_0\,B}\,\frac{d{\mit\Theta}}{2\pi},\\[0.5ex]
I_{4,j}&= \sqrt{2\,j}\oint\frac{\cos(j\,{\mit\Theta})}{B/B_0}\,\frac{d{\mit\Theta}}{2\pi},\\[0.5ex]
I_{5,j} &= \sqrt{2\,j}\oint\frac{\cos(j\,{\mit\Theta})}{2\,(B/B_0)^{\,2}}\,\frac{d{\mit\Theta}}{2\pi},\\[0.5ex]
I_6(\lambda) &=\oint \frac{\sqrt{1-\lambda\,B/B_{\rm max}}}{B/B_0}\,\frac{d{\mit\Theta}}{2\pi},
\end{align}
where $B_{\rm max}$ is the maximum value of $B$ on the magnetic
flux-surface, and $j$ a positive integer. 
The species-$a$ transit frequency is written
$\omega_{t\,a}(r)= K_t\,\gamma\,v_{T\,a}$,
where 
\begin{equation}\label{cdef}
K_t(r) = \frac{I_1^{\,2}\,I_3}{I_2^{\,2}\,\sum_{j=1,\infty} I_{4,j}\,I_{5,j}},
\end{equation}
and $v_{T\,a} = \sqrt{2\,T_a/m_a}$.  Here, $m_a$ is the
species-$a$ mass, and $T_a(r)$ the species-$a$ temperature (in energy units). The fraction of circulating particles 
is
\begin{equation}
f_c(r) = \frac{3\,I_2}{4}\,\frac{B_0^{\,2}}{B_{\rm max}^{\,2}}
\int_0^1 \frac{\lambda\,d\lambda}{I_6(\lambda)}.
\end{equation}
Finally, the dimensionless species-$a$ collisionality parameter
is written
$\nu_{\ast\,a} (r)= K_\ast\,g_t/(\omega_{t\,a}\,\tau_{aa})$,
where 
$g_t(r) =f_c/(1-f_c)$,
\begin{align}
K_\ast(r) &= \frac{3}{8\pi}\,\frac{I_2}{I_3}\,K_t^{\,2},\\[0.5ex]
\frac{1}{\tau_{aa}(r)}&= \frac{4}{3\sqrt{\pi}}\,\frac{4\pi\,n_a\,e_a^{\,4}\,\ln{\mit\Lambda}}{(4\pi\,\epsilon_0)^{\,2}\,m_a^{\,2}\,v_{T\,a}^{\,3}}.
\end{align}
Here, the Coulomb logarithm, $\ln{\mit\Lambda}$, is assumed to take the same large constant value (i.e., $\ln{\mit\Lambda}\simeq 17$), 
independent of species. 

\subsection{Collisional Friction Matrices}
Let $x_{ab}=v_{T\,b}/v_{T\,a}$. In the following,
all quantities that are of order $(m_e/m_i)^{\,1/2}$, $(m_e/m_I)^{\,1/2}$, or smaller, are neglected with respect to unity.
The  $2\times 2$ dimensionless ion collisional friction matrices, $[F^{\,ii}](r)$, $[F^{\,iI}](r)$, $[F^{\,Ii}](r)$, and $[F^{\,II}](r)$, are defined to have the following elements:
\begin{align}
F^{\,ii}_{\,00} &= \frac{\alpha_I\,(1+m_i/m_I)}{(1+x_{iI}^{\,2})^{\,3/2}},\\[0.5ex]
F^{\,ii}_{\,01}&=\frac{3}{2}\,\frac{\alpha_I\,(1+m_i/m_I)}{(1+x_{iI}^{\,2})^{\,5/2}},\\[0.5ex]
F^{\,ii}_{\,11}& =\sqrt{2}+ \frac{\alpha_I\,[13/4+4\,x_{iI}^{\,2}+(15/2)\,x_{iI}^{\,4}]}{(1+x_{iI}^{\,2})^{\,5/2}},\\[0.5ex]
F^{\,iI}_{\,01}&=\frac{3}{2}\,\frac{T_i}{T_I}\,\frac{\alpha_I\,(1+m_I/m_i)}{x_{iI}\,(1+x_{Ii}^{\,2})^{\,5/2}},\\[0.5ex]
F^{\,iI}_{\,11}& =\frac{27}{4}\,\frac{T_i}{T_I}\,\frac{\alpha_I\,x_{iI}^{\,2}}{(1+x_{iI}^{\,2})^{\,5/2}},\\[0.5ex]
F^{\,Ii}_{\,11}& =\frac{27}{4}\,\frac{\alpha_I\,x_{iI}^{\,2}}{(1+x_{iI}^{\,2})^{\,5/2}},\\[0.5ex]
F^{\,II}_{\,11}& =\frac{T_i}{T_I}\left\{\sqrt{2}\,\alpha_I^{\,2}\,x_{Ii} + \frac{\alpha_I\,[
15/2+4\,x_{iI}^{\,2}+(13/4)\,x_{iI}^{\,4}]}{(1+x_{iI}^{\,2})^{\,5/2}}\right\},
\end{align}
 $F^{\,ii}_{\,10}=F^{\,ii}_{\,01}$, $F^{\,iI}_{\,00} =F^{\,ii}_{\,00}$, $F^{\,iI}_{\,10}=F^{\,ii}_{01}$, $F^{\,Ii}_{\,00} =F^{\,ii}_{\,00}$,
$F^{\,Ii}_{\,01}=F^{\,ii}_{01}$, 
$F^{\,Ii}_{\,10}=F^{\,iI}_{01}$, $F^{\,II}_{\,00} =F^{\,ii}_{00}$,
$F^{\,II}_{\,01}=F^{\,iI}_{01}$, 
$F^{\,II}_{\,10}=F^{\,iI}_{01}$.

The $2\times 2$ dimensionless electron collisional friction matrices, $[F^{\,ee}](r)$,   $[F^{\,ei}](r)$, and
$[F^{\,eI}](r)$, are defined to have the following elements:
$F^{\,ee}_{00} = Z_{\rm eff}$,
$F^{\,ee}_{01} = (3/2)\,Z_{\rm eff}$,
$F^{\,ee}_{10} = F^{\,ee}_{01}$,
$F^{\,ee}_{11}= \sqrt{2} + (13/4)\,Z_{\rm eff}$, $F^{\,ei}_{00} = Z_{{\rm eff}\,i}$,
$F^{\,ei}_{01} =F^{\,ei}_{11}=0$, $F^{\,ei}_{10}=(3/2)\,Z_{{\rm eff}\,i}$,  $F^{\,eI}_{00} = Z_{{\rm eff}\,I}$,
$F^{\,eI}_{01} =F^{\,eI}_{11}=0$, $F^{\,eI}_{10}=(3/2)\,Z_{{\rm eff}\,I}$. 


\subsection{Neoclassical Viscosity Matrices}\label{vmatrix}
The $2\times 2$ dimensionless species-$a$ neoclassical viscosity matrix, $[\mu^{\,a}](r)$,  is defined to have the following elements:
$\mu_{00}^{\,a} = K_{00}^{\,a}$, $\mu_{01}^{\,a}= (5/2)\,K_{00}^{\,a}- K_{01}^{\,a}$,
$\mu_{10}^{\,a}= \mu_{01}^{\,a}$,
$\mu_{11}^{\,a} = K_{11}^{\,a} - 5\,K_{01}^{\,a}+(25/4)\,K_{00}^{\,a}$.
Here,
\begin{align}
 K_{jk}^{\,e} &= g_t\,\frac{4}{3\sqrt{\pi}}\int_0^\infty
\frac{e^{-x^{\,2}}\,x^{\,5+2\,(j+k)}\,\nu_D^{\,e}(x)\,dx}{[x^{\,4}+\nu_{\ast\,e}\,\nu_D^{\,e}(x)]\,[x^{\,4}+(5\pi/8)\,(\omega_{t\,e}\,\tau_{ee})^{\,-1}\,\nu_T^{\,e}(x)]},\\[0.5ex]
\nu_D^{\,e}&= \frac{3\sqrt{\pi}}{4}\left[\left(1-\frac{1}{2\,x^2}\right)\psi(x)+\psi'(x)\right]+\frac{3\sqrt{\pi}}{4}\,Z_{\rm eff},\\[0.5ex]
\nu_\epsilon^{\,e}&= \frac{3\sqrt{\pi}}{2}\left[\psi(x)-\psi'(x)\right],\\[0.5ex]
\nu_T^{\,a}(x) &= 3\,\nu_D^{\,a}(x)+\nu_{\epsilon}^{\,a}(x),
\end{align}
and
\begin{align}
\psi(x) &= \frac{2}{\sqrt{\pi}}\int_0^{x^{\,2}}\sqrt{t}\,{\rm e}^{-t}\,dt,\\[0.5ex]
\psi'(x)&= \frac{2}{\sqrt{\pi}}\,x\,{\rm e}^{-x^{\,2}}.
\end{align}
Furthermore, 
\begin{align}
 K_{jk}^{\,i} &=g_t\,\frac{4}{3\sqrt{\pi}}\int_0^\infty
\frac{e^{-x^{\,2}}\,x^{\,4+2\,(j+k)}\,\nu_D^{\,i}(x)\,dx}{[x^{\,3}+\nu_{\ast\,i}\,\nu_D^{\,i}(x)]\,[x^{\,3}+(5\pi/8)\,(\omega_{t\,i}\,\tau_{ii})^{\,-1}\,\nu_T^{\,i}(x)]},\\[0.5ex]
 \nu_D^{\,i}&= \frac{3\sqrt{\pi}}{4}\left[\left(1-\frac{1}{2\,x^{\,2}}\right)\psi(x)+\psi'(x)\right]\frac{1}{x}\nonumber\\[0.5ex]\phantom{===}
&\phantom{=}+\frac{3\sqrt{\pi}}{4}\,\alpha_I\left[\left(1-\frac{x_{iI}^{\,2}}{2\,x^{\,2}}\right)\psi\!\left(\frac{x}{x_{iI}}\right)
+\psi'\!\left(\frac{x}{x_{iI}}\right)\right]\frac{1}{x},\\[0.5ex]
 \nu_\epsilon^{\,i}&=\frac{3\sqrt{\pi}}{2}\left[\psi(x)-\psi'(x)\right]\frac{1}{x}\nonumber\\[0.5ex]\phantom{===}
&\phantom{=}+\frac{3\sqrt{\pi}}{2}\,\alpha_I\left[\frac{m_i}{m_I}\,\psi\!\left(\frac{x}{x_{iI}}\right)
-\psi'\!\left(\frac{x}{x_{iI}}\right)\right]\frac{1}{x},
\end{align}
and, finally, 
\begin{align}
 K_{jk}^{\,I} &= g_t\,\frac{4}{3\sqrt{\pi}}\int_0^\infty
\frac{e^{-x^{\,2}}\,x^{\,4+2\,(j+k)}\,\nu_D^{\,I}(x)\,dx}{[x^{\,3}+\nu_{\ast\,I}\,\nu_D^{\,I}(x)]\,[x^{\,3}+(5\pi/8)\,(\omega_{t\,I}\,\tau_{II})^{\,-1}\,\nu_T^{\,I}(x)]},\\[0.5ex]
 \nu_D^{\,I}&= \frac{3\sqrt{\pi}}{4}\left[\left(1-\frac{1}{2\,x^{\,2}}\right)\psi(x)+\psi'(x)\right]\frac{1}{x}\nonumber\\[0.5ex]\phantom{===}
&\phantom{=}+\frac{3\sqrt{\pi}}{4}\,\frac{1}{\alpha_I}\left[\left(1-\frac{x_{Ii}^{\,2}}{2\,x^{\,2}}\right)\psi\!\left(\frac{x}{x_{Ii}}\right)
+\psi'\!\left(\frac{x}{x_{Ii}}\right)\right]\frac{1}{x},\\[0.5ex]
 \nu_\epsilon^{\,I}&= \frac{3\sqrt{\pi}}{2}\left[\psi(x)-\psi'(x)\right]\frac{1}{x}\nonumber\\[0.5ex]\phantom{===}
&\phantom{=}+\frac{3\sqrt{\pi}}{2}\,\frac{1}{\alpha_I}\left[\frac{m_I}{m_i}\,\psi\!\left(\frac{x}{x_{Ii}}\right)
-\psi'\!\left(\frac{x}{x_{Ii}}\right)\right]\frac{1}{x}.
\end{align}
Note that our expressions for the neoclassical viscosity
matrices interpolate in the most accurate manner possible
between the three standard neoclassical collisionality
regimes (i.e., the banana, plateau, and Pfirsch-Sch\"{u}ter
regimes).

\subsection{Parallel Force and Heat Balance}\label{sbalance}
Let
$[\tilde{\mu}^{\,I}] =\alpha_I^{\,2}\,(T_i/T_I)\,x_{Ii}\,[\mu^{\,I}]$.
The requirement of equilibrium force and heat balance parallel to the magnetic field leads us to define
four $2\times 2$  dimensionless ion matrices, $[L^{\,ii}](r)$, $[L^{\,iI}](r)$, $[L^{\,Ii}](r)$, and $[L^{\,II}](r)$,
where
\begin{align}
\left(\begin{array}{cc} [L^{\,ii}], & [L^{\,iI}]\\[0.5ex] [L^{\,Ii}],& [L^{\,II}]\end{array}\right)=
\left(\begin{array}{cc}[F^{\,ii}+\mu^{\,i}+ Y^{\,in}/y_n], & -[F^{\,iI}]\\[0.5ex] -[F^{\,Ii}], & [F^{\,II}+\tilde{\mu}^{\,I}]\end{array}\right)^{-1}
\left(\begin{array}{cc} [F^{\,ii}+Y^{\,in}], & -[F^{\,iI}]\\[0.5ex] -[F^{\,Ii}], & [F^{\,II}]\end{array}\right),
\end{align}
the additional  four $2\times 2$  dimensionless ion matrices, $[G^{\,ii}](r)$, $[G^{\,iI}](r)$, $[G^{\,Ii}](r)$, and $[G^{\,II}](r)$,
where
\begin{align}
\left(\begin{array}{cc} [G^{\,ii}], & [G^{\,iI}]\\[0.5ex] [G^{\,Ii}],& [G^{\,II}]\end{array}\right)=\tau_{ii}\,\langle \sigma\,v\rangle_i^{\,\rm cx}
\,\langle n_n\rangle
\left(\begin{array}{cc}[F^{\,ii}+\mu^{\,i}+ Y^{\,in}/y_n], & -[F^{\,iI}]\\[0.5ex] -[F^{\,Ii}], & [F^{\,II}+\tilde{\mu}^{\,I}]\end{array}\right)^{-1},
\end{align}
and the $2\times 2$ dimensionless electron matrices, $[Q^{\,ee}](r)$, $[G^{\,ei}](r)$, $[L^{\,ee}](r)$, $[L^{\,ei}](r)$, 
and $[L^{\,eI}](r)$, 
where
\begin{align}
[Q^{\,ee}]&= [F^{\,ee}+\mu^{\,e}]^{\,-1},\\[0.5ex]
[G^{\,ei}] &= [Q^{\,ee}]\,([F^{\,ei}]\,[G^{\,ii}]+ [F^{\,eI}]\,[G^{\,Ii}]),\\[0.5ex]
[L^{\,ee}] &= [Q^{\,ee}]\,[F^{\,ee}],\\[0.5ex]
[L^{\,ei}] &= [Q^{\,ee}]\left\{[F^{\,ei}\,]\,[L^{\,ii}]-[F^{\,ei}]+[F^{\,eI}]\,[L^{\,Ii}]\right\},\\[0.5ex]
[L^{\,eI}] &= [Q^{\,ee}]\left\{[F^{\,eI}]\,[L^{\,II}]-[F^{\,eI}]+[F^{\,ei}]\,[L^{\,iI}]\right\}.
\end{align}
Here,
\begin{align}
[Y^{\,in}] &= \tau_{ii}\,\langle \sigma\,v\rangle_i^{\,\rm cx}
\,\langle n_n\rangle\left[
\begin{array}{cc} 1,& 0\\[0.5ex]0,&E_n/T_i\end{array}
\right],\\[0.5ex]
y_n &= \frac{\langle n_n\rangle\,\langle B^{\,2}\rangle}{\langle
n_n\,B^{\,2}\rangle},
\end{align}
where
\begin{equation}
\langle A\rangle (r) \equiv\left.\oint \frac{A(r,{\mit\Theta})\,d{\mit\Theta}}{B(r,{\mit\Theta})}\right/\oint \frac{d{\mit\Theta}}{B(r,{\mit\Theta})}.
\end{equation}
Moreover, $\langle \sigma\,v\rangle_i^{\,\rm cx}$ is the flux-surface
averaged rate constant for charge exchange reactions between neutrals and majority ions, $n_n(r,{\mit\Theta})$  the
neutral particle number density, and $E_n/T_i$ the ratio of the incoming neutral energy to the majority ion energy. The parameter
$y_n$ takes into account the fact that the incoming neutrals at the
edge of an H-mode tokamak plasma are usually
concentrated at the X-point (i.e., $y_n>1$).

\subsection{Neoclassical Frequencies}
The neoclassical frequencies of the three plasma species have the following definitions:
\begin{align}
\omega_{{\rm nc}\,i}(r)&=-G_{00}^{\,ii}\,\omega_E -\left[L^{\,ii}_{00}-L^{\,ii}_{01}\left(\frac{\eta_i}{1+\eta_i}\right)\right]\omega_{\ast\,i}-
\left[L^{\,iI}_{00}-L^{\,iI}_{01}\left(\frac{\eta_I}{1+\eta_I}\right)\right]\omega_{\ast\,I},\\[0.5ex]
\omega_{{\rm nc}\,I}(r)&=-G_{00}^{\,Ii}\,\omega_E -\left[L^{\,II}_{00}-L^{\,II}_{01}\left(\frac{\eta_I}{1+\eta_I}\right)\right]\omega_{\ast\,I}-
\left[L^{\,Ii}_{00}-L^{\,Ii}_{01}\left(\frac{\eta_i}{1+\eta_i}\right)\right]\omega_{\ast\,i},\\[0.5ex]
\omega_{{\rm nc}\,e}(r)&=-G_{00}^{\,ei}\,\omega_E -\left[L^{\,ee}_{00}-L^{\,ee}_{01}\left(\frac{\eta_e}{1+\eta_e}\right)\right]\omega_{\ast\,e}-
\left[L^{\,ei}_{00}-L^{\,ei}_{01}\left(\frac{\eta_i}{1+\eta_i}\right)\right]\omega_{\ast\,i}\nonumber\\[0.5ex]
&\phantom{=}-
\left[L^{\,eI}_{00}-L^{\,eI}_{01}\left(\frac{\eta_I}{1+\eta_I}\right)\right]\omega_{\ast\,I}.
\end{align}
Here,
\begin{align}
\omega_E(r) &=-\frac{d{\mit\Phi}}{d{\mit\Psi}_p},\\[0.5ex]
\omega_{\ast\,a}(r) &= -\frac{T_a}{e_a}\,\frac{d\ln p_a}{d{\mit\Psi}_p},\\[0.5ex]
\eta_a(r) &= \frac{d\ln T_a}{d\ln n_a}.
\end{align}
Moreover, $p_a(r)=n_a\,T_a$, and ${\mit\Phi}(r)$ is the equilibrium electric scalar potential. 

\subsection{Impurity Ion Angular Rotation Velocities}\label{srotation}
Let
\begin{align}
\omega_{\theta\,I}(r)&=\frac{{\bf V}^{\,I}\cdot\nabla\theta}{{\bf B}\cdot\nabla\theta}\,\frac{R_0\,B_0\,g}{R^{\,2}},\\[0.5ex]
\omega_{\phi\,I}(r) &= {\bf V}^{\,I}\cdot\nabla\phi,
\end{align}
where ${\bf V}^{\,I}$ is the impurity ion fluid velocity, and the right-hand sides are evaluated on the outboard mid-plane. According to neoclassical theory,
\begin{align}
\omega_{\theta\,I} &= K_\theta\,\omega_{{\rm nc}\,I},\\[0.5ex]
\omega_{\phi\,I} &= \omega_E+\omega_{\ast\,I} + \omega_{\theta\,I},
\end{align}
where
\begin{equation}
K_\theta(r) = \frac{R_0^{\,2}\,B_0^{\,2}\,g^{\,2}}{R^{\,2}\,\langle B^{\,2}\rangle}.
\end{equation}

\subsection{Glasser-Greene-Johnson Data}
Let $\hat{B}=|{\bf B}|/B_0$, $\hat{\nabla}= R_0\,\nabla$, $d\psi_p/dr = f(r)/R_0$, and 
\begin{align}
J_1(r) &= \oint\frac{1}{\hat{B}}\,\frac{d{\mit\Theta}}{2\pi},\\[0.5ex]
J_2(r) &= \oint \hat{B}\,\frac{d{\mit\Theta}}{2\pi},\\[0.5ex]
J_3(r) &= \oint\frac{1}{\hat{B}^{\,3}}\,\frac{d{\mit\Theta}}{2\pi},\\[0.5ex]
J_4(r) &= \oint\frac{1}{\hat{B}\,|\hat{\nabla}\psi_p|^{\,2}}\,\frac{d{\mit\Theta}}{2\pi},\\[0.5ex]
J_5(r) &= \oint\frac{\hat{B}}{|\hat{\nabla}\psi_p|^{\,2}}\,\frac{d{\mit\Theta}}{2\pi},\\[0.5ex]
J_6(r) &= \oint\frac{1}{\hat{B}^{\,3}\,|\hat{\nabla}\psi_p|^{\,2}}\,\frac{d{\mit\Theta}}{2\pi}.
\end{align}
It follows that
\begin{align}
E(r) &= - \frac{dp/d\psi_p}{(dq/d\psi_p)^{\,2}}\,\frac{1}{\gamma}\left[\frac{d}{d\psi_p}\!\left(\frac{J_1}{\gamma}\right) - g\,\frac{dq}{d\psi_p}\,\frac{J_1}{J_2}\right]J_5,\\[0.5ex]
F(r) &= \frac{(dp/d\psi_p)^{\,2}}{(dq/d\psi_p)^{\,2}}\,\frac{1}{\gamma^{\,2}}\left[g^{\,2}\left(J_5\,J_6- J_4^{\,2}\right)+ J_5\,J_3\right],
\\[0.5ex]
H(r)&= \frac{dp/d\psi_p}{dq/d\psi_p}\,\frac{g}{\gamma}\left(J_4 - \frac{J_1\,J_5}{J_2}\right),
\end{align}
where $P(r)$ is the total plasma pressure. 
Finally,
\begin{equation}
D_R(r)= E + F + H^{\,2}.
\end{equation}

\section{Plasma Response in Inner Region}
\subsection{Linear Layer Widths}\label{linear}
Let
\begin{align}
\tau_R(r) &= \mu_0\,r^{\,2}\,\sigma_{ee}\,Q_{00}^{ee},\\[0.5ex]
\sigma_{ee}(r) &=\frac{n_e\,e^{\,2}\,\tau_{ee}}{m_e},\\[0.5ex]
\tau_H(r) &= \frac{R_0}{B_0\,g}\,\frac{\sqrt{\mu_0\,\rho}}{n\,s},\\[0.5ex]
\rho(r)&= m_i\left[n_i(r) + n_b(r)\right] + m_I\,n_I(r),\\[0.5ex]
\tau_\perp(r) &= \frac{r^{\,2}}{D_\perp},\\[0.5ex]
\tau_\varphi(r) &= \frac{r^{\,2}}{\chi_\phi},\\[0.5ex]
\tau(r)& = -\frac{\omega_{\ast\,e}}{\omega_{\ast\,i}},\\[0.5ex]
d_\beta (r)&= \frac{\sqrt{(5/3)\,m_i\,[T_e + (n_i/n_e)\,T_i + (n_I/n_e)\,T_I]}}{e\,B_0\,g},\\[0.5ex]
S(r) &= \frac{\tau_R}{\tau_H},\\[0.5ex]
P_\varphi(r) &= \frac{\tau_R}{\tau_\varphi},\\[0.5ex]
P_\perp(r) &= \frac{\tau_R}{\tau_\perp},\\[0.5ex]
D(r)&= S^{\,1/3}\,\left(\frac{\tau}{1+\tau}\right)^{1/2}\,\frac{d_\beta}{r},\\[0.5ex]
Q_E(r)&= -S^{\,1/3}\,n\,\omega_E\,\tau_H,\\[0.5ex]
Q_{e,i}(r)&= -S^{\,1/3}\,n\,\omega_{\ast\,e,i}\,\tau_H
\end{align}
Here, $\chi_e(r)$, $D_\perp(r)$, and $\chi_\phi(r)$ are the perpendicular electron energy, particle,
and toroidal momentum diffusivity profiles, respectively. Moreover, $n_b(r)$ is the number density of fast (as opposed to thermal) majority ions. 
The constant-$\psi$ linear layer width is determined from the solution of
\begin{equation}\label{e87}
\frac{d^{\,2}\,Y}{dp^{\,2}} -\left[\frac{-Q_E\,(Q_E+Q_i)+{\rm i}\,(Q_E+Q_i)\,(P_\varphi+P_\perp)\,p^{\,2}+
P_\varphi\,P_\perp\,p^{\,4}}
{{\rm i}\,(Q_E+Q_e) +\{P_\perp + {\rm i}\,(Q_E+Q_i)\,D^{\,2}\}\,p^{\,2}+(1+1/\tau)\,P_\varphi\,D^{\,2}\,p^{\,4}}
\right] p^{\,2}\, Y = 0.
\end{equation}
If the small-$p$ behavior of solution of the previous equation that is well-behaved as $p\rightarrow \infty$ is
\begin{equation}
Y(p) = Y_0\left[1 - c\,p+{\cal O}(p^{\,2})\right].
\end{equation}
then the 
linear layer width is 
\begin{equation}
\delta_{\rm linear}(r) = \frac{\pi\,|c|\,r}{S^{\,1/3}}.
\end{equation}
Note that Eq.~(\ref{e87}) can only be solved when $1+1/\tau>0$. 

\subsection{Island Widths}
Let ${\mit\Psi}_k= R_0\,B_0\,\hat{\mit\Psi}_k\,{\rm e}^{-{\rm i}\,\varphi_k}$, where $\hat{\mit\Psi}_k>0$  and $\varphi_k$ are both real quantities.
The full width (in $r$) of the magnetic island chain at the
$k$th resonant surface is
\begin{equation}
W_k = 4\,R_0\left(\frac{q\,}{g\,s}\right)^{1/2}_{r_k}\hat{\mit\Psi}_k^{\,1/2}.
\end{equation}

\subsection{Critical Island Widths}
The critical full island width (in $r$) which must be exceeded before the electron temperature is flattened within the
magnetic separatrix of the magnetic island chain at the $k$th resonant surface is
\begin{equation}\label{e64g}
W_{T_e\,k} = 5.07\left(\frac{\chi_e}{\chi_{\parallel\,e}}\right)^{1/4}_{r_k}\left(\frac{1}{\epsilon\,s\,n}\right)^{1/2}_{r_k}r_k,
\end{equation}
where $\epsilon =r/R_0$, and 
\begin{align}
\chi_{\parallel\,e}&= \frac{\chi_{\parallel\,e}^{\rm brag}\,\chi_{\parallel\,e}^{\rm max}}{\chi_{\parallel\,e}^{\rm brag}+\chi_{\parallel\,e}^{\rm max}},\\[0.5ex]
\chi_{\parallel\,e}^{\rm brag} &= \frac{1.55\,\tau_{ee}\,v_{T\,e}^{\,2}}{1+0.38\,Z_{\rm eff}},\\[0.5ex]
\chi_{\parallel\,e}^{\rm max} &= \frac{2\,R_0\,v_{T\,e}}{n\,s}\,\frac{r_k}{W_{T_e\,k}}\label{e64d}
\end{align}
Equations~(\ref{e64g})--(\ref{e64d}) must be solved iteratively for $W_{T_e\,k}/r_k$. 

The critical full island width (in $r$) which must be exceeded before the ion temperature is flattened within the
magnetic separatrix of the magnetic island chain at the $k$th resonant surface is
\begin{equation}\label{e64gg}
W_{T_i\,k}= 5.07\left(\frac{\chi_i}{\chi_{\parallel\,i}}\right)^{1/4}_{r_k}\left(\frac{1}{\epsilon\,s\,n}\right)^{1/2}_{r_k}r_k,
\end{equation}
where $\chi_i(r)$ is the perpendicular ion energy diffusivity profile, 
\begin{align}
\chi_{\parallel\,i}&= \frac{\chi_{\parallel\,i}^{\rm brag}\,\chi_{\parallel\,i}^{\rm max}}{\chi_{\parallel\,i}^{\rm brag}+\chi_{\parallel\,i}^{\rm max}},\\[0.5ex]
\chi_{\parallel\,i}^{\rm brag} &= 1.95\,\tau_{ii}\,v_{T\,i}^{\,2},\\[0.5ex]
\chi_{\parallel\,i}^{\rm max} &= \frac{2\,R_0\,v_{T\,i}}{n\,s}\,\frac{r_k}{W_{T_i\,k}}.\label{e64rr}
\end{align}
Equations~(\ref{e64gg})--(\ref{e64rr}) must be solved iteratively for $W_{T_i\,k}/r_k$.

The critical full island width (in $r$) which must be exceeded before the electron density is flattened within the
magnetic separatrix of the magnetic island chain at the $k$th resonant surface is
\begin{equation}\label{e64ggg}
W_{n_e\,k}= 5.07\left(\frac{D_\perp}{\chi_{\parallel\,i}}\right)^{1/4}_{r_k}\left(\frac{1}{\epsilon\,s\,n}\right)^{1/2}_{r_k}r_k,
\end{equation}
where 
\begin{align}
\chi_{\parallel\,i}&= \frac{\chi_{\parallel\,i}^{\rm brag}\,\chi_{\parallel\,i}^{\rm max}}{\chi_{\parallel\,i}^{\rm brag}+\chi_{\parallel\,i}^{\rm max}},\\[0.5ex]
\chi_{\parallel\,i}^{\rm brag} &= 1.95\,\tau_{ii}\,v_{T\,i}^{\,2},\\[0.5ex]
\chi_{\parallel\,i}^{\rm max} &= \frac{2\,R_0\,v_{T\,i}}{n\,s}\,\frac{r_k}{W_{n_e\,k}}.\label{e64fff}
\end{align}
Equations~(\ref{e64ggg})--(\ref{e64fff}) must be solved iteratively for $W_{n_e\,k}/r_k$.

\subsection{Resonant Plasma Response Model}
Let ${\mit\chi}_k= R_0\,B_0\,\hat{\chi}_k\,{\rm e}^{-{\rm i}\,\zeta_k}$, and $E_{kk'}=\hat{E}_{kk'}\,{\rm e}^{-{\rm i}\,\xi_{kk'}}$,
where  $\hat{\chi}_k>0$, $\zeta_k$, $\hat{E}_{kk}'>0$, and $\xi_{kk'}$ are all real quantities. Furthermore,
let $X_k= \hat{\mit\Psi}_k\,\cos\varphi_k$ and $Y_k= \hat{\mit\Psi}_k\,\sin\varphi_k$. The resonant plasma response model at the 
$k$th resonant surface takes the form
\begin{align}\label{e21}
\left(\hat{W}_k +\hat{\delta}_k\right){\cal S}_k\left(\frac{dX_k}{d\hat{t}} 
+\hat{\varpi}_k\,Y_k\right)&= f_k\,X_k+ \sum_{k'=1,K}\hat{E}_{kk'}\,(\cos\xi_{kk'}\,X_{k'}  -\sin\xi_{kk'}\,Y_{k'})\nonumber\\[0.5ex] &\phantom{=}+ {\mit\Sigma}_{kw}\,U_k + \hat{E}_{kk}\,\hat{\chi}_k\,\cos\zeta_k,\\[0.5ex]
\left(\hat{W}_k +\hat{\delta}_k\right){\cal S}_k\left(\frac{dY_k}{d\hat{t}} -\hat{\varpi}_k\,X_k\right)
&= f_k\,Y_k + \sum_{k'=1,K}\hat{E}_{kk'}\,(\cos\xi_{kk'}\,Y_{k'}
+\sin\xi_{kk'}\,X_{k'}) \nonumber\\[0.5ex]
&\phantom{=}+ {\mit\Sigma}_{kw}\,V_k+ \hat{E}_{kk}\,\hat{\chi}_k\,\sin\zeta_k,\label{e22}\\[0.5ex]
\hat{\tau}_w\,\frac{dU_k}{d\hat{t}} &={\mit\Delta}_{kw}\, U_k+ {\mit\Sigma}_{kw}\,X_k,\\[0.5ex]
\hat{\tau}_w\,\frac{dV_k}{d\hat{t}} &={\mit\Delta}_{kw}\, V_k+ {\mit\Sigma}_{kw}\,Y_k,
\end{align}
where
\begin{align}\label{e23}
\hat{W}_k &=\frac{{\cal I}\,W_k}{2\,r_k}=\frac{2\,{\cal I}}{\epsilon_{100}\,\hat{r}_k}\left(\frac{q}{g\,s}\right)^{1/2}_{r_k} (X_k^{\,2} + Y_k^{\,2})^{\,1/4},\\[0.5ex]
\hat{\delta}_k &= \frac{\delta_{\rm linear}(r_k)}{R_0\,\epsilon_{100}\,\hat{r}_k},\\[0.5ex]
{\cal S}_k&=\frac{\tau_R(r_k)}{\tau_A},\\[0.5ex]
\tau_A &= \left[\frac{\mu_0\,\rho(0)\,r_{100}^{\,2}}{B_0^{\,2}}\right]^{1/2},\\[0.5ex]
\hat{\tau}_w &= \frac{\tau_w}{\tau_A},\\[0.5ex]
f_k &=f_{b\,k}+f_{c\,k} + f_{p\,k} + f_{w\,k} + f_{s\,k},
\end{align}
Here, ${\cal I} = 0.8227$, $\epsilon_{100}=r_{100}/R_0$, $\hat{r}=r/r_{100}$, $\hat{r}_k=r_k/r_{100}$, and $\hat{t}=t/\tau_A$. Note that the EPEC resonant plasma response model interpolates smoothly
between the linear regime and the
nonlinear Rutherford regime. The linear regime corresponds to $\hat{W}_k\ll\hat{\delta}_k$, whereas the nonlinear
region corresponds to $\hat{W}_k\gg\hat{\delta}_k$.  


The perturbed bootstrap current terms in the response model take the form
\begin{align}
f_{b\,k} &= f_{b\,e\,k}+f_{b\,i\,k},\\[0.5ex]
f_{b\,e\,k} &= \alpha_{b\,e\,k}\left(f_{b\,T_e\,k} + f_{b\,n_e\,k}\right),\\[0.5ex]
f_{b\,i\,k} &= \alpha_{b\,i\,k}\left(f_{b\,T_i\,k} + f_{b\,n_i\,k}\right),\\[0.5ex]
f_{b\,T_e\,k}&=\left(\frac{\eta_e}{1+\eta_e}\right)_{r_k}\,\frac{\hat{W}_k}{\hat{W}_{T_e\,k}^{\,2}+\hat{\rho}_{\theta\,e\,k}^{\,2}+\hat{W}_k^{\,2}},\\[0.5ex]
f_{b\,n_e\,k}&=\left(\frac{1}{1+\eta_e}\right)_{r_k}\,\frac{\hat{W}_k}{\hat{W}_{n_e\,k}^{\,2}+\hat{\rho}_{\theta\,e\,k}^{\,2}+\hat{W}_k^{\,2}},\\[0.5ex]
f_{b\,T_i\,k}&=\left(\frac{\eta_i}{1+\eta_i}\right)_{r_k}\,\frac{\hat{W}_k}{\hat{W}_{T_i\,k}^{\,2}+\hat{\rho}_{\theta\,i\,k}^{\,2}+\hat{W}_k^{\,2}},\\[0.5ex]
f_{b\,n_i\,k}&=\left(\frac{1}{1+\eta_i}\right)_{r_k}\,\frac{\hat{W}_k}{\hat{W}_{n_e\,k}^{\,2}+\hat{\rho}_{\theta\,i\,k}^{\,2}+\hat{W}_k^{\,2}},
\end{align}
where
\begin{align}
\alpha_{b\,e\,k} &= -2\,{\cal I}\,I_g\left(\frac{\omega_{\ast\,e}+\omega_{{\rm nc}\,e}}{\omega_\beta}\right)_{r_k},\\[0.5ex]
\alpha_{b\,i\,k} &= 2\,{\cal I}\,I_g\left[\frac{(n_i/n_e)\left(\omega_{\ast\,i}+\omega_{{\rm nc}\,i}\right)+(Z_I\,n_I/n_e)\left(\omega_{\ast\,I}+\omega_{{\rm nc}\,I}\right)}{\omega_\beta}\right]_{r_k},\\[0.5ex]
\omega_{\beta}(r)&= \frac{s\,g\,B_0}{\mu_0\,n_e\,e\,R_0^{\,2}\,q},\\[0.5ex]
\hat{W}_{T_e\,k} &=\frac{{\cal I}\,W_{T_e\,k}}{2\,r_k} ,\\[0.5ex]
\hat{W}_{T_i\,k} &= \frac{{\cal I}\,W_{T_i\,k}}{2\,r_k},\\[0.5ex]
\hat{W}_{n_e\,k} &= \frac{{\cal I}\,W_{n_e\,k}}{2\,r_k},\\[0.5ex]
\hat{\rho}_{\theta\,e\, k} &= \left(\frac{{\cal I}\,\rho_{\theta\,e}}{2\,r}\right)_{r_k}=\left(\frac{2\,{\cal I}\,v_{T\,e}\,m_e\,q\,R_0}{e\,B_0\,g\,r^{\,2}}\right)_{r_k},\\[0.5ex]
\hat{\rho}_{\theta\,i\,k} &= \left(\frac{{\cal I}\,\rho_{\theta\,i}}{2\,r}\right)_{r_k}= \left(\frac{2\,{\cal I}\,v_{T\,i}\,m_i\,q\,R_0}{e\,B_0\,g\,r^{\,2}}\right)_{r_k},
\end{align}
and $I_g = 1.58$.

The curvature terms take the form
\begin{align}
f_{c\,k} &= \alpha_{c\,k}\left(f_{c\,T_e\,k} + f_{c\,n_e\,k} + f_{c\,T_i\,k} + f_{c\,n_i\,k}\right),\\[0.5ex]
f_{c\,T_e\,k}&=\left(\frac{n_e}{n_e+n_i}\,\frac{\eta_e}{1+\eta_e}\right)_{r_k}\,\frac{\hat{W}_k}{\hat{W}_{T_e\,k}^{\,2}+\hat{W}_k^{\,2}},\\[0.5ex]
f_{c\,n_e\,k}&=\left(\frac{n_e}{n_e+n_i}\,\frac{1}{1+\eta_e}\right)_{r_k}\,\frac{\hat{W}_k}{\hat{W}_{n_e\,k}^{\,2}+\hat{W}_k^{\,2}},\\[0.5ex]
f_{c\,T_i\,k}&=\left(\frac{n_i}{n_e+n_i}\,\frac{\eta_i}{1+\eta_i}\right)_{r_k}\,\frac{\hat{W}_k}{\hat{W}_{T_i\,k}^{\,2}+\hat{W}_k^{\,2}},\\[0.5ex]
f_{c\,n_i\,k}&=\left(\frac{n_i}{n_e+n_i}\,\frac{1}{1+\eta_i}\right)_{r_k}\,\frac{\hat{W}_k}{\hat{W}_{n_e\,k}^{\,2}+\hat{W}_k^{\,2}},
\end{align}
where
\begin{align}
\alpha_{c\,k} &= 2\,{\cal I}\,I_g\,D_R(r_k).
\end{align}
Here, $D_R(r)$ is the Glasser-Greene-Johnson resistive interchange stability parameter.

The ion polarization terms take the form
\begin{align}
f_{p\,k} &= \alpha_{p\,k}\left(f_{p\,T_i\,k} + f_{p\,n_i\,k}\right),\\[0.5ex]
f_{p\,T_i\,k}&=\left(\frac{\eta_i}{1+\eta_i}\right)_{r_k}\,\frac{\hat{W}_k}{(\hat{W}_{T_i\,k}^{\,2}+\hat{W}_k^{\,2})^{\,2}},\\[0.5ex]
f_{p\,n_i\,k}&=\left(\frac{1}{1+\eta_i}\right)_{r_k}\,\frac{\hat{W}_k}{(\hat{W}_{n_e\,k}^{\,2}+\hat{W}_k^{\,2})^{\,2}},
\end{align}
where
\begin{align}
\alpha_{p\,k} &= 8\,{\cal I}^{\,3}\, I_p\left[\frac{(\omega_{\ast\,i}+\omega_{{\rm nc}\,i})\,\omega_{{\rm nc}\,i}}{\omega_\beta\,\omega_{\mit\Omega}}\right]_{r_k},\\[0.5ex]
\omega_{\mit\Omega}(r)&= \frac{e\,g\,B_0\,s\,q}{m_i}
\end{align}
and  $I_p=1.38$. 

The wall stabilization terms take the form
\begin{equation}
f_{w\,k} = \frac{{\mit\Sigma}_{kw}^{\,2}}{{\mit\Delta}_{kw}}.
\end{equation}

Finally, the island saturation terms take the form 
\begin{align}
f_{s\,k} &=-\left(\frac{f_{s\,1\,k} \,\hat{W}_{T_e\,k}^{\,2}+ f_{s\,2\,k} \,\hat{W}_k^{\,2}}{\hat{W}_{T_e\,k}^{\,2}+ \hat{W}_k^{\,2}}\right),\\[0.5ex]
f_{s\,1\,k} &= P_{1\,k}\,\frac{W_k}{r_k}\,\ln\left(\frac{r_k}{W_k}\right)+ P_{2\,k}\,\frac{W_k}{r_k},\\[0.5ex]
f_{s\,2\,k} &= P_{3\,k}\,\frac{W_k}{r_k},\\[0.5ex]
P_{1\,k} &= 0.41\,A_k^{\,2},\\[0.5ex]
P_{2\,k} &= 0.41\left(A_k^{\,2}\left[\ln\left(\frac{1}{r_k}\right)+4.85\right] - \frac{{\mit\Sigma}_k^{nw}\,A_k}{2} - B_k - 0.44\,A_k\right),\\[0.5ex]
P_{3\,k} &= 0.8\,A_k^{\,2} - 0.27\,B_k-0.09\,A_k.
\end{align}


\section{Plasma Angular Velocity Evolution}\label{a4}
\subsection{Evolution Equations}
The quantity $\hat{\varpi}_k$ that appears in Eqs.~(\ref{e21}) and (\ref{e22}) evolves in time according to
\begin{equation}\label{e19a}
\hat{\varpi}_k(\hat{t})= \hat{\varpi}_{k\,0}-\sum_{k'=1,K}^{p=1,\infty} \frac{m_k}{m_{k'}}\,\frac{y_p(\hat{r}_k)}{y_p(\hat{r}_{k'})}\,\alpha_{k',p}(\hat{t})-\sum_{k'=1,K}^{p=1,\infty}\frac{z_p(\hat{r}_k)}{z_p(\hat{r}_{k'})}\,\beta_{k',p}(\hat{t}),
\end{equation}
Here, $\hat{\varpi}_{k\,0}= \varpi_{k\,0}\,\tau_A$, $y_p(\hat{r})=J_1(j_{1,p}\,\hat{r})/\hat{r}$, and $z_p(\hat{r})=J_0(j_{0,p}\,\hat{r})$. Moreover, $\varpi_{k\,0}$
is the so-called ``natural frequency''  (in the absence of the RMP) at the $k$th resonant surface; this quantity is defined as the helical phase velocity of a naturally unstable island chain, resonant at the surface, in the
absence of an RMP (or any other island chains). Furthermore, $J_m(z)$ is a standard Bessel function, and $j_{m,p}$ denotes the $p$th zero of this function. 
The time evolution equations for the $\alpha_{k,p}$ and $\beta_{k,p}$ parameters specify how the
plasma poloidal and toroidal angular velocity profiles are modified by the electromagnetic torques that develop within the plasma, in response to the applied
RMP, and how these modifications affect the natural frequencies. The evolution equations take the form
\begin{align}
(1+2\,Q_k^{\,2})\,\frac{d\alpha_{k,p}}{d\hat{t}}+ \left(\frac{j_{1,p}^{\,2}}{\hat{\tau}_{M\,k}}+\frac{1}{\hat{\tau}_{\theta\,k}}+\frac{1}{\hat{\tau}_{{\rm cx}\,k}}\right)\alpha_{k,p}&=
\frac{m_k^{\,2}\,[y_p(\hat{r}_k)]^{\,2}}{\hat{\rho}_k\,\epsilon_{100}^{\,2}\,[J_2(j_{1,p})]^{\,2}}\,\delta\hat{T}_k,\\[0.5ex]
\frac{d\beta_{k,p}}{d\hat{t}}+ \left(\frac{j_{0,p}^{\,2}}{\hat{\tau}_{M\,k}}+\frac{1}{\hat{\tau}_{{\rm cx}\,k}}\right)\beta_{k,p}&=
\frac{n^{\,2}\,[z_p(\hat{r}_k)]^{\,2}}{\hat{\rho}_k\,[J_1(j_{0,p})]^{\,2}}\,\delta\hat{T}_k,
\end{align}
where
\begin{align}\label{e24a}
\delta \hat{T}_k&= \sum_{k'=1,K}\hat{E}_{kk'}\left[(Y_k\,X_{k'}-X_k\,Y_{k'})\,\cos\xi_{kk'}- (X_k\,X_{k'}+Y_k\,Y_{k'})\,\sin\xi_{kk'}\right]\nonumber\\[0.5ex]
&\phantom{=}+{\mit\Sigma}_{kw}\,(Y_k\,U_k-X_k\,V_k)+\hat{E}_{kk}\,\hat{\chi}_k\,(Y_k\,\cos\zeta_k-X_k\,\sin\zeta_k).
\end{align}
Here, $Q_k = Q(r_k)$, $\hat{\rho}_k=\rho(r_k)/\rho(0)$, $\hat{\tau}_{M\,k}= r_{100}^{\,2}/[\chi_\phi(r_k)\,\tau_A]$,  $\hat{\tau}_{\theta\,k}=\tau_\theta(r_k)/\tau_A$, $\hat{\tau}_{{\rm cx}\,k} =\tau_{{\rm cx}}(r_k)/\tau_A$. 
Moreover, 
\begin{equation}\label{a70}
\tau_\theta(r) = \left. \frac{\tau_{ii}}{\mu_{00}^i}\right/\left(1+\frac{q^{\,2}\,R_0^{\,2}}{r^{\,2}\,a_{kk}}\right).
\end{equation}
is the poloidal flow damping timescale, 
and
\begin{equation}
\tau_{\rm cx}(r)=\frac{1}{\langle n_n\rangle\,\langle \sigma\,v\rangle_i^{\rm cx}}
\end{equation}
is the charge exchange damping timescale. 
Furthermore,
\begin{equation}\label{a71}
Q^{\,2}(r) =\frac{q^{\,2}\,R_0^{\,2}}{2\,r^{\,2}}\left.\left(\left\langle\frac{1}{R^{\,2}}\right\rangle-\frac{1}{\langle R^{\,2}\rangle}\right)\right/\left\langle \frac{|\nabla r|^{\,2}}{R^{\,2}}\right\rangle.
\end{equation}

\subsection{Natural Frequencies}
According to linear tearing mode theory, in the absence of the RMP, the natural frequency of the tearing mode resonant at the $k$th resonant surface is given by
\begin{equation}\label{elin}
 \varpi_{e\,k} = - n\,(\omega_E+\omega_{\ast\,e})_{r_k}.
\end{equation}
According to nonlinear tearing mode theory, in the absence of the RMP, 
the natural frequency of the tearing mode resonant at the $k$th resonant surface is given by
\begin{equation}\label{e28y}
 \varpi_{i\,k}=  -n\,(\omega_E+\omega_{\ast\,i}+\omega_{{\rm nc}\,i})_{r_k}.
 \end{equation}
 It is also helpful to define
 \begin{equation}
 \varpi_{E\,k}=  -n\,(\omega_E)_{r_k}.
 \end{equation}

The EPEC model for the natural frequency is
\begin{equation}\label{ecomp}
\varpi_{k\,0} = \varpi_{E\,k} + (\varpi_{e\,k} - \varpi_{E\,k})\,f_{e\,k} + (\varpi_{i\,k} - \varpi_{E\,k})\,f_{i\,k},
\end{equation}
where
\begin{align}
f_{e\,k}&= f_{T_e\,k}+ f_{n_e\,k},\\[0.5ex]
f_{i\,k}&= f_{T_i\,k}+ f_{n_i\,k},\\[0.5ex]
f_{T_e\,k}&=\left(\frac{\eta_e}{1+\eta_e}\right)_{r_k}\,\frac{\hat{W}_{T_e\,k}^{\,2}}{\hat{W}_{T_e\,k}^{\,2}+\hat{W}_k^{\,2}},\\[0.5ex]
f_{n_e\,k}&=\left(\frac{1}{1+\eta_e}\right)_{r_k}\,\frac{\hat{W}_{n_e\,k}^{\,2}}{\hat{W}_{n_e\,k}^{\,2}+\hat{W}_k^{\,2}},\\[0.5ex]
f_{T_i\,k}&=\left(\frac{\eta_i}{1+\eta_i}\right)_{r_k}\,\frac{\hat{W}_k^{\,2}}{\hat{W}_{T_i\,k}^{\,2}+\hat{W}_k^{\,2}},\\[0.5ex]
f_{n_i\,k}&=\left(\frac{1}{1+\eta_i}\right)_{r_k}\,\frac{\hat{W}_k^{\,2}}{\hat{W}_{n_e\,k}^{\,2}+\hat{W}_k^{\,2}}.
\end{align}

\section{Pressure Reduction} \label{a5}
The full width (in ${\mit\Psi}_N$)  of the magnetic separatrix of the island chain induced at the $k$th resonant surface is
\begin{equation}
\overline{W}_k = \left(\frac{d{\mit\Psi}_N}{dr}\right)_{r_k}\,W_k=4\left(A_{1\,k}\,\hat{\mit\Psi}_k\right)^{1/2},
\end{equation}
where 
\begin{equation}
\frac{d{\mit\Psi}_N}{dr} = \frac{r\,g}{R_0^{\,2}\,q\,|\hat{\mit\Psi}_p(0)|},
\end{equation}
$A_{1\,k}=A_1(r_k)$, $A_1 = 1/(\sigma\,|\hat{\mit\Psi}_p(0)|)$,  $\sigma(r) = d\ln q/d{\mit\Psi}_N$, and $\hat{\mit\Psi}_p(r) = {\mit\Psi}_p/(R_0^{\,2}\,B_0)$.  
However, the halfway point between the inner and outer extremities of the separatrix is located at the flux-surface where ${\mit\Psi}_N = {\mit\Psi}_N(r_k) - A_{2\,k}\,W_k^{\,2}/8$, $A_{2\,k}=A_2(r_k)$, and $A_{2\,k} = [(1/3)\,(d^{\,2}q/d{\mit\Psi}_N^{\,2})/(dq/d{\mit\Psi}_N)]_{r_k}$.
On average, the presence of a magnetic island chain of full width $W_k$ at the $k$th resonant surface
causes the electron temperature, ion temperature,  and electron number density profiles to be flattened in  annular regions, centered on the
halfway points between the inner and outer radii of the island separatrices, of widths (in ${\mit\Psi}_N$)
\begin{align}
\delta_{T_e\,k}& = \frac{2}{\pi}\,\overline{W}_k\,\tanh\left(\frac{\overline{W}_k}{\overline{W}_{T_e\,k}}\right),\\[0.5ex]
\delta_{T_i\,k}& = \frac{2}{\pi}\,\overline{W}_k\,\tanh\left(\frac{\overline{W}_k}{\overline{W}_{T_i\,k}}\right),\\[0.5ex]
\delta_{n_e\,k}& = \frac{2}{\pi}\,\overline{W}_k\,\tanh\left(\frac{\overline{W}_k}{\overline{W}_{n_e\,k}}\right),
\end{align}
respectively, where
\begin{align}
\overline{W}_{T_e\,k} &=  \left(\frac{d{\mit\Psi}_N}{dr}\right)_{r_k}\,W_{T_e\,k},\\[0.5ex]
\overline{W}_{T_i\,k} &= \left(\frac{d{\mit\Psi}_N}{dr}\right)_{r_k}\,W_{T_i\,k},\\[0.5ex]
\overline{W}_{n_e\,k} &= \left(\frac{d{\mit\Psi}_N}{dr}\right)_{r_k}\,W_{n_e\,k}.
\end{align}

The total change in the plasma pressure interior to the $j$th resonant surface due to the temperature and
density flattening at the various resonant surfaces (assuming that the pressure at the LCFS is fixed) is
\begin{align}
{\mit\Delta}P_j &=\sum_{k=j,K} 
\left\{(F_1+F_3)\,\delta_{n_e\,k}+ F_2\,\delta_{T_e\,k}+F_4\,\delta_{T_i\,k} + (F_5+F_7)\,\delta_{n_e\,k}\,\overline{W}_k^{\,2}
\right.\nonumber\\[0.5ex]
&\phantom{++}\left.+F_6\, \delta_{T_e\,k}\,\overline{W}_k^{\,2}+ F_8\,\delta_{T_i\,k}\,\overline{W}_k^{\,2}+ (F_9+F_{11})\,\delta_{n_e\,k}^{\,3}
+ F_{10}\,\delta_{T_e\,k}^{\,3}+ F_{12}\,\delta_{T_i\,k}^{\,3}\right\}_{r_k},
\end{align}
where
\begin{align}
F_1 &=\frac{dn_e}{d{\mit\Psi}_N}\,T_e,\\[0.5ex]
F_2 &=\frac{dT_e}{d{\mit\Psi}_N}\,n_e,\\[0.5ex]
F_3 &=\frac{dn_i}{d{\mit\Psi}_N}\,T_i,\\[0.5ex]
F_4 &=\frac{dT_i}{d{\mit\Psi}_N}\,n_i,\\[0.5ex]
F_5&= -\frac{A_2}{8}\left(\frac{dn_e}{d{\mit\Psi}_N}\,\frac{dT_e}{d{\mit\Psi}_N}+\frac{d^{\,2}n_e}{d{\mit\Psi}_n^{\,2}}\,T_e\right),\\[0.5ex]
F_6&= -\frac{A_2}{8}\left(\frac{dn_e}{d{\mit\Psi}_N}\,\frac{dT_e}{d{\mit\Psi}_N}+\frac{d^{\,2}T_e}{d{\mit\Psi}_n^{\,2}}\,n_e\right),\\[0.5ex]
F_7& =-\frac{A_2}{8}\left(\frac{dn_i}{d{\mit\Psi}_N}\,\frac{dT_i}{d{\mit\Psi}_N}+\frac{d^{\,2}n_i}{d{\mit\Psi}_n^{\,2}}\,T_i\right),\\[0.5ex]
F_8& =-\frac{A_2}{8}\left(\frac{dn_i}{d{\mit\Psi}_N}\,\frac{dT_i}{d{\mit\Psi}_N}+\frac{d^{\,2}T_i}{d{\mit\Psi}_n^{\,2}}\,n_i\right),\\[0.5ex]
F_9&=\frac{T_e}{24}\,\frac{d^{\,3}n_e}{d{\mit\Psi}_N^{\,3}},\\[0.5ex]
F_{10}&=\frac{n_e}{24}\,\frac{d^{\,3}T_e}{d{\mit\Psi}_N^{\,3}},\\[0.5ex]
F_{11}&=\frac{T_i}{24}\,\frac{d^{\,3}n_i}{d{\mit\Psi}_N^{\,3}},\\[0.5ex]
F_{12}&=\frac{n_i}{24}\,\frac{d^{\,3}T_i}{d{\mit\Psi}_N^{\,3}},
\end{align} 
and we have neglected the impurity ion pressure. 

\end{document}
