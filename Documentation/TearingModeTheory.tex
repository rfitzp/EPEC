\documentclass[notitlepage,12pt]{article}
\usepackage {amsmath}
\pdfoutput = 1 
\usepackage {graphicx}
\usepackage{fullpage}
\allowdisplaybreaks
\newcommand {\bpsi} {\mbox{\boldmath$\psi$}}
\newcommand {\brho} {\mbox{\boldmath$\rho$}}
\newcommand{\bmu}{\mbox{\boldmath$\mu$}}
\newcommand{\bzeta}{\mbox{\boldmath$\zeta$}}
\newcommand{\bpi}{\mbox{\boldmath$\pi$}}
\newcommand{\bsigma}{\mbox{\boldmath$\sigma$}}
\newcommand{\bomega}{\mbox{\boldmath$\omega$}}

\begin{document}
\title{\bf Tearing Modes in Tokamak Plasmas}
\author{R.~Fitzpatrick\,\footnote{rfitzp@utexas.edu}\\
Institute for Fusion Studies,\\  Department of Physics,\\  University of Texas at Austin,\\  Austin TX 78712, USA}
\maketitle

\section{Fluid Theory}
\subsection{Fundamental Quantities}
Consider an idealized tokamak plasma consisting of an equal number of electrons, with
mass $m_e$ and charge $-e$ (here, $e$ denotes the  magnitude of the electron
charge), and ions, with  mass $m_i$ and charge $+e$. We shall employ 
the symbol
\begin{equation}
T_s = \frac{1}{3} \,m_s\, \langle v_s^{2}\rangle
\end{equation}
to denote a {\em kinetic temperature}\/   measured in units of energy.
Here, $v$ is a particle speed, and the angular brackets denote an
ensemble average (Reif 1965). The kinetic temperature of species $s$ is a measure of 
the mean kinetic energy of particles of
that species. (Here, $s$ represents either $e$ for electrons, or $i$ for ions.) In tokamak plasma physics, kinetic temperature
is invariably measured in {\em electron-volts}\/ \index{electron-volt} (1 joule is equivalent to
$6.24\times 10^{18}$\,eV). 

Quasi-neutrality demands that
\begin{equation}
n_i \simeq n_e, 
\end{equation}
where $n_s$ is the {\em particle number density} (that is, the number of particles 
per cubic meter) of species $s$ (Fitzpatrick 2015). 

We can
estimate typical particle speeds in terms of the so-called {\em thermal speed}, 
\begin{equation}\label{e1.3}
v_{t\,s} = \left(\frac{2\,T_s}{m_s}\right)^{1/2}.
\end{equation}

In a tokamak plasma,  the ambient magnetic field, 
${\bf B}$,  is strong enough to significantly alter charged particle
trajectories. In particular,
magnetized plasmas are highly  anisotropic, responding differently to
forces that are parallel and perpendicular to the direction
of ${\bf B}$. As is well known, charged particles respond to the Lorentz force
by freely streaming in the direction of ${\bf B}$, while executing
circular  {\em gyro-orbits} in the plane perpendicular to ${\bf B}$ (Fitzpatrick 2008). 
The typical   {\em gyroradius}\/ of a charged particle 
gyrating in a
magnetic field is given by
\begin{equation}
\rho_s = \frac{v_{t\,s}}{{\mit\Omega}_s},
\end{equation}
where
\begin{equation}
{\mit\Omega}_s = \frac{e\,B}{m_s}
\end{equation}
is the  {\em gyrofrequency}\/ associated with the
gyration. 

The electron-ion and ion-ion {\em collision times}\/ are
written
\begin{equation}\label{e3.74}
\tau_e = \frac{6\sqrt{2}\pi^{3/2}\,\epsilon_0^{2}\,\sqrt{m_e}\,\,T_e^{3/2}}
{\ln{\mit\Lambda}_c\, e^4\, n_e},
\end{equation}
and\index{collision!time!ion-ion}
\begin{equation}\label{e3.75}
\tau_i = \frac{ 12\pi^{3/2}\,\epsilon_0^{2}\,\sqrt{m_i}\,\,T_i^{3/2}}
{\ln{\mit\Lambda}_c\, e^4\, n_e},
\end{equation}
respectively (Fitzpatrick 2015). 
Here, $\ln{\mit\Lambda}_c\simeq 15$ is 
the  {\em Coulomb logarithm}\/ (Richardson 2019). Note that $\tau_e$ is the typical time required for the cumulative effect of electron-ion collisions to
deviate the path of an electron through $90^\circ$. Likewise, $\tau_i$ is the typical time required for the cumulative effect of ion-ion collisions to
deviate the path of an ion through $90^\circ$. 

\subsection{Braginskii Equations}
The electron and ion fluid equations in a collisional plasma
take the form:
\begin{align}\label{e3.79a}
\frac{dn}{dt} + n\,\nabla\cdot{\bf V}_e &= 0,\\[0.5ex]
m_e \,n\,\frac{d {\bf V}_e}{dt} + \nabla p_e+ \nabla\cdot \bpi_e + e\, n\,
({\bf E} + {\bf V}_e\times {\bf B})& = {\bf F},\\[0.5ex]
\frac{3}{2}\frac{d p_e}{dt} + \frac{5}{2}\,p_e\,\nabla\cdot{\bf V}_e
+ \bpi_e:\nabla{\bf V}_e+ \nabla\cdot{\bf q}_e &= w_e,\label{e3.79c}
\end{align}
and
\begin{align}\label{e3.80a}
\frac{dn}{dt} + n\,\nabla\cdot{\bf V}_i &= 0,\\[0.5ex]
m_i \,n\,\frac{d {\bf V}_i}{dt} + \nabla p_i + \nabla\cdot \bpi_i - e\, n\,
({\bf E} + {\bf V}_i\times {\bf B})& =- {\bf F},\\[0.5ex]
\frac{3}{2}\frac{d p_i}{dt} + \frac{5}{2}\,p_i\,\nabla\cdot{\bf V}_i
+ \bpi_i:\nabla{\bf V}_i+ \nabla\cdot{\bf q}_i &= w_i,\label{e3.80c}
\end{align}
respectively. 
Equations (\ref{e3.79a})--(\ref{e3.79c}) and (\ref{e3.80a})--(\ref{e3.80c})  are called the {\em Braginskii equations}, because they were first obtained
in a celebrated article by S.I.~Braginskii (Braginskii 1965). Here, ${\bf V}_s$ is the species-$s$ flow velocity, $p_s=n\,T_s$ is the 
species-$s$ scalar
pressure, $\bpi_s$ is the species-$s$ viscosity tensor, ${\bf F}$ is the collisional friction force density, ${\bf q}_s$ is the species-$s$ heat flux density, $w_s$ is the species-$s$ collisional heating rate density, and ${\bf E}$ is the ambient electric field strength. Moreover, $d/dt\equiv \partial/\partial t + {\bf V}_e\cdot\nabla$ in the electron fluid equations, (\ref{e3.79a})--(\ref{e3.79c}), whereas  $d/dt\equiv \partial/\partial t + {\bf V}_i\cdot\nabla$ in the ion fluid equations,  (\ref{e3.80a})--(\ref{e3.80c}). 

A tokamak plasma is highly magnetized. In other words, 
\begin{equation}
{\mit\Omega}_i\,\tau_i, \,\,\,{\mit\Omega}_e\, \tau_e \gg 1,
\end{equation}
 which implies that the electron and ion gyroradii are much  smaller than the
corresponding mean-free-paths between $90^\circ$ collisional scattering events. In this limit, a standard two-Laguerre-polynomial
Chapman-Enskog closure scheme (Chapman \& Cowling 1953) yields
\begin{align}\label{e3.91a}
{\bf F}&= n\,e\left(\frac{{\bf j}_\parallel}{\sigma_\parallel}
+\frac{{\bf j}_\perp}{\sigma_\perp}\right) -0.71\,n\,\nabla_\parallel T_e
-\frac{3\,n}{2\,{\mit\Omega}_e\,\tau_e}\,{\bf b}\times\nabla_\perp T_e,
\\[0.5ex]
w_i &= \frac{3\,m_e}{m_i} \frac{n\,(T_e-T_i)}{\tau_e},\\[0.5ex]
w_e &= -w_i + \frac{ {\bf j}\cdot {\bf F} }{n \,e}.
\end{align}
Here, ${\bf b}={\bf B}/B$ is a unit vector parallel to the magnetic field, and ${\bf j}= n\,e\,({\bf V}_i-{\bf V}_e)$ is the net plasma current density. 
Moreover, the {\em parallel electrical conductivity}, $\sigma_\parallel$,  is given by
\begin{equation}\label{e3.83}
\sigma_\parallel = 1.96\,\frac{n \,e^2\,\tau_e}{m_e}.
\end{equation} 
whereas the {\em perpendicular electrical conductivity}, $\sigma_\perp$,  takes the form
\begin{equation}\label{e3.92}
\sigma_\perp = 0.51\,\sigma_\parallel = \frac{n\,e^2\,\tau_e}{m_e}.
\end{equation}
Note that $\nabla_\parallel(\cdots) \equiv [{\bf b}\cdot\nabla
(\cdots)]\,{\bf b}$ denotes a
gradient parallel to the magnetic field, whereas $\nabla_\perp \equiv
 \nabla-\nabla_\parallel$ denotes a gradient perpendicular to the magnetic
field. Likewise, ${\bf j}_\parallel \equiv ({\bf b}\cdot{\bf j})\,{\bf b}$
represents the component of the plasma current flowing parallel to the
magnetic field, whereas ${\bf j}_\perp \equiv {\bf j} - {\bf j}_\parallel$
represents the perpendicular component of the plasma current.

In a highly magnetized plasma, the electron and ion heat flux densities are written 
\begin{align}
{\bf q}_e &= -\kappa_\parallel^e\,\nabla_\parallel T_e -\kappa_\perp^e\,
\nabla_\perp T_e
-\kappa_\times^e\,{\bf b}\times\nabla_\perp T_e- 0.71\,\frac{T_e}{e}\,{\bf j}_\parallel-
\frac{3\,T_e}{2\,{\mit\Omega}_e\,\tau_e\,e}\,{\bf b}\times{\bf j}_\perp,\label{e3.94a}\\[0.5ex]
{\bf q}_i &= -\kappa_\parallel^i\,\nabla_\parallel T_i -\kappa_\perp^i\,
\nabla_\perp T_i
+\kappa_\times^i\,{\bf b}\times\nabla_\perp T_i,\label{e3.94b}
\end{align}
respectively. Here, the {\em parallel thermal conductivities}, which control the diffusion of heat parallel to magnetic field-lines, are 
given by
\begin{align}\label{e3.87a}
\kappa_\parallel^e &= 3.2\,\,\frac{n\,\tau_e\,T_e}{m_e},\\[0.5ex]
\kappa_\parallel^i &= 3.9\,\,\frac{n\,\tau_i\,T_i}{m_i},\label{e3.87b}
\end{align} 
whereas the {\em perpendicular thermal conductivities}, which control the diffusion of heat perpendicular to magnetic flux-surfaces, take the form
\begin{align}
\kappa_\perp^e &= 4.7\,\frac{n\,T_e}{m_e\,{\mit\Omega}_e^{\,2}\,\tau_e},\\[0.5ex]
\kappa_\perp^i &= 2\, \frac{n\,T_i}{m_i\,{\mit\Omega}_i^{\,2}\,\tau_i}.
\end{align}
Finally, the {\em cross thermal conductivities}, which control the flow of heat within magnetic flux-surfaces,  are written
\begin{align}
\kappa_\times^e &= \frac{5\,n\,T_e}{2\,m_e\,{\mit\Omega}_e},\\[0.5ex]
\kappa_\times^i &=\frac{5\,n\,T_i}{2\,m_i\,{\mit\Omega}_i}.
\end{align}

In order to describe the viscosity tensor in a magnetized plasma, it is
helpful to define the {\em rate-of-strain tensor} \index{rate-of-strain tensor} 
\begin{equation}
W_{\alpha\beta} = \frac{\partial V_\alpha}{\partial r_\beta}
+ \frac{\partial V_\beta}{\partial r_\alpha} - \frac{2}{3} \,\nabla\cdot{\bf V}\,
\delta_{\alpha\beta}.
\end{equation}
Obviously, there is a separate rate-of-strain tensor for the electron and ion
fluids. It is easily demonstrated that this tensor is zero if the plasma
translates, or rotates as a rigid body, or if it undergoes isotropic
compression. Thus, the rate-of-strain tensor measures the  deformation  of
plasma volume elements. 

In a highly magnetized plasma, the viscosity tensor is best described as the
sum of  five component tensors,
\begin{equation}
\bpi  = \sum_{n=0,4} \bpi_n,
\end{equation}
where 
\begin{equation}
\bpi_0 = - 3\,\eta_0\,\left({\bf b}{\bf b} - \frac{1}{3}\,{\bf I}\right)
\left({\bf b}{\bf b} - \frac{1}{3}\,{\bf I}\right): \nabla {\bf V},
\end{equation}
with
\begin{equation}
\bpi_1 =- \eta_1\left[{\bf I}_\perp \cdot{\bf W}\cdot{\bf I}_\perp
+ \frac{1}{2}\,{\bf I}_\perp\,({\bf b}\cdot{\bf W}\cdot{\bf b})\right],
\end{equation}
and
\begin{equation}
\bpi_2 = -4\,\eta_1\,\left({\bf I}_\perp\cdot
{\bf W}\cdot{\bf b}{\bf b}
+ {\bf b}{\bf b}\cdot{\bf W} \cdot{\bf I}_\perp\right).
\end{equation}
plus
\begin{equation}
\bpi_3 = \frac{\eta_3}{2}\,\left( {\bf b}\times 
{\bf W}\cdot{\bf I}_\perp - {\bf I}_\perp\cdot{\bf W}\times{\bf b}
\right),
\end{equation}
and
\begin{equation}
\bpi_4 = 2\,\eta_3\,\left({\bf b} \times{\bf W} \cdot {\bf b}{\bf b}
- {\bf b}{\bf b} \cdot{\bf W} \times{\bf b}\right).
\end{equation}
Here, ${\bf I}$ is the identity tensor, and 
${\bf I}_\perp = {\bf I} - {\bf b}{\bf b}$. The previous 
expressions are valid for both electrons and ions. 

The tensor $\bpi_0$ describes what is known as {\em parallel viscosity}.
This is a viscosity that controls the variation along magnetic field-lines of the
velocity component parallel to  field-lines.
 The parallel
viscosity coefficients are given by 
\begin{align}\label{e3.89a}
\eta_0^e &= 0.73\,n\,\tau_e\,T_e,\\[0.5ex]
\eta_0^i &= 0.96\,n\,\tau_i\,T_i,\label{e3.89b}
\end{align}

The tensors $\bpi_1$ and $\bpi_2$ describe what is known
as {\em perpendicular viscosity}. This is a viscosity
that controls the variation perpendicular to magnetic field-lines
of the velocity components perpendicular to field-lines. The perpendicular
viscosity coefficients are given by
\begin{align}
\eta_1^e &= 0.51\, \frac{n\,T_e}{{\mit\Omega}_e^{\,2}\,\tau_e},\\[0.5ex]
\eta_1^i &= \frac{3\, n\,T_i}{10\,{\mit\Omega}_i^{\,2}\,\tau_i}.
\end{align}

Finally, the tensors $\bpi_3$ and $\bpi_4$ describe what is known
as {\em gyroviscosity}. This is not really viscosity at all, because the
associated viscous stresses are always perpendicular to the velocity, implying that
there is no dissipation ({\rm i.e.}, viscous heating) associated with
this effect. The gyroviscosity coefficients are given by 
\begin{align}
\eta_3^e &= -\frac{n\,T_e}{2\,{\mit\Omega}_e} ,\\[0.5ex]
\eta_3^i &= \frac{n\,T_i}{2\,{\mit\Omega}_i}.
\end{align}

\subsection{Normalization of Braginskii Equations}\label{s3.9}
As we have just seen, the Braginskii equations contain terms that describe a
very wide range of different physical phenomena. For this reason, they are
extremely complicated. Fortunately, however, it is not generally 
necessary to retain all of
the terms in these equations when investigating tearing mode dynamics in tokamak plasmas. In this section, we shall attempt to construct a systematic
normalization scheme for the Braginskii equations that will, hopefully, enable us
to determine which terms to keep, and which to discard, when investigating
tearing mode dynamics. 

 It is convenient to split the friction force
${\bf F}$ into a component ${\bf F}_U$ corresponding to resistivity, and a
component ${\bf F}_T$ corresponding to the thermal force. Thus,
\begin{equation}
{\bf F} = {\bf F}_U+{\bf F}_T,
\end{equation}
where
\begin{align}
{\bf F}_U&= n\,e\left(\frac{{\bf j}_\parallel}{\sigma_\parallel}
+\frac{{\bf j}_\perp}{\sigma_\perp}\right), \\[0.5ex]
{\bf F}_T &= - 0.71\,n\,\nabla_\parallel T_e
-\frac{3\,n}{2\,{\mit\Omega}_e\,\tau_e}\,{\bf b}\times\nabla_\perp T_e.
\end{align}
Likewise, the electron collisional energy gain term $w_e$ is split
into a component $-w_i$ corresponding to the energy lost to the ions (in the
ion rest frame), a component $w_U$ corresponding to work done by the friction
force ${\bf F}_U$, and a component $w_T$ corresponding to work done by the
thermal force ${\bf F}_T$. Thus,
\begin{equation}
w_e = -w_i + w_U + w_T,
\end{equation}
where
\begin{align}
w_U &= \frac{{\bf j}\cdot{\bf F}_U}{n\,e},\\[0.5ex]
w_T&= \frac{{\bf j}\cdot{\bf F}_T}{n\,e}.
\end{align}
Finally, it is helpful to split the electron heat flux density ${\bf q}_e$ into
a diffusive component ${\bf q}_{Te}$ and a convective component ${\bf q}_{Ue}$. 
Thus,
\begin{equation}
{\bf q}_e = {\bf q}_{Te} + {\bf q}_{Ue},
\end{equation}
where
\begin{align}
{\bf q}_{Te} &=-\kappa_\parallel^e\,\nabla_\parallel T_e -\kappa_\perp^e\,
\nabla_\perp T_e
-\kappa_\times^e\,{\bf b}\times\nabla_\perp T_e,\\[0.5ex]
{\bf q}_{Ue}&=  0.71\,\frac{T_e}{e}\,{\bf j}_\parallel-
\frac{3\,T_e}{2\,{\mit\Omega}_e\,\tau_e\,e}\,{\bf b}\times{\bf j}_\perp.
\end{align}

Let us, first of all, consider the electron fluid equations, which can
be written:
\begin{align}
\frac{dn}{dt} + n\,\nabla\cdot{\bf V}_e &= 0,\\[0.5ex]
m_e\, n\,\frac{d {\bf V}_e}{dt} + \nabla p_e+ \nabla\cdot \bpi_e + e\, n\,
({\bf E} + {\bf V}_e\times {\bf B})& = {\bf F}_U +{\bf F}_T,\\[0.5ex]
\frac{3}{2}\frac{d p_e}{dt} + \frac{5}{2}\,p_e\,\nabla\cdot{\bf V}_e
+ \bpi_e:\nabla{\bf V}_e+ \nabla\cdot{\bf q}_{Te} 
+ \nabla\cdot{\bf q}_{Ue} &= -w_i+w_U + w_T.
\end{align}
Let $\bar{n}$, $\bar{v}_e$, $\bar{\tau}_e$, $\bar{l}_e=\bar{v}_e\,\bar{\tau}_e$, $\bar{B}$,
and $\bar{\rho}_e =\bar{ v}_e/(e\bar{B}/m_e)$,  be typical values
of the particle density, the electron thermal velocity, the electron collision time, the electron
mean-free-path between collisions,  the magnetic field-strength, and the
electron gyroradius,  respectively. 
Suppose that the typical spatial variation lengthscale of fluid variables is $L$. Let
 \begin{align}
\delta_e &=\frac{\bar{\rho}_e}{L},\\[0.5ex]
\zeta_e &= \frac{\bar{\rho}_e}{\bar{l}_e},\\[0.5ex]
\mu &= \sqrt{\frac{m_e}{m_i}}.
\end{align}
All three of these parameters are assumed to be  small
compared to unity. Finally, the typical electron flow velocity is assumed to be of order $\delta_e\,\bar{v}_e$. This corresponds
to the so-called {\em drift ordering}\/ in which the flow velocity is comparable to the curvature and grad-B particle
drift velocities (Fitzpatrick 2015). The drift ordering is appropriate to tearing modes in tokamak plasmas, which are comparatively
slowly growing instabilities. 

 We
define the following normalized quantities: {\small
\begin{align}
\hat{n} &= \frac{n}{\bar{n}},&
\hat{v}_e &= \frac{v_e}{\bar{v}_e}, &
\hat{\bf r} &= \frac{{\bf r}}{ L},\nonumber\\[0.5ex]
\widehat{\nabla} &= L\,\nabla, &
\hat{t} &= \frac{\delta_e\,\bar{v}_e\,t}{L}, &
\widehat{\bf V}_e &= \frac{{\bf V}_e}{\delta_e\,\bar{v}_e}, \nonumber\\[0.5ex]
\widehat{\bf B}&= \frac{{\bf B}}{\bar{B}}, &
\widehat{\bf E}& = \frac{{\bf E}}{ \delta_e\,\bar{v}_e\,\bar{B}},&
\widehat{\bf j}& =\frac{{\bf j} }{n\,e\,\delta_e\,\bar{v}_e}, \nonumber\\[0.5ex]
\hat{p}_e &= \frac{p_e}{m_e\,\bar{n}\,\bar{v}_e^{2}}, &
\widehat{\bpi}_e&= \frac{\bpi_e}{\delta_e^{\,2}\,\zeta_e^{-1}\,m_e\,\bar{n}\,\bar{v}_e^{2}}, &
\widehat{\bf q}_{Te}& = \frac{{\bf q}_{Te} }{ \delta_e\,\zeta_e^{-1}\,m_e\,\bar{n}\,\bar{v}_e^{3}}, \nonumber\\[0.5ex]
\widehat{\bf q}_{Ue} &= \frac{{\bf q}_{Ue} }{\delta_e\,m_e\,\bar{n}\,\bar{v}_e^{3}}, &
\widehat{\bf F}_U &= \frac{{\bf F}_U}{\zeta_e\,m_e\,\bar{n}\,\bar{v}_e^{2}/L}, &
\widehat{\bf F}_T &= \frac{{\bf F}_T}{m_e\,\bar{n}\,\bar{v}_e^{2}/L},\nonumber\\[0.5ex]
\widehat{w}_i &= \frac{w_i}{\delta_e^{-1}\,\zeta_e\,\mu^2\, m_e\,\bar{n}\,\bar{v}_e^{3}/L},&
\widehat{w}_U &= \frac{w_U}{\delta_e\,\zeta_e\, m_e\,\bar{n}\,\bar{v}_e^{3}/L},&
\widehat{w}_T &= \frac{w_T}{\delta_e\, m_e\,\bar{n}\,\bar{v}_e^{3}/L}.\nonumber
\end{align}}

The normalization procedure is designed to make all hatted quantities ${\cal O}(1)$.
The normalization of the electric field is chosen 
 such that the ${\bf E}\times{\bf B}$
velocity is of  similar magnitude to the electron fluid velocity. Note that the parallel viscosity
makes an ${\cal O}(1)$ contribution to $\widehat{\bpi}_e$, whereas the gyroviscosity
makes an ${\cal O}(\zeta_e)$ contribution, and the perpendicular viscosity only 
makes an ${\cal O}(\zeta_e^{2})$ contribution. Likewise, the parallel thermal
conductivity 
makes an ${\cal O}(1)$ contribution to $\widehat{\bf q}_{Te}$, whereas the cross
conductivity 
makes an ${\cal O}(\zeta_e)$ contribution, and the perpendicular conductivity only 
makes an ${\cal O}(\zeta_e^{2})$ contribution. Similarly, the parallel components
of ${\bf F}_T$ and ${\bf q}_{Ue}$ are ${\cal O}(1)$, whereas the perpendicular
components are ${\cal O}(\zeta_e)$. 

The normalized electron fluid equations take the form:
\begin{align}
\frac{d\hat{n}}{d\hat{t}} + \hat{n}\,\widehat{\nabla}\cdot\widehat{\bf V}_e &=0,\label{e3.113a}\\[0.5ex]
\hat{n}\,\frac{d\widehat{\bf V}_e}{d\hat{t}} +
\delta_e^{-2}\,\widehat{\nabla}\hat{p}_e +\zeta_e^{-1}\,\widehat{\nabla}
\cdot\widehat{\bpi}_e  
+ \delta_e^{-2}\,\hat{n}\,(\widehat{\bf E} + \widehat{\bf V}_e
\times\widehat{\bf B}) &= \delta_e^{-2}\,\zeta_e\,\widehat{\bf F}_U +\delta_e^{-2}\,\widehat{\bf F}_T, \\[0.5ex]
\frac{3}{2}\frac{d\hat{p}_e}{d\hat{t}} + \frac{5}{2}\,
\hat{p}_e\,\widehat{\nabla}\cdot\widehat{\bf V}_e + \delta_e^{2}\,
\zeta_e^{-1}\,\widehat{\bpi}_e : \widehat{\nabla}\cdot\widehat{\bf V}_e&\nonumber\label{e3.113c}\\[0.5ex] 
+\zeta_e^{-1} \,\widehat{\nabla}\cdot\widehat{\bf q}_{Te}
 +\widehat{\nabla}\cdot\widehat{\bf q}_{Ue} 
&= -\delta_e^{-2}\,\zeta_e\,\mu^2\,\widehat{w}_i + \zeta_e\,
\widehat{w}_U +\widehat{w}_T.
\end{align}
The only large or small (compared to unity) quantities in these equations are the
parameters $\delta_e$, $\zeta_e$, and $\mu$. 
Here, $d/d\hat{t}\equiv\partial /\partial\hat{t} +
 \widehat{\bf V}_e\cdot\widehat{\nabla}$. It is assumed that $T_e\sim T_i$.

Let us now consider the ion fluid equations, which can be written:
\begin{align}
\frac{dn}{dt} + n\,\nabla\cdot{\bf V}_i &= 0,\\[0.5ex]
m_i\, n\,\frac{d {\bf V}_i}{dt} + \nabla p_i + \nabla\cdot \bpi_i - e\, n\,
({\bf E} + {\bf V}_i\times {\bf B})& =- {\bf F}_U -{\bf F}_T,\\[0.5ex]
\frac{3}{2}\frac{d p_i}{dt} + \frac{5}{2}\,p_i\,\nabla \cdot {\bf V}_i
+ \bpi_i:\nabla{\bf V}_i+ \nabla\cdot{\bf q}_i & =w_i.
\end{align}
It is convenient to adopt a normalization scheme for the ion equations
which is similar to, but independent of, that employed to normalize the
electron equations. Let  $\bar{n}$, $\bar{v}_i$, $\bar{l}_i$, $\bar{B}$,
and $\bar{\rho}_i =\bar{ v}_i/(e\bar{B}/m_i)$,  be typical values
of the particle density, the ion  thermal velocity, the ion
mean-free-path between collisions,  the magnetic field-strength, and the
ion gyroradius,  respectively. 
Suppose that 
the typical spatial variation lengthscale of fluid quantities is $L$. Let
 \begin{align}
\delta_i &= \frac{\bar{\rho}_i}{L},\\[0.5ex]
\zeta_i &= \frac{\bar{\rho}_i}{\bar{l}_i},\\[0.5ex]
\mu &= \sqrt{\frac{m_e}{m_i}}.
\end{align}
All three of these parameters are assumed to be  small 
compared to unity.  As before, we adopt the drift ordering in which the typical ion flow velocity is assumed to be of order $\delta_i\,\bar{v}_i$. 

We
define the following normalized quantities: 
\begin{align}
\hat{n} &=\frac{ n}{\bar{n}},&
\hat{v}_i &= \frac{v_i}{\bar{v}_i},& 
\hat{\bf r} &= \frac{{\bf r}}{ L}, \nonumber\\[0.5ex]
\widehat{\nabla} &= L\,\nabla, &
\hat{t} &= \frac{\delta_i\,\bar{v}_i\,t}{L}, &
\widehat{\bf V}_i &= \frac{{\bf V}_i}{\delta_i\,\bar{v}_i}, \nonumber\\[0.5ex]
\widehat{\bf B}&= \frac{{\bf B}}{\bar{B}}, &
\widehat{\bf E} &= \frac{{\bf E}}{ \delta_i\,\bar{v}_i\,\bar{B}},&
\widehat{\bf j} &=\frac{{\bf j} }{ n\,e\,\delta_i\,\bar{v}_i},\nonumber\\[0.5ex]
 \hat{p}_i &=\frac{ p_i}{m_i\,\bar{n}\,\bar{v}_i^{2}}, &
 \widehat{\bpi}_i&= \frac{\bpi_i}{\delta_i^{\,2}\,\zeta_i^{-1}\,m_i\,\bar{n}\,\bar{v}_i^{2}}, &
\widehat{\bf q}_{i} &= \frac{{\bf q}_{i} }{ \delta_i\,\zeta_i^{-1}\,m_i\,\bar{n}\,\bar{v}_i^{3}},  \nonumber\\[0.5ex]
\widehat{\bf F}_U &= \frac{{\bf F}_U}{ \zeta_i\,\mu\,m_i\,\bar{n}\,\bar{v}_i^{2}/L}, &
\widehat{\bf F}_T &=\frac{ {\bf F}_T}{m_i\,\bar{n}\,\bar{v}_i^{2}/L},&
\widehat{w}_i& = \frac{w_i}{\delta_i^{-1}\,\zeta_i\,\mu\, m_i\,\bar{n}\,\bar{v}_i^{3}/L}.\nonumber
\end{align}

As before, the normalization procedure is designed to make all hatted quantities ${\cal O}(1)$.
The normalization of the electric field is chosen 
 such that the ${\bf E}\times{\bf B}$
velocity is of similar  magnitude to the ion fluid velocity. Note that the parallel viscosity
makes an ${\cal O}(1)$ contribution to $\widehat{\bpi}_i$, whereas the gyroviscosity
makes an ${\cal O}(\zeta_i)$ contribution, and the perpendicular viscosity only 
makes an ${\cal O}(\zeta_i^{2})$ contribution. Likewise, the parallel thermal
conductivity 
makes an ${\cal O}(1)$ contribution to $\widehat{\bf q}_{i}$, whereas the cross
conductivity 
makes an ${\cal O}(\zeta_i)$ contribution, and the perpendicular conductivity only 
makes an ${\cal O}(\zeta_i^{2})$ contribution. Similarly, the parallel component
of ${\bf F}_T$ is  ${\cal O}(1)$, whereas the perpendicular
component is  ${\cal O}(\zeta_i\,\mu)$. 

The normalized ion fluid equations take the form:
\begin{align}\label{e3.116a}
\frac{d\hat{n}}{d\hat{t}} + \hat{n}\,\widehat{\nabla}\cdot\widehat{\bf V}_i &=0,\\[0.5ex]
\hat{n}\,\frac{d\widehat{\bf V}_i}{d\hat{t}} +
\delta_i^{-2}\,\widehat{\nabla}\hat{p}_i+\zeta_i^{-1}\,\widehat{\nabla}
\cdot\widehat{\bpi}_i  
-\delta_i^{-2}\,\hat{n}\,(\widehat{\bf E} + \widehat{\bf V}_i
\times\widehat{\bf B}) &=-\delta_i^{-2}\,\zeta_i\,\mu\,\widehat{\bf F}_U
 - \delta_i^{-2}\,\widehat{\bf F}_T, \\[0.5ex]
\frac{3}{2}\frac{d\hat{p}_i}{d\hat{t}} + \frac{5}{2}\,
\hat{p}_i\,\widehat{\nabla}\cdot\widehat{\bf V}_i + \delta_i^{2}\,
\zeta_i^{-1}\,\widehat{\bpi}_i : \widehat{\nabla}\cdot\widehat{\bf V}_i
+\zeta_i^{-1} \,\widehat{\nabla}\cdot\widehat{\bf q}_{i}
&= \delta_i^{-2}\,\zeta_i\,\mu\,\widehat{w}_i. \label{e3.116c}
\end{align}
The only large or small (compared to unity) quantities in these equations are the
parameters $\delta_i$, $\zeta_i$, and $\mu$. 
Here, $d/d\hat{t}\equiv\partial /\partial\hat{t} +
 \widehat{\bf V}_i\cdot\widehat{\nabla}$.

\subsection{Lowest-Order Fluid Equations}
If we restore dimensions to the Braginskii equations then we can write them in the form
\begin{align}
\frac{\partial n}{\partial t} + \nabla\cdot(n\,{\bf V}_e) &= 0,\label{e3.151a}\\[0.5ex]
m_e\, n\,\frac{\partial  {\bf V}_e}{\partial t} +
m_e\, n\,({\bf V}_e\cdot\nabla){\bf V}_e+
 [\delta_e^{-2}]\,\nabla p_e+
[\zeta_e^{-1}]\, \nabla\cdot \bpi_e &\label{e3.151b}\\[0.5ex]
 +[\delta_e^{-2}]\,e\, n\,
({\bf E} + {\bf V}_e\times {\bf B})& =[\delta_e^{-2}\,\zeta_e]\, {\bf F}_U + [\delta_e^{-2}]\,
{\bf F}_T,\nonumber\\[0.5ex]
\frac{3}{2}\frac{\partial  p_e}{\partial t} + \frac{3}{2}
 \,({\bf V}_e\cdot\nabla)\, p_e 
+ \frac{5}{2}\,p_e\,\nabla\cdot{\bf V}_e+[\delta_e^{\,2}\,\zeta_e^{-1}]\, \bpi_e:\nabla{\bf V}_e&\label{e3.151c}\\[0.5ex]
+ [\zeta_e^{-1}]\,\nabla\cdot{\bf q}_{Te} +\nabla\cdot{\bf q}_{Ue}  &=-[\delta_e^{-2}\,\zeta_e\,\mu^2]\,
 w_i+ [\zeta_e]\,w_U + w_T,\nonumber
\end{align}
and
\begin{align}\label{e3.152a}
\frac{\partial n}{\partial t} +\nabla\cdot(n\,{\bf V}_i) &= 0,\\[0.5ex]
m_i \,n\,\frac{\partial  {\bf V}_i}{\partial t} +
m_i \,n\,({\bf V}_i\cdot\nabla) {\bf V}_i +
 [\delta_i^{-2}]\,\nabla p_i + 
[\zeta_i^{-1}]\,\nabla\cdot \bpi_i &\label{e3.152b}\\[0.5ex]
- [\delta_i^{-2}]\,e \,n\,
({\bf E} + {\bf V}_i\times {\bf B})& =-[\delta_i^{-2}\,\zeta_i\,\mu]\, {\bf F}_U - [\delta_i^{-2}]\,{\bf F}_T,
\nonumber\\[0.5ex]
\frac{3}{2}\frac{\partial  p_i}{\partial t} +   \frac{3}{2}
 \,({\bf V}_i\cdot\nabla) \,p_i
+ \frac{5}{2}\,p_i\,\nabla\cdot{\bf V}_i+[\delta_i^{\,2}\,\zeta_i^{-1}]\, \bpi_i:\nabla{\bf V}_i\nonumber\\[0.5ex]
+ [\zeta_i^{-1}]\,\nabla\cdot{\bf q}_i &= [\delta_i^{-2}\,\zeta_i\,\mu]\,w_i.\label{e3.152c}
\end{align}
The terms in square brackets are just to remind us that the terms they precede are smaller or larger than the
other terms in the equations (by the corresponding terms within the brackets). 

If we assume that $\delta_e\sim \zeta_e\sim \mu$ then the dominant terms in the electron energy conservation equation
(\ref{e3.151c}) yield
\begin{equation}
\nabla\cdot{\bf q}_{T_e}\simeq \nabla\cdot(-\kappa_\parallel^e\,\nabla_\parallel T_e)\simeq 0,
\end{equation}
which implies that 
\begin{equation}
{\bf B}\cdot\nabla T_e=0. \label{e75}
\end{equation}
In other words, the parallel electron heat conductivity in a tokamak plasma is usually
sufficiently large that it forces the electron temperature to be constant on magnetic flux-surfaces. 
Likewise, if we assume that $\delta_i\sim \zeta_i\sim \mu$ then the dominant terms in the ion energy conservation equation
(\ref{e3.152c}) yield
\begin{equation}
\nabla\cdot{\bf q}_{T_i}\simeq \nabla\cdot(-\kappa_\parallel^i\,\nabla_\parallel T_i)\simeq 0,
\end{equation}
which implies that 
\begin{equation}
{\bf B}\cdot\nabla T_i=0.\label{e77}
\end{equation}
In other words, the parallel ion heat conductivity in a tokamak plasma is usually
sufficiently large that it forces the ion temperature to be constant on magnetic flux-surfaces. 

 The dominant terms in the electron and ion momentum conservation equations,
(\ref{e3.151b}) and (\ref{e3.152b}), respectively, yield
\begin{align}\label{e78}
\nabla p_e + n\,e\,({\bf E}+{\bf V}_e\times {\bf B}) &\simeq 0,\\[0.5ex]
\nabla p_i - n\,e\,({\bf E}+{\bf V}_i\times {\bf B}) &\simeq 0.\label{e79}
\end{align}
The sum of the preceding two equations gives
\begin{equation}\label{e80}
 {\bf j}\times {\bf B} \simeq\nabla p,
\end{equation}
where $p=p_e+p_i$. In other words, a conventional tokamak plasma exists in a state in which electromagnetic
forces exactly balance the  total scalar pressure force. 
It follows from the previous equation that 
\begin{equation}
{\bf B}\cdot\nabla p\simeq 0.
\end{equation}
In other words, the pressure in a conventional tokamak plasma is constant on magnetic flux-surfaces. 
 However, given that $p=n\,(T_e+T_i)$, and making use of Eqs.~(\ref{e75}) and (\ref{e77}), we deduce that 
\begin{equation}
{\bf B}\cdot\nabla n\simeq 0.
\end{equation}
In other words, the particle density in a conventional tokamak plasma is also  constant on
magnetic flux-surfaces. 

 Taking the difference of the electron and ion particle
conservation equations, (\ref{e3.151a}) and (\ref{e3.152a}), respectively, we obtain
\begin{equation}\label{e82}
\nabla\cdot {\bf j} = 0.
\end{equation}
 Equations~(\ref{e80}) and (\ref{e82}) can be combined to give
\begin{equation}\label{e84}
({\bf B}\cdot\nabla)\,{\bf j } - ({\bf j}\cdot\nabla)\,{\bf B} \simeq {\bf 0}.
\end{equation}

Finally, Equations~(\ref{e78}) and (\ref{e79}) imply that
\begin{align}
{\bf V}_{\perp\,e} &\simeq {\bf V}_E + {\bf V}_{\ast\,e},\\[0.5ex]
{\bf V}_{\perp\,i} &\simeq {\bf V}_E + {\bf V}_{\ast\,i},
\end{align}
where
\begin{equation}
{\bf V}_e = \frac{{\bf E}\times {\bf B}}{B^2}
\end{equation}
is the {\em E-cross-B drift velocity}, and
\begin{align}
{\bf V}_{\ast\,e}&= \frac{\nabla p_e\times {\bf B}}{e\,n\,B^2},\\[0.5ex]
{\bf V}_{\ast\,i}&=- \frac{\nabla p_i\times {\bf B}}{e\,n\,B^2},
\end{align}
are termed the {\em electron diamagnetic velocity}\/ and the {\em ion diamagnetic velocity}, respectively. 

\section{Cylindrical Tearing Mode Theory}
\subsection{Tokamak Equilibrium}
Consider a large aspect-ratio, toroidal,  tokamak plasma equilibrium whose magnetic flux-surfaces map out (almost) concentric circles in the poloidal plane. 
Let us employ a conventional set of right-handed cylindrical coordinates, $r$, $\theta$, $z$. The
equilibrium magnetic flux-surfaces lie on surfaces of constant $r$. The system is
assumed to be periodic in the $z$ (`toroidal') direction, with periodicity length $2\pi\,R_0$, 
where $R_0$ is the simulated {\em major radius}\/ of the plasma. The {\em safety-factor}\/ profile takes the form
\begin{equation}\label{e90}
q(r)= \frac{r\,B_z}{R_0\,B_\theta(r)},
\end{equation}
where $B_z$ is the constant `toroidal' magnetic field-strength, and $B_\theta(r)$ is
the poloidal magnetic field-strength. It is assumed that $q\sim{\cal O}(1)$. 
The equilibrium `toroidal' current density
satisfies
\begin{equation}
\mu_0\,j_z(r) = \frac{(r\,B_\theta)'}{r},
\end{equation}
where $'$ denotes $d/dr$. Finally, the standard large aspect-ratio
tokamak orderings,  
\begin{equation}\label{e92}
\frac{r}{R_0}\ll 1,~~~ \frac{B_\theta}{B_z}\ll 1,
\end{equation}
are adopted (Fitzpatrick 1993). 

\subsection{Perturbed Magnetic Field}
Consider a tearing mode perturbation that has $m$ periods in the poloidal direction,
and $n$ periods in the `toroidal' direction, where $m/n\sim {\cal O}(1)$.  We can write the perturbed magnetic field in the form
\begin{equation}\label{e93}
\delta{\bf B} \simeq \nabla\times (\psi\,{\bf e}_z),
\end{equation}
where
\begin{equation}\label{e93a}
\psi(r,\theta,\phi,t)= \psi(r,t)\,\exp[{\rm i}\,(m\,\theta-n\,\phi)].
\end{equation}
Here, $\phi=z/R_0$ is a simulated toroidal angle. The perturbed current density becomes
\begin{equation}\label{e94}
\mu_0\,\delta {\bf j} \simeq - \nabla^2\psi\,{\bf e}_z.
\end{equation}
It is assumed that $|\delta{\bf B}|\ll |{\bf B}|$. 

\subsection{Cylindrical Tearing Mode Equation}
To lowest order, the tearing perturbation is governed by the linearized version of Equation~(\ref{e84}), which yields
\begin{equation}\label{e96}
(\delta{\bf B}\cdot \nabla)\,{\bf j} + ({\bf B}\cdot\nabla)\,\delta{\bf j} - (\delta{\bf j}\cdot\nabla)\,{\bf B} - 
({\bf j}\cdot\nabla)\,\delta {\bf B}\simeq {\bf 0}.
\end{equation}
Making use of Equations~(\ref{e90})--(\ref{e94}), the $z$-component of the previous equation gives
the {\em cylindrical tearing mode equation}\/ (Fitzpatrick 1993),
\begin{equation}
\frac{\partial^2\psi}{dr^2} + \frac{1}{r}\,\frac{\partial\psi}{\partial r}-\frac{m^2}{r^2}\,\psi - \frac{J_z'\,\psi}{r\,(1/q-n/m)}\simeq 0,
\end{equation}
where 
\begin{equation}
J_z(r)= \frac{R_0\,\mu_0\,j_z(r)}{B_z}.
\end{equation}

\subsection{Solution in Presence of Perfectly-Conducting Wall}\label{perfect}
Suppose that the plasma occupies the region $0\leq r\leq a$, where $a$ is the plasma {\em minor radius}. It follows that
$J_z(r)=0$ for $r>a$. Suppose that the plasma is surrounded by a concentric perfectly-conducting wall of radius $r_w>a$. An appropriate
physical solution of the cylindrical tearing mode equation takes the form
\begin{equation}\label{e99}
\psi(r,t)= {\mit\Psi}_s(t)\,\hat{\psi}_s(r),
\end{equation}
where the real function $\hat{\psi}_s(r)$ is a solution of 
\begin{equation}\label{e100}
\frac{d^2\hat{\psi}_s}{dr^2} + \frac{1}{r}\,\frac{d\hat{\psi}_s}{dr}-\frac{m^2}{r^2}\,\hat{\psi}_s - \frac{J_z'\,\hat{\psi}_s}{r\,(1/q-n/m)}\simeq 0
\end{equation}
that satisfies
\begin{align}\label{e101}
\hat{\psi}_s(0) &= 0,\\[0.5ex]
\hat{\psi}_s(r_s) &= 1,\label{e102}\\[0.5ex]
\hat{\psi}_s(r\geq r_w) &= 0.\label{e103}
\end{align}

Note that Equation~(\ref{e100}) is singular at the so-called {\em resonant}\/ magnetic flux-surface, radius $r=r_s$, at which 
\begin{equation}\label{e104}
q(r_s)= \frac{m}{n}.
\end{equation}
At the resonant surface, ${\bf k}\cdot{\bf B}=0$, where ${\bf k} = (0,\,m/r,\,-n/R_0)$ is the wavevector of the tearing perturbation. 
The value of $\psi(r,t)$ at the resonant surface---namely, ${\mit\Psi}_s(t)$---is known as the {\em reconnected magnetic flux}. 
Note that ${\mit\Psi}_s$ is, in general, a complex quantity. 
Let $\rho=(r-r_s)/r_s$. 
The solution of Equation~(\ref{e100}) in the vicinity of the resonant surface is
\begin{equation}\label{e105}
\hat{\psi}_s(\rho) = 1 + {\mit\Delta}_{s+}\,\rho + \alpha_s\,\rho\,\ln|\rho| + {\cal O}(\rho^2)
\end{equation}
for $\rho>0$, and
\begin{equation}\label{e106}
\hat{\psi}_s(\rho) = 1 + {\mit\Delta}_{s-}\,\rho + \alpha_s\,\rho\,\ln|\rho| + {\cal O}(\rho^2)
\end{equation}
for $\rho<0$. Here, 
\begin{align}
\alpha_s& =- \left(\frac{r\,q\,J_z'}{s}\right)_{r=r_s},\\[0.5ex]
s(r) &= \frac{r\,q'}{q}.
\end{align}
Moreover,  the real parameters ${\mit\Delta}_{s+}$ and  ${\mit\Delta}_{s-}$ are fully determined by Equation~(\ref{e100})
and the boundary conditions (\ref{e101})--(\ref{e103}). Note that, in general, $\psi$ is continuous across the resonant surface (in accordance with Maxwell's equations),
whereas $\partial\psi/\partial r$ is discontinuous. The discontinuity in $\partial \psi/\partial r$ implies the presence of a radially
thin, helical,  current sheet  at the resonant surface. This current sheet can only be resolved by retaining more terms in the perturbed plasma equation of motion, (\ref{e96}). 

The complex quantity
\begin{equation}
{\mit\Delta\Psi}_s = \left[r\,\frac{\partial \psi}{\partial r}\right]_{r_{s-}}^{r_{s+}}
\end{equation}
parameterizes the amplitude and phase of the current sheet flowing at the resonant surface. Matching the solutions
in the so-called {\em inner region}\/ (i.e., the region of the plasma in the immediate vicinity of the resonant surface)
and the so-called {\em outer region}\/ (i.e., everywhere in the plasma other than the inner region) with the help
of Equations~(\ref{e99}), (\ref{e105}), and ({\ref{e106}), we obtain
\begin{equation}
{\mit\Delta\Psi}_s = E_{ss}\,{\mit\Psi}_s,
\end{equation}
where $E_{ss}= {\mit\Delta}_{s+}-{\mit\Delta}_{s-}$ is a real quantity determined by the solution in the outer region. 

\subsection{Solution in Presence of Resistive Wall}\label{resistive}
Suppose, now, that the wall at $r=r_w$ possesses a finite resistivity, but is surrounded by a second perfectly-conducting wall
located at radius $r_c>r_w$.
The solution in the outer region can be written
\begin{equation}\label{e111}
\psi(r,t) = {\mit\Psi}_s(t)\,\hat{\psi}_s(r) + {\mit\Psi}_w(t)\,\hat{\psi}_w(r),
\end{equation}
where the real function $\hat{\psi}_s(r)$ is specified in the previous section, and the real function  $\hat{\psi}_w(r)$ is a solution of 
\begin{equation}\label{e100a}
\frac{d^2\hat{\psi}_w}{dr^2} + \frac{1}{r}\,\frac{d\hat{\psi}_w}{dr}-\frac{m^2}{r^2}\,\hat{\psi}_w - \frac{J_z'\,\hat{\psi}_w}{r\,(1/q-n/m)}\simeq 0
\end{equation}
that satisfies
\begin{align}\label{e101a}
\hat{\psi}_w(r\leq r_s) &= 0,\\[0.5ex]
\hat{\psi}_w(r_w) &= 1,\label{e102a}\\[0.5ex]
\hat{\psi}_w(r\geq r_c) &= 0.\label{e103a}
\end{align}
It is easily seen that
\begin{equation}\label{e117}
\hat{\psi}_w(r_w< r < r_c) = \frac{(r/r_c)^{-m} - (r/r_c)^m}{(r_w/r_c)^{-m} - (r_w/r_c)^m}.
\end{equation}

In general, $\psi$ is continuous across the wall (in accordance with Maxwell's equations), whereas $\partial\psi/\partial r$ is discontinuous. The discontinuity in $\partial\psi/\partial r$ is caused by the currents induced in the wall. The complex quantity ${\mit\Psi}_w(t)$ determines the amplitude and
phase of the perturbed magnetic flux that penetrates the wall. The complex quantity
\begin{equation}\label{e118}
{\mit\Delta\Psi}_w = \left[r\,\frac{\partial\psi}{\partial r}\right]_{r_{w-}}^{r_{w+}}
\end{equation}
parameterizes the amplitude and phase of the helical current sheet flowing in the wall.
Simultaneously matching the outer solution (\ref{e111}) across the resonant surface and the wall yields
\begin{align}
{\mit\Delta\Psi}_s &= E_{ss}\,{\mit\Psi}_s + E_{sw}\,{\mit\Psi}_w,\\[0.5ex]
{\mit\Delta\Psi}_w &= E_{ws}\,{\mit\Psi}_s+ E_{ww}\,{\mit\Psi}_w
\end{align}
(Fitzpatrick 1993).
Here,
\begin{align}
E_{ww}&= \left[r\,\frac{d\hat{\psi}_w}{dr}\right]_{r_{w-}}^{r_{w+}},\\[0.5ex]
E_{sw} &=\left[r\,\frac{{d\hat\psi}_w}{dr}\right]_{r=r_{s+}},\label{e121}\\[0.5ex]
E_{ws} &=-\left[r\,\frac{{d\hat\psi}_s}{dr}\right]_{r=r_{w-}}\label{e122}
\end{align}
are real quantities determined by the solution in the outer region.

Equations~(\ref{e100}) and (\ref{e100a}) can be combined to give
\begin{equation}
\frac{d}{dr}\!\left(\hat{\psi}_s\,r\,\frac{d\hat{\psi}_w}{dr} - \hat{\psi}_w\,r\,\frac{d\hat{\psi}_s}{dr}\right) = 0.
\end{equation}
If we integrate the previous equation from $r=r_{s+}$ to $r=r_{w-}$, making use of Equations~(\ref{e102}), (\ref{e103}), (\ref{e101a}), (\ref{e102a}), (\ref{e121}),
and (\ref{e122}), then we obtain
\begin{equation}\label{e123}
E_{sw} = E_{ws}
\end{equation}
(Fitzpatrick 1993). 

\subsection{Resistive Wall Physics}
It is clear from Equation~(\ref{e93}) that
\begin{equation}
\delta {\bf A} \simeq \psi\,{\bf e}_z,
\end{equation}
where $\delta {\bf A}$ is the perturbed magnetic vector potential. 
Hence, the perturbed electric field can be written
\begin{equation}
\delta {\bf E} \simeq -\nabla\delta{\mit\Phi} - \frac{\partial\psi}{\partial t}\,{\bf e}_z,
\end{equation}
where $\delta{\mit\Phi}$ is the perturbed electric scalar potential. 
Assuming that $|\nabla\delta{\mit\Phi}|\sim |\partial\psi/\partial t|$, the $z$-component of the previous
equation yields
\begin{equation}
\delta E_z\simeq -\frac{\partial\psi}{\partial t},
\end{equation}
where use has been made of the large aspect-ratio orderings (\ref{e92}). Within the wall, we
can write
\begin{equation}
\delta E_z \simeq -\frac{d{\mit\Psi}_w}{dt}.
\end{equation}
Here, we are making use of the so-called {\em thin wall approximation}, according to which $\psi$ is assumed to only vary weakly in $r$ across
the wall. Ohm's law implies that
\begin{equation}
\delta j_z = \frac{\delta E_z}{\eta_w} = -\frac{1}{\eta_w}\,\frac{d{\mit\Psi}_w}{dt}
\end{equation}
within the wall, where $\eta_w$ is the wall resistivity. Equations~(\ref{e94}) and (\ref{e118}) yield
\begin{equation}\label{e132}
{\mit\Delta\Psi}_w = -\mu_0\int_{r_{w-}}^{r_{w+}}r\,\delta j_z\,dr.
\end{equation}
Suppose that the radial thickness of the wall is $\delta_w\ll r_w$. The previous two equations give
\begin{equation}\label{e133}
{\mit\Delta\Psi}_w = \tau_w\,\frac{d{\mit\Psi}_w}{dt},
\end{equation}
where
\begin{equation}
\tau_w = \frac{\mu_0\,r_w\,\delta_w}{\eta_w}
\end{equation}
is the so-called {\em time constant}\/ of the wall (Fitzpatrick 1993). The thin wall approximation is
valid as long as
\begin{equation}
\frac{r_w}{\delta_w}\gg \tau_w\left|\frac{d\ln{\mit\Psi}_w}{dt}\right|.
\end{equation}

\subsection{Resonant Layer Physics}
By analogy with Equation~(\ref{e133}), we can write
\begin{equation}
{\mit\Delta\Psi}_s = \tau_s\left(\frac{d{\mit\Psi}_w}{dt} + {\rm i}\,\varpi_s\,{\mit\Psi}_s\right).
\end{equation}
Here, $\tau_s$ is the {\em reconnection time}\/ (i.e., the typical timescale on which magnetic reconnection takes place in the
resonant layer surrounding the resonant surface), whereas $\varpi_s$ is the so-called {\em natural frequency}. 
The natural frequency is the phase velocity of the tearing mode when the wall is perfectly conducting. The
natural frequency is non-zero because the reconnected flux in the resonant layer is convected by the local plasma
flow (Fitzpatrick 1993). 

\subsection{Solution in Presence of External Magnetic Field-Coil}
Suppose that the perfectly-conducting wall at $r=r_c$ is replaced by a magnetic field-coil that carries a
helical current possesing $m$ periods in the poloidal direction, and $n$ periods in the `toroidal' direction. Let the current density in the field-coil
take the form
\begin{equation}\label{e135}
\delta j_z = -\frac{I_c(t)}{r_c}\,\delta(r-r_c)\,{\rm e}^{\,{\rm i}\,(m\,\theta-n\,\phi)}.
\end{equation}
Here, the complex quantity $I_c(t)$ parameterizes the amplitude and phase of the helical current flowing in the field-coil. 

The solution in the outer region can be written
\begin{equation}
\psi(r,t) = {\mit\Psi}_s(t)\,\hat{\psi}_s(r) + {\mit\Psi}_w(t)\,\hat{\psi}_w(r) + {\mit\Psi}_c\,\hat{\psi}_c(r),
\end{equation}
where the real functions $\hat{\psi}_s(r)$ and $\hat{\psi}_w(r)$ are specified in Sections~\ref{perfect} and \ref{resistive}, respectively. 
Moreover, 
the real function  $\hat{\psi}_c(r)$ is a solution of 
\begin{equation}\label{e100b}
\frac{d^2\hat{\psi}_c}{dr^2} + \frac{1}{r}\,\frac{d\hat{\psi}_c}{dr}-\frac{m^2}{r^2}\,\hat{\psi}_c\simeq 0
\end{equation}
that satisfies
\begin{align}
\hat{\psi}_c(r\leq r_w) &= 0,\\[0.5ex]
\hat{\psi}_c(r_c) &= 1,\\[0.5ex]
\hat{\psi}_c(\infty) &= 0.
\end{align}
It is easily seen that
\begin{align}
\hat{\psi}_c(r\leq r_w) &= 0,\\[0.5ex]
\hat{\psi}_c(r_w<r \leq r_c) &= \frac{(r/r_w)^m-(r/r_w)^{-m}}{(r_c/r_w)^m - (r_c/r_w)^{-m}},\label{e142}\\[0.5ex]
\hat{\psi}_c(r> r_c) &=\left(\frac{r}{r_c}\right)^{-m}.\label{e143}
\end{align}

In general, $\psi$ is continuous across the field-coil (in accordance with Maxwell's equations), whereas $\partial\psi/\partial r$ is discontinuous. The discontinuity in $\partial\psi/\partial r$ is caused by the currents flowing in the coil. The complex quantity ${\mit\Psi}_c(t)$ determines the amplitude and
phase of the perturbed magnetic flux at the coil. The complex quantity
\begin{equation}\label{e118a}
{\mit\Delta\Psi}_c = \left[r\,\frac{\partial\psi}{\partial r}\right]_{r_{c-}}^{r_{c+}}
\end{equation}
parameterizes the amplitude and phase of the current sheet flowing in the coil.
It follows from Equations~(\ref{e93a}), (\ref{e94}), and (\ref{e135})  that
\begin{equation}
{\mit\Delta\Psi}_c = \mu_0\,I_c.
\end{equation}

Simultaneously matching the outer solution across the resonant surface, the wall, and the field-coil, we
obtain
\begin{align}\label{e146}
{\mit\Delta\Psi}_s &= E_{ss}\,{\mit\Psi}_s + E_{sw}\,{\mit\Psi}_w,\\[0.5ex]
{\mit\Delta\Psi}_w &= E_{ws}\,{\mit\Psi}_s + E_{ww}\,{\mit\Psi}_w + E_{wc}\,{\mit\Psi}_c,\\[0.5ex]
{\mit\Delta\Psi}_c&= E_{cw}\,{\mit\Psi}_w + E_{cc}\,{\mit\Psi}_c. \label{e148}
\end{align}
Here,  
\begin{align}
E_{cc} &= \left[r\,\frac{d\hat{\psi}_c}{dr}\right]_{r_{c-}}^{r_{c+}}=- \frac{2m}{1-(r_w/r_c)^{2m}},\\[0.5ex]
E_{wc}&= \left[r\,\frac{d\hat{\psi}_c}{dr}\right]_{r_{w+}} = \frac{2m\,(r_w/r_c)^m}{1-(r_w/r_c)^{2m}},\label{e150a}\\[0.5ex]
E_{cw}&=- \left[r\,\frac{d\hat{\psi}_w}{dr}\right]_{r_{c-}} = \frac{2m\,(r_w/r_c)^m}{1-(r_w/r_c)^{2m}},\label{e151}
\end{align}
where use has been made of Equations~(\ref{e117}), (\ref{e142}), and (\ref{e143}). 

\subsection{Electromagnetic Torques}\label{storque}
The flux-surface integrated poloidal and `toroidal' electromagnetic torque densities acting on the plasma
can be written
\begin{align}\label{e152}
T_\theta(r) &= \left\langle r\,{\bf j}\times {\bf B}\cdot {\bf e}_\theta\right\rangle\\[0.5ex]
T_\phi(r) &= \left\langle R_0\,{\bf j}\times {\bf B}\cdot {\bf e}_z\right\rangle,
\end{align}
respectively, 
where
\begin{equation}
\left\langle\cdots\right\rangle \equiv \oint\oint r\,R_0\,(\cdots)\,d\theta\,d\phi
\end{equation}
is the flux-surface average operator. 
 However, according to Equation~(\ref{e80}),
\begin{equation}\label{e150}
{\bf j}\times {\bf B} \simeq \nabla p.
\end{equation}
Given that the scalar pressure is a single-valued function of $\theta$ and $\phi$, it immediately follows
that $T_\theta=T_\phi = 0$ throughout the plasma. The only exception to this rule occurs in the immediate vicinity of the
resonant surface, where Equation~(\ref{e150}) breaks down. It follows that we can write
\begin{align}\label{e156a}
T_\theta(r) &= T_{\theta\,s}\,\delta(r-r_s),\\[0.5ex]
T_\phi(r) &= T_{\phi\,s}\,\delta(r-r_s),
\end{align}
where
\begin{align}\label{e153}
T_{\theta\,s} &=  \frac{1}{4}\int_{r_{s-}}^{r_{s+}}\oint\oint R_0\,r^{\,2}\,(\delta j_z\,\delta B_r^{\,\ast} + \delta j_z^{\,\ast}\,\delta B_r)\,dr\,d\theta\,d\phi,\\[0.5ex]
T_{\phi\,s} &=  -\frac{1}{4}\int_{r_{s-}}^{r_{s+}}\oint\oint R_0^{\,2}\,r\,(\delta j_\theta\,\delta B_r^{\,\ast} + \delta j_\theta^{\,\ast}\,\delta B_r)\,dr\,d\theta\,d\phi
\end{align}
are the net poloidal and `toroidal' torques, respectively, acting at the resonant surface. 
Note that the equilibrium torques are zero because $B_r=0$. Furthermore, the linear (in perturbed quantities) torques 
average to zero over the flux-surface. Hence, the largest non-zero torques  are quadratic in perturbed quantities. 
According to Equation~(\ref{e82}), 
\begin{equation}
\nabla \cdot \delta {\bf j} = {\rm i}\left(\frac{m}{r}\,\delta j_\theta - \frac{n}{R_0}\,\delta j_z\right) = 0,
\end{equation}
where $\delta{\bf j}$ is the density of the current sheet flowing at the resonant surface, and we have made use
of the fact that $\delta j_r\simeq 0$. 
 It follows from the previous three equations that
\begin{equation}\label{e156}
T_{\phi\,s} = - \frac{n}{m}\,T_{\theta\,s}.
\end{equation}
Finally, Equations~(\ref{e93}), (\ref{e94}), (\ref{e153}), and (\ref{e156}) imply that
\begin{align}\label{e157}
T_{\theta\,s} &= -\frac{2\pi^2\,R_0\,m}{\mu_0}\,{\rm Im}({\mit\Delta\Psi}_s\,{\mit\Psi}_s^{\,\ast}),\\[0.5ex]
T_{\phi\,s} &= \frac{2\pi^2\,R_0\,n}{\mu_0}\,{\rm Im}({\mit\Delta\Psi}_s\,{\mit\Psi}_s^{\,\ast})
\end{align}
(Fitzpatrick 1993).

By analogy with the previous analysis,  the net poloidal and `toroidal' electromagnetic torques acting on the
resistive wall are 
\begin{align}
T_{\theta\,w} &= -\frac{2\pi^2\,R_0\,m}{\mu_0}\,{\rm Im}({\mit\Delta\Psi}_w\,{\mit\Psi}_w^{\,\ast}),\\[0.5ex]
T_{\phi\,w} &= \frac{2\pi^2\,R_0\,n}{\mu_0}\,{\rm Im}({\mit\Delta\Psi}_w\,{\mit\Psi}_w^{\,\ast}),
\end{align}
respectively. 
Finally, 
the net poloidal and `toroidal' electromagnetic torques acting on the
magnetic field-coil are 
\begin{align}
T_{\theta\,c} &= -\frac{2\pi^2\,R_0\,m}{\mu_0}\,{\rm Im}({\mit\Delta\Psi}_c\,{\mit\Psi}_c^{\,\ast}),\\[0.5ex]
T_{\phi\,c} &= \frac{2\pi^2\,R_0\,n}{\mu_0}\,{\rm Im}({\mit\Delta\Psi}_c\,{\mit\Psi}_c^{\,\ast}),\label{e166}
\end{align}
respectively. 

It follows from Equations~(\ref{e146})--(\ref{e148}) that
\begin{align}
{\rm Im}({\mit\Delta\Psi}_s\,{\mit\Psi}_s^{\,\ast}) &= E_{sw}\,{\rm Im}({\mit\Psi}_w\,{\mit\Psi}_s^{\,\ast}),\\[0.5ex]
{\rm Im}({\mit\Delta\Psi}_w\,{\mit\Psi}_w^{\,\ast}) &= E_{ws}\,{\rm Im}({\mit\Psi}_s\,{\mit\Psi}_w^{\,\ast})+E_{wc}\,{\rm Im}({\mit\Psi}_c\,{\mit\Psi}_w^{\,\ast}),\\[0.5ex]
 {\rm Im}({\mit\Delta\Psi}_c\,{\mit\Psi}_c^{\,\ast}) &=E_{cw}\,{\rm Im}({\mit\Psi}_w\,{\mit\Psi}_c^{\,\ast}).
 \end{align}
 Thus,
 \begin{align}
 {\rm Im}({\mit\Delta\Psi}_s\,{\mit\Psi}_s^{\,\ast})+ {\rm Im}({\mit\Delta\Psi}_w\,{\mit\Psi}_w^{\,\ast}) + {\rm Im}({\mit\Delta\Psi}_c\,{\mit\Psi}_c^{\,\ast}) & = (E_{sw}-E_{ws})\,{\rm Im}({\mit\Psi}_w\,{\mit\Psi}_s^{\,\ast}) \nonumber\\[0.5ex]
 &\phantom{=}+ (E_{wc}-E_{cw})\,{\rm Im}({\mit\Psi}_c\,{\mit\Psi}_w^{\,\ast}).
 \end{align}
 However, according to Equations~(\ref{e123}), (\ref{e150a}), and (\ref{e151}), $E_{sw}=E_{ws}$ and $E_{wc}=E_{cw}$. 
We deduce that
 \begin{equation} 
  {\rm Im}({\mit\Delta\Psi}_s\,{\mit\Psi}_s^{\,\ast})+ {\rm Im}({\mit\Delta\Psi}_w\,{\mit\Psi}_w^{\,\ast}) + {\rm Im}({\mit\Delta\Psi}_c\,{\mit\Psi}_c^{\,\ast})=0.
  \end{equation}
Hence, Equations~(\ref{e157})--(\ref{e166})   yield
\begin{align}
T_{\theta\,s} + T_{\theta\,w}+T_{\theta\,c} &= 0,\\[0.5ex]
T_{\phi\,s} + T_{\phi\,w}+T_{\phi\,c} &= 0.
\end{align}
In other words, the plasma/wall/field-coil system exerts zero net poloidal electromagnetic torque, and zero net toroidal electromagnetic
torque, on itself. 

\subsection{Plasma Angular Equations of Motion}
The sum of the electron fluid equation of motion, (\ref{e3.151b}), and the ion fluid equation of motion, (\ref{e3.152b}), yields the plasma equation of motion, 
\begin{equation}
\rho\,\frac{\partial  {\bf V}_i}{\partial t} +
\rho\,({\bf V}_i\cdot\nabla) {\bf V}_i +
 [\delta_i^{-2}]\,\nabla p + 
[\zeta_i^{-1}]\,\nabla\cdot \bpi_i-[\delta_i^{-2}]\,{\bf j}\times {\bf B} =0,\label{e175}
\end{equation}
where $\rho=m_i\,n$ is the plasma mass density.
Here, we have neglected electron inertial terms with respect to ion inertial terms (because the
former are order $m_e/m_i$ smaller than the latter). We have also neglected electron viscosity terms
with respect to ion viscosity terms (because the
former are order $\sqrt{m_e/m_i}$ smaller than the latter). Recall that the terms in square brackets are to remind us that the terms they precede are larger than the
other terms in the equations (by the corresponding terms within the brackets). 

Taking the flux-surface average of the scalar product of $r\,{\bf e}_\theta$ with the plasma equation of motion, (\ref{e175}), we obtain 
\begin{equation}
\left\langle\rho\,\frac{\partial (r\,V_{\theta\,i})}{\partial t} \right\rangle 
+\left\langle \rho\left[V_{r\,i}\,\frac{\partial (r\,V_{\theta\,i})}{\partial r} + V_{z\,i}\,\frac{\partial(r\, V_{\theta\,i})}{\partial z}
\right]\right\rangle + [\zeta_i^{-1}]\,\langle r\,\nabla\cdot \bpi_i\cdot{\bf e}_\theta\rangle
= [\delta_i^{-2}]\,T_{\theta\,s}\,\delta(r-r_s),
\end{equation}
where use has been made of Equations~(\ref{e152}) and (\ref{e156a}). Note that the dominant $\nabla p$ term
has averaged to zero. Moreover, because $T_{\theta\,s}$ is quadratic in perturbed quantities (see Section~\ref{storque}), it is plausible that the right-hand side of the previous equation is not a dominant term [i.e., it is not ${\cal O}(\delta_i^{-2})$ larger than the
other terms in the equation]. 


We can write
\begin{align}
{\mit\Omega}_\theta(r,t)&= {\mit\Omega}_{\theta\,0}(r) + {\mit\Delta\Omega}_\theta(r,t),\\[0.5ex]
{\mit\Omega}_\phi(r,t)&= {\mit\Omega}_{\phi\,0}(r) + {\mit\Delta\Omega}_\phi(r,t),
\end{align}
where ${\mit\Omega}_\theta(r,t)$ and ${\mit\Omega}_\phi(r,t)$ are the poloidal and 	`toroidal' plasma angular velocity profiles, respectively. [To be more exact,  ${\mit\Omega}_\theta(r,t)$ and ${\mit\Omega}_\phi(r,t)$ are the ion fluid angular velocity
profiles.] Moreover, ${\mit\Omega}_{\theta\,0}(r)$ and ${\mit\Omega}_{\phi\,0}(r)$ are the corresponding unperturbed profiles,
and ${\mit\Delta\Omega}_{\theta}(r,t)$ and ${\mit\Delta\Omega}_{\phi}(r,t)$ are the respective modifications to the profiles
induced by the electromagnetic torques that develop at the resonant surface. The modifications to the angular
velocity profiles are governed by the poloidal and toroidal angular equations of motion of the plasma, which
take the respective forms
\begin{align}
4\pi^2\,R_0\left[(1+2\,q^2)\,\rho\,r^{\,3}\,\frac{\partial{\mit\Delta\Omega}_\theta}{\partial t}-\frac{\partial}{\partial r}\!\left(
\mu\,r^{\,3}\,\frac{\partial{\mit\Delta\Omega}_\theta}{\partial r}\right) + \frac{\rho}{\tau_\theta}\,r^{\,3}\,{\mit\Delta\Omega}_\theta\right] = T_{\theta\,s}\,\delta(r-r_s),\\[0.5ex]
4\pi^2\,R_0^{\,3}\left[\rho\,r\,\frac{\partial{\mit\Delta\Omega}_\phi}{\partial t}-\frac{\partial}{\partial r}\!\left(
\mu\,r\,\frac{\partial{\mit\Delta\Omega}_\phi}{\partial r}\right) \right] = T_{\phi\,s}\,\delta(r-r_s).
\end{align}
Here, $\rho(r)$ is the plasma mass density profile, $\mu(r)$ is the anomalous plasma perpendicular ion viscosity
(due to plasma turbulence), and $\tau_\theta(r)$ is the characteristic time at which neoclassical
parallel viscosity relaxes the plasma poloidal angular velocity profile. 
  
\section*{Bibliography}
\begin{description}
\item Braginskii, S.I.\ 1965. {\em Transport Processes in a Plasma}. In {\em Reviews of Plasma Physics}. Consultants Bureau. Volume~1, 205.
\item Chapman, S., and Cowling, T.G.\ 1953. {\em The Mathematical Theory of Non-Uniform Gases}. Cambridge.
\item Fitzpatrick, R.\ 1993. {\em Interaction of Tearing Modes with External Structures in Cylindrical Geometry}, Nucl.\ Fusion {\bf 33}, 1049 (1993). 
\item Fitzpatrick, R.\ 2008. {\em Maxwell's Equations and the Principles of Electromagnetism}. Jones \& Bartlett.
\item Fitzpatrick, R.\ 2015. {\em Plasma Physics: An Introduction}. Taylor \& Francis, CRC Press. 
\item Reif, F.\ 1965. {\em Fundamentals of Statistical and Thermal Physics}. McGraw-Hill.
\item Richardson, A.S.\ 2019. {\em 2019 NRL Plasma Formulary}.  Naval Research Laboratory.
\end{description}

\end{document}