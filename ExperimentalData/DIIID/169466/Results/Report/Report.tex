\documentclass[12pt]{article}
\usepackage{amsmath}
\usepackage{fullpage}
\usepackage{graphicx}
\allowdisplaybreaks

\title{\bf DIII-D Discharge 169466}
\date{}
\author{}

\begin{document}
\maketitle

\section{Llama Data}
Figure~\ref{fig1} shows the following fit to the Llama data:
\begin{equation}
\langle n_n\rangle(R) = \frac{n_n}{1+(R-R_{\rm sep})^{\,2}/l_n^{\,2}},
\end{equation}
where $n_n = 1.4\times 10^{17}\,{\rm m}^{-3}$, $l_n= 0.015\,{\rm m}$, $R_{\rm sep}= 1.92\,{\rm m}$.  The fit is not perfect. However, given
the uncertainties in quantities such as the charge exchange cross-section and the poloidal asymmetry of the neutral distribution, it is probably adequate. The separatrix is identified with the peak in the neutral distribution. Another important neutral
distribution parameter is the poloidal asymmetry parameter:
\begin{equation}
y_n = \frac{\langle n_n\rangle\langle B^{\,2}\rangle}{\langle n_n\,B^{\,2}\rangle}.
\end{equation}

\section{Natural Frequencies}
Figures~\ref{fig2} and \ref{fig3} show the natural frequencies at the $n=3$ resonant surfaces for the baseline scenario
at times $t=3200$ and 3600 ms, respectively. It can be seen that the zero crossing is pushed to lower ${\mit\Psi}_N$ values the later time. 
However, the natural frequency at the 9/3 surface is actually smaller at the later time, which is not consistent with the later time
corresponding to lack of ELM suppression.

Figures~\ref{fig4} and \ref{fig5} show what happens to the baseline scenario when the neutral density is increased by a factor of
10. It can be seen that the nonlinear natural frequency is pulled strongly in the electron direction.

Figures~\ref{fig6} and \ref{fig7} show what happens to the baseline scenario when the neutral density is decreased by a factor of
10. It can be seen that the nonlinear natural frequency is pulled somewhat in the ion direction.

Figures~\ref{fig8} and \ref{fig9} show what happens to the baseline scenario when there is no poloidal asymmetry in the neutral distribution. It can be seen that the nonlinear natural frequency is pulled somewhat in the ion direction. In this case, the natural frequency at the 10/3 surface is smaller at the earlier time, which is consistent with the earlier time
corresponding to  ELM suppression.

\section{Conclusions}
The only clear trend is that the zero crossing moves inwards in ${\mit\Psi}_N$ at the later time. However, it is not clear that this
accounts for the lack of ELM suppression at the later time. 

\begin{figure}
\centering
\includegraphics[height=4in]{neutral.eps}
\caption{Fit to Llama data. Black: Data points. Red: Fit.}\label{fig1}
\end{figure}



\begin{figure}
\centering
\includegraphics[height=4in]{3200_sim1.eps}
\caption{Natural frequencies at $t=3200$ ms. Neutral parameters: $n_n= 1.4\times 10^{17}\,{\rm m}^{-3}$, $l_n= 0.015\,{\rm m}$, $y_n=1.5$. Red: Linear natural frequency. Blue: Nonlinear natural frequency. Black: ${\bf E}\times {\bf B}$ frequency. The points correspond to the
6/3, 7/3, 8/3, 9/3, 10/3, 11/3, 12/3, 13/3, and 14/3 rational surfaces in order from the left to the right.}\label{fig2}
\end{figure}

\begin{figure}
\centering
\includegraphics[height=4in]{3600_sim1.eps}
\caption{Natural frequencies at $t=3600$ ms. Neutral parameters: $n_n= 1.4\times 10^{17}\,{\rm m}^{-3}$, $l_n= 0.015\,{\rm m}$, $y_n=1.5$. Red: Linear natural frequency. Blue: Nonlinear natural frequency. Black: ${\bf E}\times {\bf B}$ frequency. The points correspond to the
6/3, 7/3, 8/3, 9/3, 10/3, 11/3, 12/3, 13/3, and 14/3 rational surfaces in order from the left to the right.}\label{fig3}
\end{figure}

\begin{figure}
\centering
\includegraphics[height=4in]{3200_sim2.eps}
\caption{Natural frequencies at $t=3200$ ms. Neutral parameters: $n_n= 1.4\times 10^{18}\,{\rm m}^{-3}$, $l_n= 0.015\,{\rm m}$, $y_n=1.5$. Red: Linear natural frequency. Blue: Nonlinear natural frequency. Black: ${\bf E}\times {\bf B}$ frequency. The points correspond to the
6/3, 7/3, 8/3, 9/3, 10/3, 11/3, 12/3, 13/3, and 14/3 rational surfaces in order from the left to the right.}\label{fig4}
\end{figure}

\begin{figure}
\centering
\includegraphics[height=4in]{3600_sim2.eps}
\caption{Natural frequencies at $t=3600$ ms. Neutral parameters: $n_n= 1.4\times 10^{18}\,{\rm m}^{-3}$, $l_n= 0.015\,{\rm m}$, $y_n=1.5$. Red: Linear natural frequency. Blue: Nonlinear natural frequency. Black: ${\bf E}\times {\bf B}$ frequency. The points correspond to the
6/3, 7/3, 8/3, 9/3, 10/3, 11/3, 12/3, 13/3, and 14/3 rational surfaces in order from the left to the right.}\label{fig5}
\end{figure}

\begin{figure}
\centering
\includegraphics[height=4in]{3200_sim3.eps}
\caption{Natural frequencies at $t=3200$ ms. Neutral parameters: $n_n= 1.4\times 10^{16}\,{\rm m}^{-3}$, $l_n= 0.015\,{\rm m}$, $y_n=1.5$. Red: Linear natural frequency. Blue: Nonlinear natural frequency. Black: ${\bf E}\times {\bf B}$ frequency. The points correspond to the
6/3, 7/3, 8/3, 9/3, 10/3, 11/3, 12/3, 13/3, and 14/3 rational surfaces in order from the left to the right.}\label{fig6}
\end{figure}

\begin{figure}
\centering
\includegraphics[height=4in]{3600_sim3.eps}
\caption{Natural frequencies at $t=3600$ ms. Neutral parameters: $n_n= 1.4\times 10^{16}\,{\rm m}^{-3}$, $l_n= 0.015\,{\rm m}$, $y_n=1.5$. Red: Linear natural frequency. Blue: Nonlinear natural frequency. Black: ${\bf E}\times {\bf B}$ frequency. The points correspond to the
6/3, 7/3, 8/3, 9/3, 10/3, 11/3, 12/3, 13/3, and 14/3 rational surfaces in order from the left to the right.}\label{fig7}
\end{figure}

\begin{figure}
\centering
\includegraphics[height=4in]{3200_sim4.eps}
\caption{Natural frequencies at $t=3200$ ms. Neutral parameters: $n_n= 1.4\times 10^{17}\,{\rm m}^{-3}$, $l_n= 0.015\,{\rm m}$, $y_n=1.0$. Red: Linear natural frequency. Blue: Nonlinear natural frequency. Black: ${\bf E}\times {\bf B}$ frequency. The points correspond to the
6/3, 7/3, 8/3, 9/3, 10/3, 11/3, 12/3, 13/3, and 14/3 rational surfaces in order from the left to the right.}\label{fig8}
\end{figure}

\begin{figure}
\centering
\includegraphics[height=4in]{3600_sim4.eps}
\caption{Natural frequencies at $t=3600$ ms. Neutral parameters: $n_n= 1.4\times 10^{17}\,{\rm m}^{-3}$, $l_n= 0.015\,{\rm m}$, $y_n=1.0$. Red: Linear natural frequency. Blue: Nonlinear natural frequency. Black: ${\bf E}\times {\bf B}$ frequency. The points correspond to the
6/3, 7/3, 8/3, 9/3, 10/3, 11/3, 12/3, 13/3, and 14/3 rational surfaces in order from the left to the right.}\label{fig9}
\end{figure}





\end{document}
