\documentclass{article}[12pt]
\usepackage{fullpage}

\begin{document}
\begin{center}
{\em Modeling $q_{95}$ Windows for the Suppression of Edge Localized Modes by Resonant Magnetic}\\[0.5ex]
{\em Perturbations in the DIII-D tokamak}\\[1ex]
by Richard Fitzpatrick\\[1ex]
{\bf Reply to Referees' Comments}
\end{center}

First, let me thanks both referees for their careful reading of the paper, and their helpful comments. In response to these 
comments, I have made the following major changes to the paper.
\begin{enumerate}
\item I have taken the material in the Appendix relating to the natural frequency, and have converted it into a
preliminary section in the main text. The remainder of the Appendix has been discarded, because it was
not germane to the argument of the paper. The new section allows me to state the first and second aims of
the paper right next to one another (at the end of the Introduction), and also helps to frame the discussion of the three calculations 
that are described in the paper. 
\item I have improved the figures, in that  I have  make the axes labels larger, and
have added legends to Figs.~4--13. 
\item I have added the neutral particle profile to Fig.~3.
\item I have added a summary figure (Fig.~14) that helps to compare and
contrast the results of the three calculations that are described in the paper.
\item The new summary figure makes it apparent that the results of the calculation with the linear natural frequency are
not as bad as I had originally thought. Hence, I have amended the discussion in the paper relating to this calculation.
\item I have tried to make it clear throughout the paper that, in the absence of the RMP,  the  {\tt EPEC} code uses profile information to reconstruct the natural frequencies
of tearing modes that are unmodified by RMP-induced electromagnetic torques in the plasma. In the presence of the RMP, the {\tt EPEC} code
calculates how these frequencies are modified by the torques. 
 Because it is clear, from a comparison of {\tt EPEC} simulations made with and without the RMP, that
the natural frequencies are not appreciably modified by the torques, except in those relatively short time intervals
in which a mode locks to the RMP, it is not unreasonable to use the profiles outside these intervals as inputs to the code. 
\end{enumerate}

The following gives responses to specific queries from Referee 1 that are not covered by the previous discussion (all reference numbers refer to references in the
original manuscript):
\begin{enumerate}
\item p.~3: I have made it clear that the maximum practical currents that can be driven in RMP coils render
the coils incapable of locking tearing modes except when the natural frequencies of the modes are
comparatively small (i.e., 5 krad/s, instead of the usual 100 krad/s, in DIII-D).
\item p.~4: I have replaced references to the ``guiding-center fluid" by references to the ${\bf E}\times {\bf B}$ velocity.
\item p.~4: I have add the Lyons (2017) reference. 
\item p.~5: I have added more asymptotic matching references. 
\item p.~5: I have made it clear that the observation that the driven widths of non-locked magnetic
island chains exceed the linear layer widths actually applies to DIII-D discharge 158115, and have argued
that it is not unreasonable to assume that the same holds true in discharge 145380. I have also explained
why I think that this state of affairs will persist in ITER.
\item p.~5: I did mean Ref.~12 instead of 14.
\item  p.~7: If the density pump-out modifies profiles in a manner that is not associated with the generation of
electromagnetic torques at the various rational surfaces in the plasma then the pump-out modified 
profiles would be the appropriate profiles to use as the starting point of the calculation. 
\item p.~7: {\tt EPEC} does not modify the plasma equilibrium. 
\item p.~8: I have replaced ``actual widths" by ``shielded widths". 
\item p.~9: The 1/3 power is correct in Ref.~22. If you read the final section of Ref.~49 you will see that
when you take into account the fact that parallel transport in high-temperature tokamak plasmas is
convective, rather than conductive, then the critical island width scales as $\chi_\perp^{1/3}$, rather than
$\chi_\perp^{1/4}$. I have added an additional reference to make this clear. I have also made it clear that the
generic $\chi_\perp$ used in Ref.~22 has been replaced by separate diffusivities for momentum, energy, and particles
in the present paper. 
\item p.~10: Because the $m=11$ ELM suppression window only differs from the other two windows by the presence of
a single ELM in the middle of the window, I do not think that it makes sense to make a significant distinction between the $m=11$
window and the other two. 
\item p.~12: Unfortunately, I do not know how to run the {\tt EPED} model. 
\item Fig.~1: The thin grey line seems to be a ``feature'' of the interaction of obsolete versions of Powerpoint and the Mac operating system.
Because the line does not actually occur in the figure, I cannot delete it. (I also cannot redraw the figure, because I did not generate it in the first place. I think that it is a pretty old figure.) I can only hope that the line does not appear in the published version of the paper. 
\end{enumerate}

The following gives responses to specific queries from Referee 2 that are not covered by the previous discussion:
\begin{enumerate}
\item {\tt EPEC}, like {\tt GPEC}, uses a flux-coordinate system, and is essentially spectral in the poloidal and toroidal directions. Thus,
{\tt EPEC} cannot treat an X-point exactly, but it can get asymptotically close to an X-point by making the cutoff surface arbitrarily close to
the LCFS. 
\end{enumerate}
\end{document}