\documentclass[12pt,prb,aps]{revtex4-1}
\usepackage{amsmath}
\pdfoutput=1 
\usepackage{graphicx}
\newcommand {\bomega}{\mbox{\boldmath$\omega$}}
\newcommand {\bpi}{\mbox{\boldmath$\pi$}}

\begin{document}

\title {Modelling $q_{95}$ Windows for the Suppression of Edge Localized Modes by Resonant Magnetic Perturbations in the DIII-D Tokamak}

\author{R.~Fitzpatrick}
\affiliation{Institute for Fusion Studies,  Department of Physics,  University of Texas at Austin,  Austin TX, 78712, USA}
 
\maketitle

\section{Introduction}
Tokamak discharges operating in high-confinement mode (H-mode)\,\cite{wagner} exhibit intermittent bursts of heat and particle transport, 
emanating from the outer regions of the plasma, that are known as ``type-I edge localized modes'' (ELMs). \cite{zohm}
It is estimated that the heat load that ELMs
will deliver to the tungsten plasma-facing components in a reactor-scale tokamak, such as ITER, will be large enough to cause
massive tungsten ion influx into the plasma core, and that the erosion associated with this process will 
unacceptably limit the lifetimes of these components. \cite{loarte} Consequently, the development of robust and effective
methods for ELM control is a high priority for the international magnetic fusion program. 

The most promising method for the control of ELMs in H-mode tokamak discharges is via the application of static   ``resonant magnetic perturbations'' (RMPs). Complete RMP-induced ELM suppression was first demonstrated on the DIII-D tokamak.\cite{evans} Subsequently, either mitigation or compete suppression of
ELMs has been demonstrated on the JET,\cite{jet} ASDEX-U,\cite{asdex} KSTAR,\cite{kstar} MAST\,\cite{mast}, and EAST\,\cite{east} tokamaks.

The application of a static RMP, resonant in the pedestal region (i.e., the region of strong pressure and current density gradients characteristic of the edge region of an H-mode tokamak discharge), to an H-mode tokamak discharge is observed to give rise to  two distinct phenomena.\cite{schmitz, lanctot,paz1,d158115,paz} The first of these  is the so-called ``density pump-out'', which  is characterized by a reduction in the electron number density
in the pedestal region that varies smoothly with the amplitude of the applied RMP,  is (usually) accompanied by a similar, but significantly smaller, reduction
in the electron and ion temperatures, but  is not associated with ELM suppression. The second phenomenon is  ``ELM suppression'' itself, which 
 occurs when the amplitude of the applied RMP exceeds a certain threshold value. 
ELM suppression  is only observed to take place when $q_{95}$ (i.e., the safety-factor on the magnetic flux-surface that encloses 95\% of the poloidal flux enclosed by
the last closed flux-surface) takes values that lie in certain narrow windows. \cite{paz1,d158115}

Numerical simulations made using the cylindrical,  nonlinear, two-fluid, reduced-magneto\-hydro\-dynamical (MHD), initial-value code, TM1\,\cite{tm1,tm2,tm3} have shed considerable light on the hitherto poorly understood  physical mechanism that underlies RMP-induced ELM
suppression in H-mode tokamak discharges.\cite{hu} The simulations  in question make a plausible case  that the density  pump-out  phenomenon is associated with the formation of   locked (i.e., non-rotating) helical magnetic
island chains at the bottom of the pedestal, whereas the ELM suppression phenomenon is associated with the formation of a locked helical magnetic island chain at the
top of the pedestal. The prevailing hypothesis is that such an island chain suppresses ELMs by limiting the expansion of the
pedestal, and, thereby, preventing it from attaining a width sufficient to destabilize peeling-ballooning modes\,\cite{connor}  (which are thought to trigger ELMs).\cite{d3d}

Recently, a toroidal generalization of the 
cylindrical asymptotic matching model presented
in Ref.~\onlinecite{rf1} was formulated and
used to model  RMP-induced ELM suppression experiments performed on the DIII-D tokamak, \cite{rf2}
leading to similar conclusions to the aforementioned TM1
studies. The primary aim of this paper is to employ this
new model to try to account for the  $q_{95}$ ELM
suppression windows  that are apparent when the edge safety-factor is slowly ramped in
a particular DIII-D discharge in which an $n=3$
RMP is used to control ELMs.\cite{d3d,d3d2} 

It is well-known that magnetic reconnection caused when a  (stable) tearing mode is driven by a static RMP that is resonant at a particular magnetic flux-surface in a tokamak plasma is
facilitated when the associated ``natural frequency'' is relatively small.\cite{rfa,rfb}  The natural frequency  of a  (stable) tearing mode is the
helical phase velocity that the mode would possess were it naturally unstable (in the absence of the RMP).  Driven magnetic reconnection leads to the
formation of a locked magnetic island chain at the resonant surface in question. 
Hence, the prevailing hypothesis is that a $q_{95}$ window for
RMP-induced ELM suppression occurs when $q_{95}$ is such that the natural frequency of a tearing mode resonant  at 
the top of the pedestal is close to zero. 
Unfortunately, there is currently some uncertainty in the fusion community regarding the form of the natural frequency. 

According to
linear tearing mode theory, a tearing mode is essentially convected by
the local electron fluid at the resonant surface (see the discussion at the end of the Appendix).\cite{hender,cole} Hence, if the response of an H-mode tokamak plasma to an applied RMP
is governed by linear physics then we would expect the
natural frequency to take the form\,\cite{lin1,lin2,lin3}
\begin{equation}
\varpi_{\rm linear} = - n\,(\omega_E+\omega_{\ast\,e}),\label{e1}
\end{equation}
where
\begin{align}
\omega_E &= -\frac{d{\mit\Phi}}{d{\mit\Psi}_p},\\[0.5ex]
\omega_{\ast\,e} &=\frac{T_e}{e}\,\frac{d\ln p_e}{d{\mit\Psi}_p}.
\end{align}
Here, ${\mit\Psi}_p$ is the equilibrium poloidal magnetic
flux (divided by $2\pi$), ${\mit\Phi}({\mit\Psi}_p)$ the equilibrium scalar
electric potential, $p_e({\mit\Psi}_p)$ the equilibrium
electron pressure, $T_e({\mit\Psi}_p)$ the equilibrium
electron temperature, $e$ the magnitude of the electron
charge, and $n$ the toroidal mode number of the RMP. 
Moreover, the right-hand side of Eq.~(\ref{e1}) is evaluated
at the ``rational''  (i.e., resonant) magnetic flux-surface at which the
safety-factor
\begin{equation}
q({\mit\Psi}_p)= \frac{d{\mit\Psi}_t}{d{\mit\Psi}_p},
\end{equation}
takes the rational value $n/m$, where $m$ is a positive integer. Here, $m$ and $n$ are the numbers of poloidal and
toroidal periods, respectively, of the helical magnetic island chain driven at the rational surface.  Furthermore, ${\mit\Psi}_t({\mit\Psi}_p)$ is the equilibrium toroidal magnetic flux (divided by $2\pi$).

According to nonlinear tearing mode theory, a tearing mode is essentially convected by
the local ion fluid at the rational surface (again, see the discussion at the end of the Appendix).\cite{nl1,nl2,nl3}
In fact, if the response of an H-mode tokamak plasma to an applied RMP
is governed by nonlinear physics then we would expect the
natural frequency to take the form\,\cite{rf2}
\begin{equation}\label{e2}
\varpi_{\rm nonlinear} = -n\left(\omega_E +\left[1-L_{00}^{\,ii}+L_{01}^{\,ii}\left(\frac{\eta_i}{1+\eta_i}\right)\right]
\omega_{\ast\,i}
%\right.\nonumber\\[0.5ex]&\left.
-\left[L_{00}^{\,iI}- L_{01}^{\,iI}\left(\frac{\eta_I}{1+\eta_I}\right)\right]\omega_{\ast\,I}\right),
\end{equation}
where
\begin{align}
\omega_{\ast\,a}({\mit\Psi}_p) &=-\frac{T_a}{Z_a\,e}\,\frac{d\ln p_a}{d{\mit\Psi}_p},\\[0.5ex]
\eta_a({\mit\Psi}_p)&=\frac{d\ln T_a}{d\ln n_a},
\end{align}
for $a=i, I$. 
Here,  $Z_i$, $n_i({\mit\Psi}_p)$, $T_i({\mit\Psi}_p)$, and $p_i({\mit\Psi}_p)=n_i\,T_i$   are the charge number, equilibrium number density, 
equilibrium temperature, and equilibrium pressure
of the majority (thermal) ions, respectively,
whereas  $Z_I$, $n_I$, $T_I$, $p_I = n_I\,T_I$
are the corresponding quantities for the impurity ions.  
Furthermore, $L_{00}^{ii}({\mit\Psi}_p)$, $L_{01}^{ii}({\mit\Psi}_p)$,  $L_{00}^{II}({\mit\Psi}_p)$, and $L_{01}^{II}({\mit\Psi}_p)$ are neoclassical parameters that are defined
in Ref.~\onlinecite{rf2}. Note that these parameters are
affected by charge exchange  with neutrals. Finally,
as before, the right-hand side of Eq.~(\ref{e2}) is evaluated
at the rational magnetic flux-surface. 

A third possibility is that a tearing mode is convected by the local guiding center fluid at the rational surface,\cite{heyn,paz1}
in which case we would expect the natural frequency to take the form 
\begin{equation}\label{e3}
\varpi_{EB} =-n\,\omega_E.
\end{equation}
As before, the right-hand side of Eq.~(\ref{e3}) is evaluated
at the rational magnetic flux-surface. 

The secondary aim of this paper is to determine which of the three aforementioned choices for the natural frequency can best account for the   $q_{95}$ ELM
suppression windows  observed in DIII-D. 

\section{Description of Theoretical Model}
The theoretical model of the response of a tokamak plasma to an externally applied RMP that is employed in this
paper is described in detail in Ref.~\onlinecite{rf2}. The model employs a standard asymptotic matching approch.\cite{fkr,am1,am2,am3}
According to this approach, the response of the plasma to the applied RMP is governed by a combination of flux-freezing and
perturbed force balance (this combination is often referred to as ``marginally-stable ideal-MHD", which is a misnomer because MHD {\em per se}\/ plays no
role) everywhere in the plasma apart from a number of relatively narrow (in the radial direction) regions in which the applied
perturbation resonates with the equilibrium magnetic field. Magnetic reconnection can take place within the resonant regions to
produce relatively thin magnetic islands. Within the resonant regions, the plasma response is governed by nonlinear,
as opposed to linear, two-fluid resistive MHD. This is the case because the widths of the magnetic island chains
driven at the resonant surfaces exceed the linear layer widths (which invalidates linear theory).\cite{rf1}
Thus, when employing the asymptotic matching approach, the equations of flux-freezing and perturbed force balance
are solved in the so-called ``outer region'' that comprises most of the plasma (and the surrounding vacuum), the equations of 
nonlinear two-fluid resistive-MHD are solved in the various resonant layers that constitute the so-called ``inner region'', and the two
sets of solutions are then asymptotically matched to one another. 

A toroidal tokamak equilibrium exhibits two distinct types of response to an applied RMP.\cite{paz1,kink0,kink1}
The first of these is known as the ``tearing response''---this is a non-ideal-MHD response that is associated with the
formation of current sheets and magnetic island chains at various resonant surfaces within the plasma. The second
response type is known as the ``kink response''---this is an edge-localized ideal-MHD response that is associated
with coupling to a stable non-resonant kink mode. For the case of the tearing response, our model employs an
approximation in which the plasma response is assumed to be vacuum-like between the various resonant surfaces. On the other hand,
the kink response of the plasma is calculated using the {\tt GPEC} code.\cite{ipec,gpec}

Our model has been implemented in the {\tt EPEC} (Extended Perturbed Equilibrium Code) code. The name
of this code reflects the fact that the nonlinear evolution of tearing modes in a
tokamak plasma has far more in common with the $1\tfrac{1}{2}$-D
evolution of the global plasma equilibrium than it does with
conventional linear tearing mode physics. (See the discussion in the Appendix.) In particular, the Alfv\'{e}n time is
an irrelevant timescale in nonlinear tearing mode theory, and
is also  very much shorter than the timescales on which
physical quantities of interest actually evolve. (Note that all timescales are normalized to the Alfv\'{e}n time
in Ref.~\onlinecite{rf2}. However, this is just a matter of convention. With the benefit of hindsight, it would
have been better to normalize the timescales with respect to a diamagnetic timescale, in which case the Alfv\'{e}n time would have completely dropped
out of the final system of equations.)

\section{Modeling of DIII-D Discharge \#145380}

\section{Summary and Discussion}

\section*{Acknowledgements}
This research was funded by the U.S.\ Department of Energy under contract DE-FG02-04ER-54742.
The author would like to thank Q.M.~Hu and R.~Nazikian for providing the experimental data used in this paper. 
 
\section*{References}
\begin{thebibliography}{99}\baselineskip 5ex

\bibitem{wagner} F.~Wagner, G.~Becker, K.~Behringer, D.~Campbell, A.~Eberhagen, W.~Engelhardt, G.~Fussmann, O.~Gehre, J.~Gernhardt, G.~v.~Gierke, {\it et al.}, 
Phys.\ Rev.\ Lett.\ {\bf 49}, 1408 (1982).

\bibitem{zohm} H.~Zohm, Plasma Phys.\ Control.\ Fusion {\bf 38}, 105 (1996).

\bibitem{loarte} A.~Loarte, G.~Saibene, R.~Sartori, M.~B\'{e}coulet, L.~Horton, T.~Eich, A.~Herrmann, M.~Laux, G.~Matthews, S.~Jachmich, {\it et al.},  
J.\ Nucl.\ Materials {\bf 313}--{\bf 316}, 962 (2003).

\bibitem{evans}  T.E.~Evans, R.A.~Moyer, J.G.~Watkins, P.R.~Thomas, T.H.~Osborne, J.A.~Boedo, M.E.~Fenstermacher, K.H.~Finken, R.J.~Groebner, M.~Groth, {\it et al.},  
Phys.\ Rev.\ Lett.\ {\bf 92}, 235003 (2004).

\bibitem{jet} Y.~Liang,  H.R.~Koslowski, P.R.~Thomas, E.~Nardon, B.~Alper, P.~Andrew, Y.~Andrew, G.~Arnoux,  Y.~Baranov, M.~B\'{e}coulet, {\it et al.},  
Phys.\ Rev.\ Lett.\ {\bf 98}, 265004 (2007).

\bibitem{asdex} W.~Suttrop,  T.~Eich, J.C.~Fuchs, S.~G\"{u}nter, A.~Janzer, A.~Herrmann, A.~Kallenbach, P.T.~Lang, T.~Lunt, M.~Maraschek, {\it et al.},   
Phys.\ Rev.\ Lett.\ {\bf 106}, 225004 (2011).

\bibitem{kstar} Y.M.~Jeon,  J.-K.~Park, S.W.~Yoon, W.H.~Ko, S.G.~Lee, K.D.~Lee, G.S.~Yun, Y.U.~Nam, W.C.~Kim, J.-G.~Kwak, K.S.~Lee, H.K.~Kim, and H.L.~Yang, {\it et al.}, 
Phys.\ Rev.\ Lett.\ {\bf 109}, 035004 (2012).

\bibitem{mast} A.~Kirk, I.T.~Chapman, Y.~Liu, P.~Cahyna, P.~Denner, G.~Fishpool, C.J.~Ham, J.R.~Harrison, Y.~Liang, E.~Nardon, S.~Saarelma, R.~Scannell, A.J.~Thornton, and  MAST Team, 
Nucl.\  Fusion {\bf 53}, 043007 (2013).

\bibitem{east} T.~Sun,  Y.~Liang, Y.Q.~Liu, S.~Gu, X.~Yang, W.~Guo, T.~Shi, M.~Jia, L.~Wang, B.~Lyu, {\it et al.},  
Phys.\ Rev.\ Lett.\  {\bf 117}, 115001 (2016).

\bibitem{schmitz} O.~Schmitz, T.E.~Evans, M.E.~Fenstermacher, M.~Lehnen, H.~Stoschus, E.A.~Unterberg, J.W.~Coenen, H.~Frerichs, M.W.~Jakubowski, R.~Laengner, {\em et al.}, 
Nucl.\ Fusion {\bf 52}, 043005 (2012). 

\bibitem{lanctot} M.J.~Lanctot, R.J.~Buttery, J.S.~de\,Grassie, T.E.~Evans, N.M.~Ferraro, J.M.~Hanson, S.R.~Haskey,  R.A.~Moyer, R.~Nazikian, T.H.~Osborne, {\em et al.}, \\
Nucl.\ Fusion {\bf 53}, 083019 (2013). 

\bibitem{paz} C.~Paz-Solden, R.~Nazikian, S.R.~Haskey, N.C.~Logan, E.J.~Strait, N.M.~Ferraro, J.M.~Hanson, J.D.~King, M.J.~Lanctot, R.A.~Moyer, {\it et al.}, 
Phys.\ Rev.\ Lett.\ {\bf 114}, 105001 (2015).

\bibitem{d158115} R.~Nazikian, C.~Paz-Soldan, J.D.~Callen,  J.S.~de\,Grassie, D.~Eldon, T.E.~Evans, N.M.~Ferraro,  B.A.~Grierson, R.J.~Groebner, S.R.~Haskey, {\em et al.}, 
Phys.\ Rev.\ Lett.\ {\bf 114}, 105002 (2015).

\bibitem{paz1} C.~Paz-Solden, R.~Nazikian, L.~Cui, B.C.~Lyons,  D.M.~Orlov, A.~Kirk, N.C.~Logan, T.H.~Osborne, W.~Suttrop, and D.B.~Weisberg, 
Nucl.\ Fusion {\bf 59}, 056012 (2019). 

\bibitem{tm1} Q.~Yu, S.~G\"{u}nter,  and B.D.~Scott, Phys.\ Plasmas {\bf 10}, 797 (2003).

\bibitem{tm2} Q.~Yu, Nucl.\ Fusion {\bf 50}, 025014 (2010).

\bibitem{tm3} Q.~Yu, and S.~G\"unter,  Nucl.\ Fusion {\bf 51},  073030 (2011).

\bibitem{hu} Q.M.~Hu,  R.~Nazikian,  B.A.~Grierson,  N.C.~Logan,  J.-K.~Park,  C.~Paz-Soldan, and Q.~Yu,  Phys.\ Plasmas {\bf 26}, 120702 (2019).
 
\bibitem{connor} J.W.~Connor,  R.J.~Hastie, H.R.~Wilson, and R.L.~Miller, Phys.\ Plasmas  {\bf 5}, 2687 (1998).

 \bibitem{d3d} P.B.~Snyder, T.H.~Osboune, K.H.~Burrell, R.J.~Groebner, A.W.~Leonard, R.~Nazikian, D.M.~Orlov, O.~Schmitz, M.R.~Wade, and H.R.~Wilson, 
 Phys.\ Plasmas {\bf 19}, 056115 (2012).

\bibitem{rf1} R.~Fitzpatrick, Phys.\ Plasmas {\bf 27}, 042506 (2020).

\bibitem{rf2} R.~Fitzpatrick, and A.O.~Nelson, Phys.\ Plasmas {\bf 27}, 072501 (2020).

\bibitem{d3d2} Q.M.~Hu, R.~Nazikian, B.A.~Grierson, N.C.~Logan, D.M.~Orlov, C.~Paz-Solden, and Q.~Yu,
{\em The $q_{95}$ Windows of Edge-Localized-Mode Suppression using Resonant Magnetic Perturbations in the DIII-D Tokamak}, 
Phys.\ Rev.\ Lett., to appear (2020).

\bibitem{rfa} R.~Fitzpatrick, Nucl.\ Fusion {\bf 33}, 1049 (1993).

\bibitem{rfb} R.~Fitzpatrick, Phys.\ Plasmas {\bf 5}, 3325 (1998).

\bibitem{hender} R.~Fitzpatrick, and T.C.~Hender, Phys.\ Fluids B {\bf 3}, 644 (1991).

\bibitem{cole} A.~Cole, and R.~Fitzpatrick, Phys.\ Plasmas {\bf 13}, 032503 (2006).

\bibitem{lin1} M.~B\'{e}coulet, F.~Orain, P.~Maget, N.~Mellet, X.~Garbet, E.~Nardon, G.T.A.~Huysmans, T.~Caspar, A.~Loarte,  P.~Cayna, {\em et al.}, 
Nucl.\ Fusion {\bf 52}, 054003  (2012).

\bibitem{lin2} N.M.~Ferraro, Phys.\ Plasmas {\bf 19}, 056105  (2012).

\bibitem{lin3} F.~Orain, M.~B\'{e}coulet, G.~Dif-Pradalier, G.T.A.~Huysmans, S.~Pamela,   E.~Nardon, C.~Passeron, G.~Latu, V.~Grandgirard, A.~Fil, {\em et al.}, 
Phys.\ Plasmas {\bf 20}, 102510 (2013). 

\bibitem{nl1} R. Fitzpatrick, and F.L. Waelbroeck, Phys.\ Plasmas {\bf 12}, 022307 (2005).

\bibitem{nl2} R. Fitzpatrick, Phys.\ Plasmas {\bf 25}, 042503 (2018).

\bibitem{nl3} R. Fitzpatrick, Phys.\ Plasmas {\bf 25}, 112505 (2018).

\bibitem{heyn} M.F.~Heyn, I.B.~Ivanov, S.V.~Kasilov, W.~Kernbichler, I.~Joseph, R.A.~Moyer,  and A.M.~Runov, Nucl.\ Fusion {\bf 48}, 024005 (2008). 

\bibitem{fkr} H.P.~Furth,  J.~Killeen, and M.N.~Rosenbluth,  Phys.\ Fluids {\bf 6}, 459 (1963).

\bibitem{am1} R.~Fitzpatrick, R.J.~Hastie, T.J.~Martin, and C.M.~Roach, Nucl.\ Fusion {\bf 33}, 1533 (1993).

\bibitem{am2} A.H.~Glasser, Z.R.~Wang, anf J.-K.~Park, Phys.\ Plasmas {\bf 23}, 112506 (2016).

\bibitem{am3} R.~Fitzpatrick, Phys.\ Plasmas {\bf 24}, 072506 (2017). 

\bibitem{kink0} S.R.~Haskey, M.J.~Lanctot, Y.Q.~Liu, C.~Paz-Soldan, J.D.~King, B.D.~Blackwell, and O.~Schmitz,  Plasma 
Phys.\ Control.\ Fusion {\bf 57}, 025015 (2015).

\bibitem{kink1} D.A.~Ryan,  Y.Q.~Liu, A.~Kirk, W.~Suttrop, B.~Dudson, M.~Dunne, R.~Fischer, J.C.~Fuchs, M.~Garcia-Munoz, B.~Kurzan, {\it et al.}, 
Plasma Phys.\ Control.\ Fusion {\bf 57}, 095008 (2015).

\bibitem{ipec} J.~K.~Park, M.J.~Schaffer, J.E.~Menard, and A.H.~Boozer, Phys.\ Rev.\ Lett.\ {\bf 99}, 195003 (2007).

\bibitem{gpec} J.K.~Park, and N.C.~Logan, Phys.\ Plasmas {\bf 24}, 032505 (2017).

\bibitem{transp} R.J.~Hawryluk, {\em Physics of Plasma Close to Thermonuclear Conditions: Vol.~1}. (Commission of the European Communities, Brussels, 1980.) 
Internal Document DUR-FU-BRU-XII/476180.

\bibitem{haz} R.D.~Hazeltine, and J.D.~Meiss, Rev.\ Mod.\ Phys.\ {\bf 121}, 1 (1985).

\bibitem{ruth} P.H.~Rutherford,  Phys.\ Fluids  {\bf 16}, 1903 (1973).

\bibitem{wesson} J.A.~Wesson, private communication (1993).

\bibitem{rffig} R.~Fitzpatrick, Phys.\ Plasmas {\bf 10}, 2304 (2003).

\bibitem{ara} G.~Ara,  B.~Basu, B.~Coppi, G.~Laval, M.N.~Rosenbluth, and B.V.~Waddell, Ann.\ Phys.\ (N.Y.) {\bf 112}, 443 (1978). 

\bibitem{wat} R.~Fitzpatrick, P.G.~Watson, and F.L.~Waelbroeck, Phys.\ Plasmas {\bf 12}, 082510 (2005).

\end{thebibliography}

\appendix
\section{Linear Versus Nonlinear Tearing Mode Theory in Tokamak Plasmas}\label{appa}
The discussion in this Appendix outlines certain facts pertaining
to tearing modes in tokamak plasmas that ought to be common knowledge in the 
magnetic fusion community (but appear not to be). 

Consider a conventional tokamak plasma equilibrium. To lowest order, the equilibrium is governed by force balance,
\begin{equation}\label{ea}
-\nabla p + {\bf j}\times {\bf B} \simeq {\bf 0},
\end{equation}
and
flux freezing,
\begin{equation}\label{eb}
{\bf E} + {\bf V}\times {\bf B} \simeq {\bf 0}.
\end{equation}
This is the case because the other terms in the plasma equation of motion,
(\ref{ea}), and the plasma Ohm's law, (\ref{eb}), are much
smaller in magnitude than the retained terms. (Here, $p$ is the
scalar pressure, ${\bf j}$ the current density, ${\bf B}$ the
magnetic field strength, ${\bf E}$ the electric field strength, and ${\bf V}$ the plasma velocity). If we average the full plasma equation of motion and the full Ohm's law, as well as the full continuity and energy evolution equations, over magnetic flux-surfaces
then the dominant terms in these equations are annihilated, and
we end up with a set of $1\tfrac{1}{2}$-D evolution equations,
according to which the magnetic field and plasma current
evolve on the relatively long resistive timescale, $\tau_R$,
whereas the density and temperature evolve on a somewhat
shorter transport timescale.\cite{transp}

A linear tearing mode is a helical instability of a tokamak plasma equilibrium that is related to a shear-Alfv\'{e}n wave.\cite{haz} The mode resonates with the plasma at
the so-called rational magnetic flux-surface at which the
shear-Alfv\'{e}n velocity is zero. If $q$ is the safety-factor, $m$ the poloidal mode number, and $n$ the
toroidal mode number, then the rational
flux-surface is characterized by $q=m/n$. A thin resistive layer forms
around the rational surface that permits the reconnection of magnetic flux on a much faster timescale than the global
resistive evolution timescale, $\tau_R$, resulting in the formation of a helical magnetic island chain. Let the Alfv\'{e}n time,
$\tau_A$, be the
typical timescale on which a (compressional) Alfv\'{e}n
wave traverses the plasma.
In conventional tokamak plasmas, $\tau_A\ll \tau_R$. In fact,
the magnetic Lundquist number, $S\equiv \tau_R/\tau_A$,  typically exceeds $10^{\,8}$ in present-day tokamak plasmas,
and will likely exceed $10^{\,10}$ in ITER plasmas.
 It turns out that the width of the resistive layer is typically a factor $S^{\,-2/5}$ smaller than the
plasma minor radius, whereas magnetic flux is reconnected on
the hybrid timescale $\tau_A\ll \tau_A^{2/5}\,\tau_R^{3/5}\ll \tau_R$.\cite{fkr} Note that  the
structure of the tearing mode  outside the resistive layer is simply governed by a combination of flux-freezing and perturbed force balance. 

 Figure~\ref{fig14} sketches the typical plasma current and vorticity patterns in a linear tearing layer. Observe that there is
 no distinction between a magnetic X-point and a magnetic 
O-point. The reason for this is that the magnetic separatrix
of the reconnected magnetic island does not present an obstacle
to plasma motion because the linear layer is so thin that the
plasma can diffuse resistively across magnetic field-lines very rapidly. However, this is only the case as long
as the radial width of the magnetic island chain is less than the linear layer width. As soon as magnetic reconnection
at the rational surface has proceeded to such an extent that
the magnetic island width exceeds the layer width then we enter the nonlinear regime. 

Figure~\ref{fig14} also sketches the
typical plasma current and vorticity patterns in the nonlinear regime. Observe that there is now a considerable distinction between an X-point and an O-point. The reason for this is that the island is sufficiently wide that the plasma cannot diffuse across it resistively sufficiently rapidly to
avoid being trapped inside the magnetic separatrix. Indeed, both the regions inside and outside the magnetic separatrix
are governed by a combination of flux-freezing and perturbed force balance, whereas the resistive
layer, within which the plasma can easily slip through the magnetic field,  is diverted onto the magnetic separatrix. Moreover, as
the island grows, the width of the separatrix layer becomes
increasing negligible with respect to the island width. It
turns out that the separatrix layer does not significantly affect the evolution of a nonlinear magnetic island chain. Indeed,
the celebrated Rutherford island width evolution equation,
which governs the evolution of the island width, is simply a
$1\tfrac{1}{2}$-D evolution equation for a helical magnetic
equilibrium localized around the rational surface.\cite{ruth}
The Alfv\'{e}n time plays no role in this evolution (any more
than it plays a role in the $1\tfrac{1}{2}$-D evolution of
the global magnetic field). Likewise, the equation that
determines the phase velocity of a nonlinear magnetic island chain (see Ref.~\onlinecite{nl2}, and references therein) is essentially an
expression of the $1\tfrac{1}{2}$-D evolution of the plasma
flow profiles across the island region. As before, the  Alfv\'{e}n time plays no role in this evolution (any more
than it plays a role in the $1\tfrac{1}{2}$-D evolution of
the global plasma flow profiles). 

It follows, from the previous discussion, that the nonlinear evolution of a tearing mode in a tokamak plasma has far more in common with the $1\tfrac{1}{2}$-D
evolution of the global plasma equilibrium than it does with
conventional linear tearing mode physics. In particular, the Alfv\'{e}n time is
an irrelevant timescale in nonlinear tearing mode theory, and
is also  very much shorter than the timescales on which
physical quantities of interest actually evolve. It should be noted that linear tearing layers in present-day tokamak plasmas are sufficiently thin that by the time a tearing mode is detectable it is already in the nonlinear regime (i.e., the island width exceeds the linear layer width).\cite{wesson} It, therefore, makes very little sense to
attempt to model tearing mode evolution in tokamak plasmas using a toroidal nonlinear magnetohydrodynamical (MHD) code. The reason for this is obvious: employing  an MHD code necessarily introduces the very short Alfv\'{e}n time into the calculation,
but this timescale is actually irrelevant to the problem in hand. In fact, this is exactly the same reason that MHD codes are not usually used to reconstruct global plasma equilibria, or to model their time evolution.

The natural frequency of a (stable) tearing mode is
the helical phase velocity with which it would propagate
were it actually unstable.\cite{rfa} The natural frequency
is determined by the equilibrium plasma flow at the
rational surface. Now, a magnetic island is a helical
pattern in the magnetic field  generated by a helical current perturbation that is localized in the
vicinity of the rational surface.
Given that plasma current is predominately carried by the
electrons, it is natural to suppose that a magnetic
island chain (as well as the tearing mode perturbation
away from the rational surface)  is convected by the
electron fluid in the immediate vicinity of the rational
surface. This is indeed the case in the linear regime.\cite{ara} Of course, as a consequence of diamagnetic flows, if the island chain is
convected by the electron fluid at the rational surface then it propagates with respect to the local ion fluid. However, this is not a
problem because a linear layer is sufficiently thin that the magnetic field can diffuse through the plasma very rapidly, which implies that
the ion fluid is not tied to the magnetic structure of the island chain. The situation is very different in the nonlinear regime. 
As we have seen, the region inside the magnetic separatrix of a nonlinear magnetic island chain is governed by a combination of flux-freezing and perturbed force balance.
This implies that both the electron and the ion fluids are trapped inside the separatrix, and are, therefore,  forced to co-rotate with the island chain. There
is no such constraint outside the separatrix, so the electron and ion fluids rotate at different speeds in this region,
as a consequence of diamagnetism.  Consequently, one or other of the electron and the ion fluid rotation profiles must exhibit a strong gradient across
the separatrix. The island propagation velocity is determined by which of the two fluids is most resistant to the formation of such a gradient. 
Of course, it is the ion fluid which is more resistant because of its much greater perpendicular viscosity,\cite{nl1,wat} as well as its much larger 
neoclassical stress tensor.\cite{nl2} Hence, a nonlinear magnetic island chain is convected by the ion fluid in the vicinity of the rational
surface, because this choice of propagation speed minimizes the ion fluid velocity gradient across the separatrix.

\newpage

\begin{figure}
\includegraphics[height=5in]{fig1.pdf}
\caption{Overview of DIII-D discharge \#145380.
(a) Safety factor at  ${\mit\Psi}_N=0.95$. 
(b) $D_\alpha$ (i.e., Deuterium Balmer-alpha) signal, as well as $n=3$ current flowing in upper and lower sections of I-coil. 
(c) Pedestal (i.e., ${\mit\Psi}_N=0.94$)
electron pressure. (The red curve is the running average over 10 ms.) (d) Pedestal electron number density. The common vertical yellow bands indicate the ELM suppression windows.}\label{fig1}
\end{figure}

\begin{figure}
\includegraphics[height=5in]{fig2.pdf}
\caption{Left Panel: Contours of the equilibrium poloidal
magnetic flux in   DIII-D discharge \#145380 at time $t=2500$ ms. The scale major radius is $R_0=1.70\,{\rm m}$. The white dot indicates the magnetic axis, the white curve indicates the
last closed magnetic flux-surface, and the thick black line indicates the limiter. 
Upper-Right Panel: Safety-factor profile in DIII-D discharge \#145380 at time $t=2500$ ms. Lower-Right Panel:  Total plasma pressure profile in DIII-D discharge \#145380 at time $t=2500$ ms. } \label{fig2}
\end{figure}

\begin{figure}
\includegraphics[height=7in]{fig3.pdf}
\caption{Top Panel: The green, red, blue, and cyan curves show the electron number density ($10^{19}\,{\rm m}^{-3}$),
electron temperature (keV), (thermal) ion temperature (keV), and
 C-VI ion number density  ($10^{18}\,{\rm m}^{-3}$),  profiles, respectively,  in  DIII-D discharge \#145380 at time $t=2500$ ms. Middle Panel:  ${\bf E}\times {\bf B}$ frequency profile in  DIII-D discharge \#145380 at time $t=2500$. Bottom Panel: Perpendicular momentum diffusivity 
profile in  DIII-D discharge \#145380 at time $t=2500$ ms.
The   common vertical dotted lines indicate the location of the top
of the pedestal, ${\mit\Psi}_N=0.925$.} \label{fig3}
\end{figure}

\begin{figure}
\includegraphics[height=4in]{fig4.pdf}
\caption{Safety-factors as functions of time in  DIII-D discharge \#145380. The red, green, blue, and black curves show the safety-factors
at the magnetic axis (${\mit\Psi}_N=0.00$),
the $95\%$ flux surface (${\mit\Psi}_N=0.950$), the
effective plasma boundary for the GPEC and EPEC calculations
(${\mit\Psi}_N=0.997$), and the true plasma boundary (${\mit\Psi}_N=1.00$), respectively. The horizontal dotted lines indicate the safety-factors at the various $n=3$ rational surfaces. 
} \label{fig4}
\end{figure}

\begin{figure}
\includegraphics[height=6in]{fig5.pdf}
\caption{Top Panel: $n=3$ natural frequencies as functions of time
in   DIII-D discharge \#145380, assuming that the natural frequency is determined by nonlinear island physics.
Bottom Panel:  $n=3$ island frequencies as functions of time
in   DIII-D discharge \#145380, assuming that the natural frequency is determined by nonlinear island physics. The red, green, blue, yellow, cyan, magenta, brown, pink,
purple, and orange  curves correspond to $m=5$, 6, 7, 8, 9, 10, 11, 12, 13, and 14, respectively. The yellow vertical bands indicate the ELM suppression windows.} \label{fig5}
\end{figure}

\begin{figure}
\includegraphics[height=6in]{fig6.pdf}
\caption{Top Panel: Full  $n=3$ vacuum island widths as functions of time
in   DIII-D discharge \#145380.
Bottom Panel:  Full $n=3$ island widths as functions of time
in   DIII-D discharge \#145380, assuming that the natural frequency is determined by nonlinear island physics. The blue, yellow, cyan, magenta, brown, pink,
purple, and orange  areas correspond to $m=7$, 8, 9, 10, 11, 12, 13, and 14, respectively. The yellow vertical bands indicate the ELM suppression windows. The horizontal dotted lines indicate the top of the pedestal, ${\mit\Psi}_N=0.925$.} \label{fig6}
\end{figure}

\begin{figure}
\includegraphics[height=6in]{fig7.pdf}
\caption{Top Panel: Density flattening widths associated with induced $n=3$ magnetic island  chains as functions of time
in   DIII-D discharge \#145380, assuming that the natural frequency is determined by nonlinear island physics.
Bottom Panel:  Electron temperature flattening widths associated with induced $n=3$ magnetic island chains as functions of time
in   DIII-D discharge \#145380, assuming that the natural frequency is determined by nonlinear island physics. The blue, yellow, cyan, magenta, brown, pink,
purple, and orange  areas correspond to $m=7$, 8, 9, 10, 11, 12, 13, and 14, respectively. The yellow vertical bands indicate the ELM suppression windows. The  horizontal dotted lines indicate the top of the pedestal, ${\mit\Psi}_N=0.925$.} \label{fig7}
\end{figure}

\begin{figure}
\includegraphics[height=6in]{fig8.pdf}
\caption{Top Panel: $n=3$ natural frequencies as functions of time
in   DIII-D discharge \#145380, assuming that the natural frequency is coincident with the ${\bf E}\times {\bf B}$
frequency.
Bottom Panel:  $n=3$ island frequencies as functions of time
in   DIII-D discharge \#145380, assuming that the natural frequency is  coincident with the ${\bf E}\times {\bf B}$
frequency. The red, green, blue, yellow, cyan, magenta, brown, pink,
purple, and orange  curves correspond to $m=5$, 6, 7, 8, 9, 10, 11, 12, 13, and 14, respectively. The yellow vertical bands indicate the ELM suppression windows.} \label{fig8}
\end{figure}

\begin{figure}
\includegraphics[height=6in]{fig9.pdf}
\caption{Top Panel: $n=3$ vacuum island widths as functions of time
in   DIII-D discharge \#145380.
Bottom Panel:  $n=3$ island widths as functions of time
in   DIII-D discharge \#145380, assuming that the natural frequency is  coincident with the ${\bf E}\times {\bf B}$
frequency. The blue, yellow, cyan, magenta, brown, pink,
purple, and orange  areas correspond to $m=7$, 8, 9, 10, 11, 12, 13, and 14, respectively. The yellow vertical bands indicate the ELM suppression windows. The horizontal dotted lines indicate the top of the pedestal, ${\mit\Psi}_N=0.925$.} \label{fig9}
\end{figure}

\begin{figure}
\includegraphics[height=6in]{fig10.pdf}
\caption{Top Panel: Density flattening widths associated with induced $n=3$ magnetic island  chains as functions of time
in   DIII-D discharge \#145380, assuming that the natural frequency is  coincident with the ${\bf E}\times {\bf B}$
frequency.
Bottom Panel:  Electron temperature flattening widths associated with induced $n=3$ magnetic island chains as functions of time
in   DIII-D discharge \#145380, assuming that the natural frequency is  coincident with the ${\bf E}\times {\bf B}$
frequency. The blue, yellow, cyan, magenta, brown, pink,
purple, and orange  areas correspond to $m=7$, 8, 9, 10, 11, 12, 13, and 14, respectively. The yellow vertical bands indicate the ELM suppression windows. The horizontal dotted lines indicate the top of the pedestal, ${\mit\Psi}_N=0.925$. } \label{fig10}
\end{figure}

\begin{figure}
\includegraphics[height=6in]{fig11.pdf}
\caption{Top Panel: $n=3$ natural frequencies as functions of time
in   DIII-D discharge \#145380, assuming that the natural frequency is determined by linear layer physics.
Bottom Panel:  $n=3$ island frequencies as functions of time
in   DIII-D discharge \#145380, assuming that the natural frequency is determined by linear layer physics. The black, red, green, blue, yellow, cyan, magenta,  brown, purple, and orange curves correspond to $m=4$, 5, 6, 7, 8, 9, 10,  11, 13, and 14, respectively. The yellow vertical bands indicate the ELM suppression windows. } \label{fig11}
\end{figure}

\begin{figure}
\includegraphics[height=6in]{fig12.pdf}
\caption{Top Panel: $n=3$ vacuum island widths as functions of time
in   DIII-D discharge \#145380.
Bottom Panel:  $n=3$ island widths as functions of time
in   DIII-D discharge \#145380, assuming that the natural frequency is determined by linear layer physics. The blue, yellow, cyan, magenta, brown, pink,
purple, and orange  areas correspond to $m=7$, 8, 9, 10, 11, 12, 13, and 14, respectively. The yellow vertical bands indicate the ELM suppression windows. The horizontal dotted lines indicate the top of the pedestal, ${\mit\Psi}_N=0.925$.} \label{fig12}
\end{figure}

\begin{figure}
\includegraphics[height=6in]{fig13.pdf}
\caption{Top Panel: Density flattening widths associated with induced $n=3$ magnetic island  chains as functions of time
in   DIII-D discharge \#145380, assuming that the natural frequency is determined by linear layer physics.
Bottom Panel:  Electron temperature flattening widths associated with induced $n=3$ magnetic island chains as functions of time
in   DIII-D discharge \#145380, assuming that the natural frequency is determined by linear layer physics. The blue, yellow, cyan, magenta, brown, pink,
purple, and orange  areas correspond to $m=7$, 8, 9, 10, 11, 12, 13, and 14, respectively. The yellow vertical bands indicate the ELM suppression windows. The horizontal dotted lines indicate  the top of the pedestal, ${\mit\Psi}_N=0.925$.} \label{fig13}
\end{figure}

\begin{figure}
\includegraphics[height=4in]{fig14.pdf}
\caption{Schematic diagram showing the perturbed plasma current and vorticity patterns around the magnetic X- and O-points of a linear and a nonlinear tearing mode. The horizontal axis measures radial distance from the rational
surface, whereas the vertical axis measures distance along equilibrium magnetic field-lines. Red and blue
correspond to positive and negative current/vorticity values, respectively. (See Figs.~7 and 10 in 
Ref.~\onlinecite{rffig}.)}\label{fig14}
\end{figure}


\end{document}

